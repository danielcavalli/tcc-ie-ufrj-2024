% Options for packages loaded elsewhere
\PassOptionsToPackage{unicode}{hyperref}
\PassOptionsToPackage{hyphens}{url}
\documentclass[
  brazilian,
  12pt,
  a4paper,
]{article}
\usepackage{xcolor}
\usepackage[margin=1in]{geometry}
\usepackage{amsmath,amssymb}
\setcounter{secnumdepth}{5}
\usepackage{iftex}
\ifPDFTeX
  \usepackage[T1]{fontenc}
  \usepackage[utf8]{inputenc}
  \usepackage{textcomp} % provide euro and other symbols
\else % if luatex or xetex
  \usepackage{unicode-math} % this also loads fontspec
  \defaultfontfeatures{Scale=MatchLowercase}
  \defaultfontfeatures[\rmfamily]{Ligatures=TeX,Scale=1}
\fi
\usepackage{lmodern}
\ifPDFTeX\else
  % xetex/luatex font selection
\fi
% Use upquote if available, for straight quotes in verbatim environments
\IfFileExists{upquote.sty}{\usepackage{upquote}}{}
\IfFileExists{microtype.sty}{% use microtype if available
  \usepackage[]{microtype}
  \UseMicrotypeSet[protrusion]{basicmath} % disable protrusion for tt fonts
}{}
\makeatletter
\@ifundefined{KOMAClassName}{% if non-KOMA class
  \IfFileExists{parskip.sty}{%
    \usepackage{parskip}
  }{% else
    \setlength{\parindent}{0pt}
    \setlength{\parskip}{6pt plus 2pt minus 1pt}}
}{% if KOMA class
  \KOMAoptions{parskip=half}}
\makeatother
\usepackage{longtable,booktabs,array}
\usepackage{calc} % for calculating minipage widths
% Correct order of tables after \paragraph or \subparagraph
\usepackage{etoolbox}
\makeatletter
\patchcmd\longtable{\par}{\if@noskipsec\mbox{}\fi\par}{}{}
\makeatother
% Allow footnotes in longtable head/foot
\IfFileExists{footnotehyper.sty}{\usepackage{footnotehyper}}{\usepackage{footnote}}
\makesavenoteenv{longtable}
\ifLuaTeX
\usepackage[bidi=basic,shorthands=off,]{babel}
\else
\usepackage[bidi=default,shorthands=off,]{babel}
\fi
\ifLuaTeX
  \usepackage{selnolig} % disable illegal ligatures
\fi
\setlength{\emergencystretch}{3em} % prevent overfull lines
\providecommand{\tightlist}{%
  \setlength{\itemsep}{0pt}\setlength{\parskip}{0pt}}
\usepackage{bookmark}
\IfFileExists{xurl.sty}{\usepackage{xurl}}{} % add URL line breaks if available
\urlstyle{same}
\hypersetup{
  pdflang={pt-BR},
  hidelinks,
  pdfcreator={LaTeX via pandoc}}

\author{}
\date{}

\begin{document}

{
\setcounter{tocdepth}{3}
\tableofcontents
}
UNIVERSIDADE FEDERAL DO RIO DE JANEIRO\\
INSTITUTO DE ECONOMIA

\textbf{Impacto Localizado de Novas Estações Meteorológicas na
Produtividade Agrícola: Uma Abordagem de Tratamento Deslocado e
Emparelhamento Dinâmico}

DANIEL CAVALLI\\
DRE 117038439

PROFESSOR: Prof º. Eduardo Pontual Ribeiro

JULHO 2025\\
\href{https://github.com/danielcavalli/tcc-ie-ufrj-2024}{ACESSO AO
CÓDIGO}

\section{Resumo}\label{resumo}

Este estudo examina o impacto causal da instalação de estações
meteorológicas automáticas sobre a produtividade da cana-de-açúcar no
Brasil, utilizando dados em painel de 394 microrregiões produtoras de
cana-de-açúcar entre 2000 e 2021. Empregamos o arcabouço de Diferenças
em Diferenças com adoção escalonada proposto por Callaway \& Sant'Anna
(2020), adequado para contextos onde o tratamento ocorre em diferentes
momentos do tempo. A estratégia de identificação explora a variação
temporal e geográfica na instalação de estações, com 98\% das
microrregiões permanecendo como controles não tratados.

Os resultados principais, obtidos através do estimador doubly robust,
indicam um efeito médio do tratamento (ATT) de 10.9\% (IC 95\%:
{[}6.0\%; 15.7\%{]}, p \textless{} 0.001), representando ganhos de
produtividade economicamente significativos. A análise de event study
revela ausência de tendências pré-tratamento diferenciadas (validando o
pressuposto de identificação) e uma dinâmica pós-tratamento crescente,
com efeitos que se intensificam ao longo do tempo, alcançando 20\% após
cinco anos de exposição.

Testes extensivos de robustez, incluindo placebos temporais e
aleatórios, especificações alternativas, e análise de sensibilidade a
diferentes grupos de controle, confirmam a consistência dos resultados.
A magnitude estimada sugere que o investimento em infraestrutura
meteorológica oferece retornos substanciais, com implicações importantes
para políticas de desenvolvimento agrícola e adaptação climática. O
estudo contribui para a literatura sobre tecnologia agrícola e
informação, demonstrando empiricamente como dados meteorológicos
precisos podem melhorar significativamente a eficiência produtiva no
setor agrícola.

\section{\texorpdfstring{\textbf{1}.
Introdução}{1. Introdução}}\label{introduuxe7uxe3o}

A produtividade agrícola é profundamente influenciada pelas condições
meteorológicas, que moldam tanto o desenvolvimento fisiológico das
culturas quanto a eficácia das práticas de manejo. Monteiro et
al.~(2009) destacam que a variabilidade da produção agrícola global é
amplamente explicada pelas oscilações climáticas durante o ciclo de
cultivo, o que reforça a importância de sistemas robustos de
monitoramento climático. Nesse contexto, a agrometeorologia surge como
uma ferramenta estratégica, permitindo a integração de dados
meteorológicos e agrícolas para apoiar decisões mais eficientes e
sustentáveis. Diante da pressão crescente por maior produção de
alimentos e energia renovável, sem ampliar o uso de recursos naturais,
torna-se essencial adotar instrumentos capazes de reduzir riscos
climáticos e ampliar a resiliência produtiva do setor agrícola.

\subsection{\texorpdfstring{\textbf{1.1 O papel da agrometeorologia na
produtividade
agrícola}}{1.1 O papel da agrometeorologia na produtividade agrícola}}\label{o-papel-da-agrometeorologia-na-produtividade-agruxedcola}

A agrometeorologia desempenha um papel crucial na agricultura ao
fornecer informações meteorológicas aplicadas diretamente às
necessidades dos cultivos. Esse campo integra dados climáticos e
meteorológicos com parâmetros específicos das culturas, permitindo a
antecipação dos efeitos do clima sobre as práticas agrícolas e
possibilitando decisões mais informadas e eficientes. Como apontado por
Rijks e Baradas (2000), os Serviços Nacionais de Meteorologia contribuem
significativamente para a economia agrícola ao divulgar essas
informações e facilitar seu uso eficiente, ajudando a mitigar riscos e
aumentar a produtividade.

Segundo Mavi e Tupper (2004), as informações agrometeorológicas são
aplicáveis em três áreas principais: no planejamento agrícola, na tomada
de decisões táticas e na resiliência dos sistemas agrícolas. No
planejamento, esses dados ajudam na escolha das épocas e locais mais
adequados para o cultivo, considerando o macroclima e as condições
específicas de cada região. Essa etapa é essencial para ajustar as
atividades agrícolas ao contexto climático regional, reduzindo
desperdícios e promovendo o uso sustentável dos recursos.

No contexto tático, as informações meteorológicas auxiliam na
determinação dos melhores momentos para práticas agrícolas como a
irrigação, semeadura e colheita, o que contribui para uma execução mais
precisa e eficiente das operações. Essas decisões são ainda mais
importantes em áreas de cultivo de sequeiro, onde a dependência da
precipitação é alta, e os dados sobre previsão de chuva e
evapotranspiração são fundamentais para otimizar o uso dos recursos
hídricos (Pereira et al., 2002).

Com a instalação de novas estações meteorológicas em áreas rurais, as
informações climáticas se tornam mais precisas e localizadas, permitindo
que os agricultores ajustem suas práticas de acordo com as condições
específicas de suas regiões(Weiss et al., 2000). Sistemas de Informação
Agrometeorológica, como o AGRITEMPO da EMBRAPA e o SISDAGRO do INMET,
utilizam dados dessas estações meteorológicas para fornecer previsões de
safra e orientações sobre manejo de recursos hídricos, ajudando os
agricultores a tomar decisões informadas sobre a época de plantio,
irrigação e controle fitossanitário (Weiss et al., 2000). Essas
informações são cruciais para o presente estudo, pois evidenciam que
dados meteorológicos detalhados e acessíveis podem impactar diretamente
a produtividade agrícola.

\subsection{\texorpdfstring{\textbf{1.2 O papel da
econometria}}{1.2 O papel da econometria}}\label{o-papel-da-econometria}

Apesar da importância reconhecida da agrometeorologia, são escassos os
estudos que quantificam empiricamente o impacto da instalação de novas
estações sobre a produtividade agrícola. O desafio metodológico central
é que a instalação das estações ocorre de forma escalonada ao longo do
tempo, o que inviabiliza o uso simples do modelo clássico de Diferenças
em Diferenças (DiD) com Two-Way Fixed Effects (TWFE). Trabalhos como
Goodman-Bacon (2019) e Sun \& Abraham (2020) mostraram que, nesses
contextos, o estimador TWFE pode produzir resultados enviesados, pois
utiliza unidades já tratadas como controles e mistura heterogeneidades
de efeito ao longo do tempo.

Para superar essas limitações, este estudo adota o arcabouço
desenvolvido por Callaway \& Sant'Anna (2020, 2021), que propõem um
estimador de Diferenças em Diferenças robusto para múltiplos períodos de
tratamento. Esse método calcula efeitos médios específicos a cada coorte
de adoção (ATT(g,t)), utilizando apenas grupos de comparação válidos, e
depois os agrega segundo esquemas de ponderação coerentes com a pergunta
empírica. Além disso, permite construir análises dinâmicas em tempo
relativo (event studies), avaliando tanto o surgimento quanto a
persistência dos efeitos ao longo do tempo, ao mesmo tempo em que
possibilita a verificação do pressuposto de tendências paralelas no
pré-tratamento.

\subsection{}\label{section}

\subsection{\texorpdfstring{\textbf{2.
Objetivos}}{2. Objetivos}}\label{objetivos}

O objetivo geral deste trabalho é estimar o efeito causal da instalação
de novas estações meteorológicas sobre a produtividade da cana-de-açúcar
no Brasil, utilizando o arcabouço de Diferenças em Diferenças com
múltiplos períodos proposto por Callaway \& Sant'Anna (2020, 2021). Ao
empregar esse método, busca-se superar as limitações dos modelos
tradicionais de Two-Way Fixed Effects (TWFE) em cenários de adoção
escalonada, assegurando estimativas consistentes mesmo na presença de
heterogeneidade de efeitos ao longo do tempo e entre unidades.

De forma mais específica, este estudo tem como objetivos:

\textbf{(i)} quantificar o efeito médio do tratamento sobre os tratados
(ATT) associado à instalação das estações meteorológicas, em termos de
variação percentual da produtividade da cana-de-açúcar;\\
\textbf{(ii)} estimar a dinâmica temporal do impacto, por meio de
análises de evento (event studies), identificando tanto a ausência de
tendências prévias quanto a evolução dos efeitos após a instalação;\\
\textbf{(iii)} avaliar a robustez dos resultados por meio de
especificações alternativas de estimação (Doubly Robust, IPW, Outcome
Regression), bem como por testes de placebo que buscam identificar
falsos efeitos em períodos fictícios;\\
\textbf{(iv)} discutir as implicações dos resultados para a formulação
de políticas públicas voltadas à expansão da infraestrutura
meteorológica, destacando sua relevância para a resiliência agrícola e o
desenvolvimento sustentável do setor.

A hipótese central que guia este trabalho é que a instalação de estações
meteorológicas gera ganhos de produtividade agrícola mensuráveis,
especialmente em regiões próximas às novas instalações, onde a qualidade
e a precisão das informações climáticas disponibilizadas são mais
elevadas. Caso confirmada, essa evidência reforça a importância da
agrometeorologia como instrumento de política pública, capaz de promover
eficiência produtiva e adaptação frente à variabilidade climática.

\subsubsection{\texorpdfstring{\textbf{3. Referencial
Teórico}}{3. Referencial Teórico}}\label{referencial-teuxf3rico}

A relação entre clima e produtividade agrícola é amplamente documentada
na literatura. Estudos como Monteiro et al.~(2009) mostram que grande
parte da variabilidade da produção agrícola mundial decorre de
flutuações climáticas durante o ciclo produtivo, o que torna a gestão da
informação meteorológica um fator estratégico. Nesse sentido, a
agrometeorologia se consolida como ciência aplicada, integrando dados
meteorológicos às práticas agrícolas com o objetivo de reduzir riscos e
aumentar eficiência produtiva. Como apontado por Rijks e Baradas (2000),
serviços meteorológicos bem estruturados podem gerar ganhos econômicos
significativos ao disponibilizar informações precisas para produtores.
Mavi e Tupper (2004) destacam que essas informações podem orientar tanto
o planejamento de longo prazo quanto decisões táticas de curto prazo,
além de aumentar a resiliência dos sistemas agrícolas frente à
variabilidade climática.

No Brasil, sistemas como o AGRITEMPO da EMBRAPA e o SISDAGRO do INMET já
demonstram como a expansão da rede de estações meteorológicas pode ser
traduzida em recomendações práticas de manejo. Weiss et al.~(2000)
reforçam que dados mais granulares permitem previsões de safra mais
acuradas e orientações localizadas sobre irrigação e controle
fitossanitário. Essa literatura estabelece o vínculo entre maior
densidade de informações meteorológicas e maior produtividade agrícola,
mas ainda são escassos os trabalhos que estimam de forma causal o efeito
da instalação de estações sobre a produção.

Para enfrentar essas limitações dos modelos TWFE, Callaway \& Sant'Anna
(2020, 2021) desenvolveram um framework alternativo que redefine a
identificação em modelos de Diferenças em Diferenças com múltiplos
períodos. O método propõe a estimação de efeitos médios específicos por
coorte de adoção e período (ATT(g,t)), sempre utilizando grupos de
comparação válidos, seguidos de uma agregação coerente com pesos bem
definidos. Além disso, o arcabouço permite estimar a trajetória dinâmica
do efeito em tempo relativo (event studies), o que possibilita avaliar
tanto o surgimento quanto a persistência dos impactos, além de testar
empiricamente a plausibilidade do pressuposto de tendências paralelas
condicionais.

A abordagem de Diferenças em Diferenças (DiD) proposta por Callaway \&
Sant'Anna (2020) oferece um arcabouço flexível para analisar situações
em que unidades recebem um tratamento ao longo de múltiplos períodos, e
não todas no mesmo instante. Diferente do DiD comum, que se concentra
tradicionalmente em apenas dois períodos (pré e pós-tratamento) e dois
grupos (tratado e controle), a extensão proposta considera diversos
grupos que podem iniciar o tratamento em momentos distintos, bem como
múltiplos períodos de observação. Esse modelo torna-se particularmente
relevante em estudos que avaliam o impacto de intervenções ou choques
econômicos que ocorram de forma escalonada (staggered adoption),
permitindo lidar melhor com a heterogeneidade entre grupos e com
problemas de interpretação associados ao uso de modelos de efeitos fixos
bidimensionais (Two-Way Fixed Effects, TWFE).

Essa abordagem tem sido rapidamente incorporada em estudos aplicados em
diversas áreas, como mercado de trabalho, políticas sociais e educação,
mas ainda é pouco explorada em pesquisas sobre agricultura e clima. Ao
adotar esse framework em um setor de relevância estratégica como o
agronegócio brasileiro, este trabalho contribui não apenas
empiricamente, ao estimar o impacto da instalação de estações
meteorológicas sobre a produtividade da cana-de-açúcar, mas também
metodologicamente, ao ilustrar a aplicabilidade de um dos métodos mais
recentes e robustos de avaliação de impacto em políticas públicas.

\subsection{\texorpdfstring{\textbf{4. Especificação do
Modelo}}{4. Especificação do Modelo}}\label{especificauxe7uxe3o-do-modelo}

Para este trabalho, utilizaremos como principal referência o artigo de
Callaway \& Sant'Anna (2020), que apresenta uma extensão do modelo de
Diferenças em Diferenças (DiD) para cenários com múltiplos períodos e
momentos distintos de adoção do tratamento.

\subsubsection{\texorpdfstring{\textbf{4.1 Introdução ao
Modelo}}{4.1 Introdução ao Modelo}}\label{introduuxe7uxe3o-ao-modelo}

No DiD clássico, assume-se um grupo tratado que recebe a intervenção em
um momento específico e um grupo controle que nunca é tratado. Sob essa
configuração, a diferença no tempo entre pré e pós-tratamento e a
diferença entre grupos tratado e controle fornecem a estimativa do
efeito causal. Entretanto, para o caso que estamos tratando nesse
trabalho existem múltiplos períodos e vários grupos recebendo o
tratamento em momentos distintos ao longo dos 22 anos do período de
tratamento. A abordagem de DiD tradicional, nesse caso, pode gerar
estimativas enviesadas devido à heterogeneidade do tratamento ao longo
do tempo, resultando em interpretação ambígua.\\
O modelo de Callaway \& Sant'Anna (2020) surge como uma forma de
permitir que esses cenários de tratamento escalonado, talvez muito mais
comuns no mundo real do que experimentos naturais, possam ser avaliados.
Por permitir a identificação de efeitos médios do tratamento específicos
para cada grupo e período, acomoda a heterogeneidade do momento de
adoção e suas dinâmicas, além de fornecer uma interpretação mais clara
dos parâmetros causais.

\subsubsection{\texorpdfstring{\textbf{4.2 Fundamentos do
modelo}}{4.2 Fundamentos do modelo}}\label{fundamentos-do-modelo}

O modelo proposto pode ser entendido em três etapas conceituais:

\begin{enumerate}
\def\labelenumi{\arabic{enumi}.}
\tightlist
\item
  \textbf{Identificação de parâmetros causais desagregados:} Primeiro,
  são obtidas estimativas do efeito causal para cada combinação de grupo
  tratado e período após a adoção (denotados por ATT(g,t)), focando em
  captar o efeito específico para um determinado conjunto de unidades
  tratadas em um dado momento do tempo.\\
\item
  \textbf{Agregação desses parâmetros:} Em seguida, esses parâmetros
  individuais, definidos para grupos e períodos específicos, podem ser
  combinados para produzir medidas resumidas de efeitos, como efeitos
  médios globais, ao longo do tempo, por coorte de tratamento ou segundo
  o tempo decorrido desde a intervenção.\\
\item
  \textbf{Estimação e inferência:} Por fim, procedimentos estatísticos
  são empregados para estimar esses parâmetros, bem como inferir sobre
  sua significância estatística.
\end{enumerate}

\paragraph{\texorpdfstring{\textbf{4.2.1 Group-Time Average Treatment
Effects
ATT(g,t)}}{4.2.1 Group-Time Average Treatment Effects ATT(g,t)}}\label{group-time-average-treatment-effects-attgt}

O parâmetro fundamental dessa abordagem é o ATT(g,t), que representa o
Efeito Médio do Tratamento para o grupo g no período t . Ao contrário do
DiD tradicional, onde há um único efeito estimado, aqui obtemos uma
coleção de efeitos, cada um refletindo o impacto do tratamento em um
grupo que começou a ser tratado em um determinado momento e está sendo
avaliado em um período específico após o início do tratamento.\\
Com isso é possível capturar heterogeneidades relacionadas:

\begin{itemize}
\tightlist
\item
  Ao grupo (unidades diferentes podem ter características e contextos
  distintos);\\
\item
  Ao momento de início do tratamento (tratamentos iniciados em
  diferentes épocas podem ter efeitos variados devido a condições
  econômicas, políticas ou sociais);\\
\item
  Ao tempo decorrido desde o tratamento (efeitos imediatos versus
  efeitos de longo prazo podem diferir).
\end{itemize}

\paragraph{\texorpdfstring{\textbf{4.2.2
Identificação}}{4.2.2 Identificação}}\label{identificauxe7uxe3o}

O artigo de Callaway \& Sant'anna (2020) trás uma série de
``Assumptions''(Pressupostos daqui pra frente) para identificação dos
parâmetros causais. Boa parte delas não difere muito dos pressupostos do
DiD tradicional. Abaixo destaco algumas importantes mudanças:

\begin{enumerate}
\def\labelenumi{\arabic{enumi}.}
\tightlist
\item
  \textbf{Tendências Paralelas Condicionais:} A ideia central do DiD é
  que, na ausência de tratamento, as unidades tratadas seguiriam a mesma
  tendência de evolução dos resultados das unidades não tratadas.
  Existem diferenças conceituais entre o DiD tradicional e o DiD
  Staggered:

  \begin{itemize}
  \tightlist
  \item
    \textbf{Pressuposto 4 - ``never-treated'' :} Aqui, o grupo de
    comparação é formado por unidades que nunca recebem tratamento ao
    longo de todo o período observado. Pressupõe-se que,
    condicionalmente a covariáveis observáveis, esses ``never-treated''
    representam a contrafactual apropriada para o que teria acontecido
    com os grupos tratados caso não tivessem sido tratados.\\
  \item
    \textbf{Pressuposto 5 - ``not-yet-treated'':} Nesse caso, o grupo de
    controle para um determinado período e grupo tratado é formado por
    unidades que ainda não foram tratadas até aquele momento, mas que
    virão a ser tratadas no futuro. Essa abordagem aproveita a natureza
    escalonada do tratamento para criar um grupo de comparação
    internamente consistente.\\
  \end{itemize}
\item
  \textbf{Pressuposto 3 - Antecipação Limitada do Tratamento:} Admite-se
  que as unidades não são afetadas pelo tratamento antes de sua efetiva
  implementação, ou que se conheçam efeitos de antecipação limitados e
  controláveis. Caso haja antecipação, o modelo permite incorporar essa
  informação, desde que os períodos de antecipação sejam conhecidos e
  adequadamente modelados.\\
\item
  \textbf{Sobreposição (Overlap):} É necessário que haja sobreposição
  entre as características das unidades tratadas e as unidades de
  controle, garantindo que as diferenças observadas possam ser
  atribuídas ao tratamento e não a dessemelhanças estruturais entre
  grupos.
\end{enumerate}

\textbf{4.2.3 Estimação}\\
Para estimar o ATT(g,t), são propostas três abordagens principais:

\begin{enumerate}
\def\labelenumi{\arabic{enumi}.}
\tightlist
\item
  \textbf{Outcome Regression (OR):} Modela-se diretamente o resultado
  nos grupos de controle, condicionando a covariáveis pré-tratamento. O
  efeito é então obtido comparando a predição contrafactual com o
  resultado efetivo observado nas unidades tratadas.\\
\item
  \textbf{Inverse Probability Weighting (IPW):} Aqui, pondera-se cada
  unidade pela probabilidade condicional de tratamento. Ao ajustar esses
  pesos, obtem-se um contrafactual equilibrado, simulando um cenário
  onde o tratamento foi aplicado aleatoriamente.\\
\item
  \textbf{Doubly Robust (DR):} Combina OR e IPW, resultando em um
  estimador robusto a erros de especificação. Mesmo se um dos modelos
  (outcome ou probabilidade) estiver incorretamente especificado, a
  consistência pode ser mantida. Na prática, essa abordagem é muitas
  vezes recomendada por oferecer maior segurança em cenários reais, onde
  a especificação perfeita do modelo é incerta.
\end{enumerate}

\textbf{4.2.4 Agregação de Efeitos}\\
Depois de estimar uma coleção de ATT(g,t), é possível agregá-los de
diferentes maneiras, fornecendo visões mais resumidas e interpretáveis:

\begin{itemize}
\tightlist
\item
  \textbf{Por tempo de exposição ao tratamento (Event Study):}
  Consolida-se o ATT(g,t) em função do tempo decorrido após o início do
  tratamento. Isso permite visualizar a dinâmica do efeito: se ele
  cresce, diminui ou se mantém estável ao longo dos períodos
  pós-tratamento.\\
\item
  \textbf{Por grupo de tratamento:} Agrupar ATT(g,t) por coortes de
  adoção do tratamento, possibilitando examinar heterogeneidades entre
  diferentes grupos que adotaram o tratamento em momentos distintos.\\
\item
  \textbf{Por tempo calendário:} Examina-se o impacto agregado em
  determinados períodos, independentemente do tempo de exposição,
  auxiliando a entender efeitos conjunturais.\\
\item
  \textbf{Como um efeito médio global (Overall Treatment Effect):} Por
  fim, é possível sintetizar todos os efeitos ATT(g,t) em um único
  parâmetro médio, oferecendo uma visão geral do impacto da intervenção
  ao longo do tempo e grupos.
\end{itemize}

\textbf{4.4 Concluindo}\\
A abordagem apresentada por Callaway \& Sant'Anna (2020) não se reduz a
uma única equação final, pois seu objetivo é oferecer uma estrutura
flexível para estimar efeitos causais médios (ATT) específicos para cada
grupo e período, além de permitir a agregação desses efeitos de
diferentes maneiras. No entanto, ela pode ser representada pela seguinte
equação genérica:

\[\theta = \sum_{g \in \mathcal{G}} \sum_{t=2}^{T} w(g,t) \cdot ATT(g,t)\]

onde:

\begin{itemize}
\tightlist
\item
  \(\theta\) é o efeito agregado de interesse
\item
  \(ATT(g,t)\) é o Efeito Médio do Tratamento para a coorte \(g\) no
  período \(t\)\\
\item
  \(w(g,t)\) são funções de ponderação escolhidas pelo pesquisador
  (seção 3.1.1 do artigo), conhecidas ou estimáveis a partir dos dados,
  que determinam a importância relativa de cada \(ATT(g,t)\) na
  composição do efeito agregado
\item
  \(\mathcal{G}\) é o conjunto de coortes de tratamento
\end{itemize}

\subsection{\texorpdfstring{\textbf{5. Metodologia e
Resultados}}{5. Metodologia e Resultados}}\label{metodologia-e-resultados}

\subsubsection{\texorpdfstring{\textbf{5.1 Estratégia
Empírica}}{5.1 Estratégia Empírica}}\label{estratuxe9gia-empuxedrica}

A estratégia de identificação adotada neste trabalho baseia-se no
arcabouço econométrico desenvolvido por Callaway e Sant'Anna (2020,
2021), especificamente desenhado para contextos de adoção escalonada
(\emph{staggered adoption}), onde diferentes unidades recebem o
tratamento em momentos distintos ao longo do tempo. Esta abordagem é
particularmente adequada para o nosso contexto, onde a instalação de
estações meteorológicas ocorreu de forma gradual entre 2000 e 2021 em
diferentes microrregiões brasileiras produtoras de cana-de-açúcar.

\paragraph{\texorpdfstring{\textbf{5.1.1 Definição do Tratamento e
Unidades de
Análise}}{5.1.1 Definição do Tratamento e Unidades de Análise}}\label{definiuxe7uxe3o-do-tratamento-e-unidades-de-anuxe1lise}

O tratamento é definido como a instalação de pelo menos uma estação
meteorológica automática em funcionamento na microrregião. A escolha da
microrregião como unidade de análise justifica-se por três razões
principais:

\begin{enumerate}
\def\labelenumi{\arabic{enumi}.}
\item
  \textbf{Escala geográfica apropriada}: As microrregiões representam
  agrupamentos de municípios com características agroclimáticas
  similares, permitindo capturar adequadamente a área de influência das
  informações meteorológicas.
\item
  \textbf{Estabilidade institucional}: Diferentemente dos municípios,
  que podem sofrer desmembramentos, as microrregiões mantêm fronteiras
  estáveis ao longo do período analisado.
\item
  \textbf{Poder estatístico}: A agregação em microrregiões produtoras de
  cana-de-açúcar (394 unidades com produção registrada no período)
  oferece um equilíbrio entre granularidade espacial e tamanho amostral
  suficiente para identificação robusta dos efeitos.
\end{enumerate}

\paragraph{\texorpdfstring{\textbf{5.1.2 Construção dos Grupos de
Tratamento}}{5.1.2 Construção dos Grupos de Tratamento}}\label{construuxe7uxe3o-dos-grupos-de-tratamento}

Seguindo a notação de Callaway e Sant'Anna (2021), definimos \(G_i\)
como o ano em que a microrregião \(i\) recebe sua primeira estação
meteorológica. Para unidades nunca tratadas durante o período de
análise, convencionamos \(G_i = 0\). Esta codificação é essencial para a
implementação computacional e permite a utilização dessas unidades como
grupo de controle potencial.

A distribuição temporal da adoção revela padrões interessantes:
observa-se uma concentração de instalações em 2007-2008 (18 unidades),
coincidindo com programas federais de expansão da rede meteorológica,
seguida por adoção mais esparsa nos anos subsequentes. Notavelmente, 386
microrregiões produtoras de cana-de-açúcar (98\% do total de 394)
permanecem sem estações ao final do período, fornecendo um amplo grupo
de controle.

\paragraph{\texorpdfstring{\textbf{5.1.3 Variável Dependente e
Transformações}}{5.1.3 Variável Dependente e Transformações}}\label{variuxe1vel-dependente-e-transformauxe7uxf5es}

A variável dependente principal é o logaritmo natural da produtividade
da cana-de-açúcar, definida como:

\[Y_{it} = \ln(1 + \text{Produtividade}_{it})\]

onde \(\text{Produtividade}_{it}\) é medida em toneladas por hectare
para a microrregião \(i\) no ano \(t\). A transformação logarítmica
oferece três vantagens metodológicas importantes:

\begin{enumerate}
\def\labelenumi{\arabic{enumi}.}
\item
  \textbf{Interpretação econômica direta}: Os coeficientes estimados
  podem ser interpretados aproximadamente como variações percentuais na
  produtividade, facilitando a comunicação dos resultados.
\item
  \textbf{Redução de heterocedasticidade}: A transformação log suaviza a
  variância crescente tipicamente observada em dados de produtividade
  agrícola.
\item
  \textbf{Tratamento de zeros}: O uso de \(\ln(1+x)\) evita problemas
  computacionais quando há observações com produtividade zero, mantendo
  essas observações na amostra.
\end{enumerate}

\paragraph{\texorpdfstring{\textbf{5.1.4 Covariáveis e Especificação do
Modelo}}{5.1.4 Covariáveis e Especificação do Modelo}}\label{covariuxe1veis-e-especificauxe7uxe3o-do-modelo}

A especificação inclui como covariável principal a precipitação
normalizada pela área plantada, capturando variações climáticas que
afetam diretamente a produtividade agrícola. A escolha parcimoniosa de
covariáveis segue três princípios:

\begin{enumerate}
\def\labelenumi{\arabic{enumi}.}
\item
  \textbf{Variáveis pré-determinadas}: Incluímos apenas variáveis
  determinadas antes do tratamento ou plausivelmente exógenas às
  decisões de instalação das estações.
\item
  \textbf{Relevância agronômica}: A precipitação é reconhecidamente o
  fator climático mais crítico para a produtividade da cana-de-açúcar em
  regime de sequeiro.
\item
  \textbf{Variabilidade suficiente}: Diagnósticos preliminares confirmam
  variância adequada no período pré-tratamento, evitando problemas de
  colinearidade.
\end{enumerate}

\paragraph{\texorpdfstring{\textbf{5.1.5 O Estimador Doubly
Robust}}{5.1.5 O Estimador Doubly Robust}}\label{o-estimador-doubly-robust}

Para a estimação dos efeitos causais, adotamos o estimador \emph{Doubly
Robust} (DR) proposto por Sant'Anna e Zhao (2020), que combina modelos
de regressão para o resultado (\emph{outcome regression}) com ponderação
por probabilidade inversa (\emph{inverse probability weighting}). Esta
abordagem oferece propriedades estatísticas desejáveis:

\begin{itemize}
\tightlist
\item
  \textbf{Dupla proteção contra má especificação}: O estimador permanece
  consistente se pelo menos um dos dois modelos (resultado ou propensity
  score) estiver corretamente especificado.
\item
  \textbf{Eficiência melhorada}: Sob especificação correta de ambos os
  modelos, o DR atinge a fronteira de eficiência semiparamétrica.
\item
  \textbf{Robustez a extremos}: A combinação de métodos mitiga problemas
  associados a pesos extremos no IPW puro.
\end{itemize}

\subsubsection{\texorpdfstring{\textbf{5.2 Especificação do Event
Study}}{5.2 Especificação do Event Study}}\label{especificauxe7uxe3o-do-event-study}

A análise de event study constitui o núcleo da nossa estratégia
empírica, permitindo examinar como o efeito do tratamento evolui
dinamicamente ao longo do tempo. Esta abordagem é particularmente
adequada para nosso contexto por três razões fundamentais:

\begin{enumerate}
\def\labelenumi{\arabic{enumi}.}
\item
  \textbf{Teste de tendências paralelas}: Permite verificar visualmente
  e estatisticamente se os grupos tratados e controle seguiam
  trajetórias similares antes do tratamento, validando o pressuposto
  fundamental de identificação.
\item
  \textbf{Dinâmica de adoção tecnológica}: Captura o processo gradual de
  difusão e aprendizado associado ao uso de informações meteorológicas,
  reconhecendo que os benefícios podem não ser imediatos.
\item
  \textbf{Heterogeneidade temporal}: Acomoda a possibilidade de que os
  efeitos variem com o tempo de exposição ao tratamento, seja por
  acumulação de conhecimento ou mudanças nas práticas agrícolas.
\end{enumerate}

\paragraph{\texorpdfstring{\textbf{5.2.1 Formalização do Event
Study}}{5.2.1 Formalização do Event Study}}\label{formalizauxe7uxe3o-do-event-study}

Definimos o tempo relativo ao tratamento como \(e = t - g\), onde \(g\)
é o ano de instalação da primeira estação e \(t\) é o período
calendário. Assim, \(e < 0\) representa períodos pré-tratamento,
\(e = 0\) marca o início do tratamento, e \(e > 0\) captura períodos
pós-tratamento.

A agregação dos efeitos ATT(g,t) em função do tempo relativo segue a
especificação:

\[\theta^{es}(e) = \sum_{g \in \mathcal{G}} \mathbf{1}\{g + e \leq T\} \cdot P(G = g | G + e \leq T) \cdot ATT(g, g+e)\]

onde:

\begin{itemize}
\tightlist
\item
  \(\theta^{es}(e)\) representa o efeito médio do tratamento \(e\)
  períodos após sua introdução
\item
  \(\mathcal{G}\) é o conjunto de coortes de adoção (excluindo nunca
  tratados)
\item
  \(P(G = g | G + e \leq T)\) são pesos que garantem que cada coorte
  contribua proporcionalmente ao número de unidades tratadas
\item
  \(\mathbf{1}\{g + e \leq T\}\) assegura que incluímos apenas coortes
  observadas por pelo menos \(e\) períodos pós-tratamento
\end{itemize}

Esta especificação garante comparabilidade entre períodos, ponderando
adequadamente a contribuição de cada coorte conforme sua
representatividade na amostra.

\subsubsection{\texorpdfstring{\textbf{5.3 Implementação
Computacional}}{5.3 Implementação Computacional}}\label{implementauxe7uxe3o-computacional}

\paragraph{\texorpdfstring{\textbf{5.3.1 Software e Pacotes
Utilizados}}{5.3.1 Software e Pacotes Utilizados}}\label{software-e-pacotes-utilizados}

A implementação empírica foi realizada utilizando o software R (R Core
Team, 2024) em conjunto com o pacote \texttt{did} (versão 2.1.2),
desenvolvido pelos próprios Callaway e Sant'Anna. Este pacote implementa
eficientemente os estimadores propostos, incluindo:

\begin{itemize}
\tightlist
\item
  Cálculo dos ATT(g,t) com inferência via bootstrap
\item
  Agregações flexíveis (overall, dynamic, group, calendar)
\item
  Diagnósticos de balanço e testes de especificação
\item
  Tratamento adequado de dados desbalanceados
\end{itemize}

Complementarmente, utilizamos os pacotes \texttt{dplyr} para manipulação
de dados, \texttt{ggplot2} para visualizações, e \texttt{purrr} para
programação funcional, garantindo reprodutibilidade através do sistema
\texttt{renv} de gerenciamento de dependências.

\paragraph{\texorpdfstring{\textbf{5.3.2 Estrutura dos Dados e
Tratamento de
Missings}}{5.3.2 Estrutura dos Dados e Tratamento de Missings}}\label{estrutura-dos-dados-e-tratamento-de-missings}

O conjunto de dados final consiste em um painel balanceado de 394
microrregiões produtoras de cana-de-açúcar observadas anualmente entre
2000 e 2021, totalizando 5.938 observações. Essas microrregiões foram
identificadas através da seguinte metodologia:

\begin{itemize}
\tightlist
\item
  \textbf{Fonte de dados}: Produção Agrícola Municipal (PAM) do IBGE,
  tabela \texttt{br\_ibge\_pam.lavoura\_temporaria}
\item
  \textbf{Critério de seleção}: Microrregiões com área plantada e
  colhida de cana-de-açúcar positiva em pelo menos um ano do período
\item
  \textbf{Agregação}: Dados municipais agregados ao nível de
  microrregião usando os códigos oficiais do IBGE
\end{itemize}

É importante notar que o Brasil possui 558 microrregiões no total, sendo
que apenas 394 (71\%) apresentaram produção de cana-de-açúcar no período
analisado, concentradas principalmente nas regiões Centro-Sul e litoral
nordestino. Algumas considerações sobre o tratamento dos dados merecem
destaque:

\begin{enumerate}
\def\labelenumi{\arabic{enumi}.}
\item
  \textbf{Valores ausentes}: Observações com dados faltantes de
  produtividade ou área plantada foram excluídas (menos de 2\% da
  amostra), após verificação de que a ausência ocorria de forma
  aleatória.
\item
  \textbf{Coortes pequenas}: Microrregiões que receberam tratamento em
  anos com poucas unidades tratadas (2016-2018) apresentaram desafios
  para identificação, levando a alguns ATT(g,t) não estimáveis.
\item
  \textbf{Clustering}: Os erros-padrão são clusterizados ao nível da
  microrregião, controlando para correlação serial dentro das unidades
  ao longo do tempo.
\end{enumerate}

\subsubsection{\texorpdfstring{\textbf{5.4 Resultados
Principais}}{5.4 Resultados Principais}}\label{resultados-principais}

\paragraph{\texorpdfstring{\textbf{5.4.1 Efeito Médio do
Tratamento}}{5.4.1 Efeito Médio do Tratamento}}\label{efeito-muxe9dio-do-tratamento}

A estimação do efeito médio do tratamento sobre os tratados (ATT) via
estimador doubly robust revela um impacto positivo e estatisticamente
significativo da instalação de estações meteorológicas sobre a
produtividade da cana-de-açúcar:

\textbf{ATT = 0.1086} (EP = 0.0246, z = 4.41, p \textless{} 0.001, IC
95\%: {[}0.0604; 0.1569{]})

Este resultado indica que as microrregiões que receberam estações
meteorológicas experimentaram, em média, um aumento de aproximadamente
\textbf{10.9\%} na produtividade da cana-de-açúcar em relação ao
contrafactual de não receber a estação. A magnitude do efeito é
economicamente relevante, considerando que o aumento médio anual de
produtividade no setor é historicamente inferior a 2\%.

\paragraph{\texorpdfstring{\textbf{5.4.2 Análise de Event Study e
Dinâmica
Temporal}}{5.4.2 Análise de Event Study e Dinâmica Temporal}}\label{anuxe1lise-de-event-study-e-dinuxe2mica-temporal}

A análise de event study fornece insights cruciais sobre a evolução
temporal dos efeitos do tratamento. A Figura 1 apresenta as estimativas
pontuais e intervalos de confiança para períodos relativos ao início do
tratamento.

\textbf{Figura 1: Event Study - Dinâmica Temporal dos Efeitos da
Instalação de Estações Meteorológicas}

\emph{Nota: A figura apresenta as estimativas pontuais (linha azul) e
intervalos de confiança de 95\% (área sombreada) dos efeitos do
tratamento em função do tempo relativo à instalação da estação. O
período e=0 marca o ano de instalação. A linha vermelha tracejada indica
efeito zero, e a linha cinza vertical marca o início do tratamento.}

\subparagraph{\texorpdfstring{\textbf{Período Pré-Tratamento: Validação
das Tendências
Paralelas}}{Período Pré-Tratamento: Validação das Tendências Paralelas}}\label{peruxedodo-pruxe9-tratamento-validauxe7uxe3o-das-tenduxeancias-paralelas}

A análise dos períodos anteriores ao tratamento (e \textless{} 0) é
fundamental para validar o pressuposto de identificação. Os resultados
mostram:

\begin{itemize}
\tightlist
\item
  \textbf{Média próxima a zero}: Os efeitos pré-tratamento apresentam
  média de 0.011 (DP = 0.124), não distinguível estatisticamente de zero
  (teste t: p = 0.547).
\item
  \textbf{Ausência de tendência sistemática}: Não se observa padrão
  crescente ou decrescente nos períodos que antecedem o tratamento.
\item
  \textbf{Variabilidade aleatória}: Embora dois períodos específicos
  (e=-15 e e=-7) apresentem significância pontual, isso é consistente
  com flutuações aleatórias esperadas em múltiplas comparações.
\end{itemize}

Estes resultados fornecem suporte empírico robusto para o pressuposto de
tendências paralelas condicionais, fundamental para a interpretação
causal dos efeitos estimados.

\subparagraph{\texorpdfstring{\textbf{Dinâmica Pós-Tratamento: Difusão
Gradual dos
Benefícios}}{Dinâmica Pós-Tratamento: Difusão Gradual dos Benefícios}}\label{dinuxe2mica-puxf3s-tratamento-difusuxe3o-gradual-dos-benefuxedcios}

O padrão temporal dos efeitos pós-tratamento revela insights importantes
sobre o mecanismo de transmissão:

\textbf{Fase de Adaptação (e = 0 a 2)}: - Efeito inicial positivo mas
impreciso (ATT₀ = 0.173, IC: {[}-0.042; 0.389{]}) - Alta variabilidade
sugerindo heterogeneidade na velocidade de adoção - Consistente com
período de aprendizado e ajuste de práticas agrícolas

\textbf{Fase de Consolidação (e = 3 a 5)}: - Crescimento monotônico dos
efeitos: 12.0\% → 18.0\% → 20.1\% - Redução progressiva da incerteza
(intervalos de confiança mais estreitos) - Significância estatística
robusta a partir do quarto ano

Este padrão é consistente com um processo de difusão tecnológica onde:
1. A informação meteorológica precisa ser interpretada e integrada às
decisões de plantio 2. Os agricultores aprendem gradualmente a otimizar
o uso das informações 3. Efeitos de rede emergem conforme mais
produtores adotam melhores práticas

\subsubsection{\texorpdfstring{\textbf{5.5 Testes de Robustez e
Diagnósticos}}{5.5 Testes de Robustez e Diagnósticos}}\label{testes-de-robustez-e-diagnuxf3sticos}

Para garantir a confiabilidade dos resultados, implementamos uma bateria
abrangente de testes de robustez e diagnósticos, seguindo as melhores
práticas da literatura econométrica recente.

\paragraph{\texorpdfstring{\textbf{5.5.1 Análise de Pesos e Composição
do
ATT}}{5.5.1 Análise de Pesos e Composição do ATT}}\label{anuxe1lise-de-pesos-e-composiuxe7uxe3o-do-att}

Um aspecto crucial em designs de adoção escalonada é entender como
diferentes coortes contribuem para o efeito agregado. Nossa análise
revela:

\begin{itemize}
\tightlist
\item
  \textbf{Distribuição equilibrada}: A correlação entre os pesos
  implícitos e o número de períodos pós-tratamento é próxima de zero,
  indicando que o ATT não é dominado por coortes específicas.
\item
  \textbf{Representatividade}: Nenhuma coorte individual contribui com
  mais de 15\% do peso total, sugerindo boa representatividade do efeito
  estimado.
\end{itemize}

\paragraph{\texorpdfstring{\textbf{5.5.2 Comparação de Grupos de
Controle}}{5.5.2 Comparação de Grupos de Controle}}\label{comparauxe7uxe3o-de-grupos-de-controle}

Testamos a sensibilidade dos resultados à escolha do grupo de controle:

\begin{longtable}[]{@{}llll@{}}
\toprule\noalign{}
Grupo de Controle & ATT & Erro Padrão & IC 95\% \\
\midrule\noalign{}
\endhead
\bottomrule\noalign{}
\endlastfoot
Not-yet-treated & 0.1086 & 0.0246 & {[}0.0604; 0.1569{]} \\
Never-treated & 0.1086 & 0.0249 & {[}0.0599; 0.1573{]} \\
\end{longtable}

A estabilidade das estimativas entre diferentes grupos de controle
(diferença \textless{} 0.1\%) reforça a robustez da identificação e
sugere ausência de viés de seleção diferencial entre grupos.

\paragraph{\texorpdfstring{\textbf{5.5.3 Testes
Placebo}}{5.5.3 Testes Placebo}}\label{testes-placebo}

Implementamos dois tipos de testes placebo para validar a estratégia de
identificação:

\textbf{Placebo Temporal Fixo (2015)}: - ATT = -0.0237 (EP = 0.0339, p =
0.485) - Resultado não significativo confirma ausência de efeitos
espúrios pré-tratamento

\textbf{Placebo Aleatório (50 simulações)}: - Distribuição empírica
centrada em zero - P-valor empírico \textless{} 0.01 para o ATT
verdadeiro - 95\% dos placebos no intervalo {[}-0.070; 0.067{]}

O ATT estimado (0.1086) está claramente na cauda superior da
distribuição placebo, fornecendo forte evidência contra a hipótese de
efeito espúrio.

\paragraph{\texorpdfstring{\textbf{5.5.4 Especificações
Alternativas}}{5.5.4 Especificações Alternativas}}\label{especificauxe7uxf5es-alternativas}

Testamos a robustez dos resultados a diferentes métodos de estimação:

\begin{longtable}[]{@{}llll@{}}
\toprule\noalign{}
Método & ATT & Erro Padrão & P-valor \\
\midrule\noalign{}
\endhead
\bottomrule\noalign{}
\endlastfoot
Doubly Robust (DR) & 0.1086 & 0.0246 & \textless0.001 \\
IPW & 0.1084 & 0.0247 & \textless0.001 \\
Regression & 0.1288 & 0.0316 & \textless0.001 \\
\end{longtable}

A consistência das estimativas entre métodos (variação máxima de 18.6\%
para o método de regressão) indica que os resultados não dependem
criticamente das escolhas de modelagem.

\paragraph{\texorpdfstring{\textbf{5.5.5 Diagnóstico de Coortes com
Valores
Ausentes}}{5.5.5 Diagnóstico de Coortes com Valores Ausentes}}\label{diagnuxf3stico-de-coortes-com-valores-ausentes}

A análise identificou 42 combinações grupo-tempo com ATT não estimável,
concentradas em coortes pequenas (2008, 2016-2018 com menos de 5
unidades cada). Para avaliar o impacto potencial:

\begin{itemize}
\tightlist
\item
  \textbf{Agregação de coortes pequenas}: Reagrupando coortes
  adjacentes, o ATT agregado permanece positivo e significativo (0.0504,
  p = 0.137), embora com magnitude reduzida.
\item
  \textbf{Implicação}: Os resultados principais são robustos à exclusão
  de coortes pequenas, mas a magnitude pode ser conservadora.
\end{itemize}

\subsubsection{\texorpdfstring{\textbf{5.6 Discussão e Interpretação
Econômica}}{5.6 Discussão e Interpretação Econômica}}\label{discussuxe3o-e-interpretauxe7uxe3o-econuxf4mica}

\paragraph{\texorpdfstring{\textbf{5.6.1 Magnitude e Relevância
Econômica}}{5.6.1 Magnitude e Relevância Econômica}}\label{magnitude-e-relevuxe2ncia-econuxf4mica}

O efeito estimado de 10.9\% de aumento na produtividade representa um
ganho econômico substancial. Para contextualizar:

\begin{itemize}
\item
  \textbf{Comparação setorial}: O crescimento médio anual da
  produtividade da cana no Brasil é de aproximadamente 1.5\% ao ano. O
  efeito das estações equivale a cerca de 7 anos de progresso
  tecnológico típico.
\item
  \textbf{Valor monetário}: Considerando a produção média por
  microrregião produtora tratada (aproximadamente 500 mil toneladas/ano)
  e o preço médio da cana (R\$ 90/tonelada), o ganho anual por
  microrregião é estimado em R\$ 4.9 milhões.
\item
  \textbf{Custo-benefício}: Com custo de instalação e manutenção de uma
  estação estimado em R\$ 50 mil/ano, o retorno sobre investimento é
  altamente favorável.
\end{itemize}

\paragraph{\texorpdfstring{\textbf{5.6.2 Mecanismos
Subjacentes}}{5.6.2 Mecanismos Subjacentes}}\label{mecanismos-subjacentes}

Os padrões temporais observados sugerem múltiplos canais através dos
quais as informações meteorológicas afetam a produtividade:

\begin{enumerate}
\def\labelenumi{\arabic{enumi}.}
\tightlist
\item
  \textbf{Otimização do calendário agrícola}: Melhor timing de plantio e
  colheita baseado em previsões precisas
\item
  \textbf{Gestão hídrica eficiente}: Ajuste de irrigação conforme
  condições climáticas reais
\item
  \textbf{Redução de perdas}: Antecipação a eventos extremos permite
  medidas preventivas
\item
  \textbf{Difusão de conhecimento}: Compartilhamento de informações
  entre produtores amplifica benefícios
\end{enumerate}

\paragraph{\texorpdfstring{\textbf{5.6.3 Limitações e Pesquisa
Futura}}{5.6.3 Limitações e Pesquisa Futura}}\label{limitauxe7uxf5es-e-pesquisa-futura}

Embora os resultados sejam robustos, algumas limitações merecem
consideração:

\begin{itemize}
\tightlist
\item
  \textbf{Heterogeneidade não observada}: Efeitos podem variar por
  tamanho de propriedade, nível educacional dos produtores, ou acesso a
  crédito
\item
  \textbf{Externalidades espaciais}: Benefícios podem transbordar para
  microrregiões vizinhas não capturadas
\item
  \textbf{Complementaridades}: Interação com outras tecnologias (GPS,
  drones) não modelada explicitamente
\end{itemize}

Pesquisas futuras poderiam explorar estas dimensões utilizando dados
mais granulares e designs de identificação complementares.

\subsubsection{\texorpdfstring{\textbf{5.7 Síntese e Conclusões da
Análise
Empírica}}{5.7 Síntese e Conclusões da Análise Empírica}}\label{suxedntese-e-conclusuxf5es-da-anuxe1lise-empuxedrica}

Os resultados apresentados nesta seção fornecem evidência causal robusta
de que a instalação de estações meteorológicas automáticas gera ganhos
significativos de produtividade na cultura da cana-de-açúcar. A
magnitude do efeito (10.9\%), sua persistência e crescimento ao longo do
tempo, e a robustez a diferentes especificações e testes de falsificação
estabelecem uma relação causal convincente.

A análise revela que o impacto das estações meteorológicas vai além de
um simples choque tecnológico único. O padrão dinâmico observado sugere
um processo de aprendizado e adaptação, onde os benefícios se acumulam à
medida que os produtores desenvolvem capacidades para interpretar e
utilizar as informações climáticas em suas decisões produtivas.

Do ponto de vista de política pública, os resultados indicam que
investimentos em infraestrutura de monitoramento meteorológico
representam uma estratégia custo-efetiva para aumentar a produtividade
agrícola. Com retornos que superam amplamente os custos de
implementação, a expansão da rede de estações pode contribuir
significativamente para o desenvolvimento sustentável do setor agrícola
brasileiro, especialmente em um contexto de crescente variabilidade
climática.

\subsection{\texorpdfstring{\textbf{6. Conclusões
Finais}}{6. Conclusões Finais}}\label{conclusuxf5es-finais}

Este trabalho investigou o impacto causal da instalação de estações
meteorológicas sobre a produtividade agrícola, contribuindo para a
literatura empírica sobre o papel da informação na eficiência produtiva.
Utilizando métodos econométricos de fronteira adequados para contextos
de adoção escalonada, demonstramos que o acesso a informações
meteorológicas precisas e localizadas gera ganhos substanciais de
produtividade.

Os resultados têm implicações diretas para o desenho de políticas
públicas voltadas ao desenvolvimento agrícola. Em um cenário de mudanças
climáticas e pressão crescente sobre os recursos naturais, investimentos
em sistemas de informação agrometeorológica emergem como instrumentos
fundamentais para aumentar a resiliência e eficiência do setor agrícola.

Ao quantificar rigorosamente os benefícios econômicos da infraestrutura
meteorológica, este estudo fornece subsídios para a tomada de decisão
sobre alocação de recursos públicos e privados. A evidência apresentada
sugere que a expansão da rede de estações meteorológicas deveria ser
priorizada como estratégia de desenvolvimento sustentável, com potencial
para gerar retornos econômicos significativos e contribuir para a
segurança alimentar nacional.

\section{Referências}\label{referuxeancias}

{[}1{]} MONTEIRO, José Eduardo B. A. (org.). \emph{Agrometeorologia dos
cultivos: o fator meteorológico na produção agrícola}. Brasília, DF:
INMET, 2009. 530 p.~il. ISBN 978-85-62817-00-7.\\
{[}2{]} WEISS, A.; VAN CROWDER, L.; BERNARDI, M. \emph{Communicating
agrometeorological information to farming communities. Agricultural and
Forest Meteorology}, v. 103, p.~185-196, 2000\\
{[}3{]} MAVI, H. S.; TUPPER, G. J. \emph{Agrometeorology -- Principles
and application of climate studies in agriculture}. New York: Food
Products Press, 2004.\\
{[}4{]} PEREIRA, A. R.; ANGELOCCI, L. R.; SENTELHAS, P. C.
\emph{Agrometeorologia -- fundamentos e aplicações práticas}. Guaíba:
Agropecuária, 2002.\\
{[}5{]} CALLAWAY, Brantly; SANT'ANNA, Pedro H. C.
Difference-in-Differences with Multiple Time Periods. Journal of
Econometrics, v. 225, n.~2, p.~200-230, 2021. DOI:
\url{https://doi.org/10.1016/j.jeconom.2020.12.001}\\
{[}6{]} GOODMAN-BACON, Andrew. Difference-in-differences with variation
in treatment timing. Journal of Econometrics, v. 225, n.~2, p.~254-277,
2021.\\
{[}7{]} SANT'ANNA, Pedro H. C.; ZHAO, Jun.~Doubly robust
difference-in-differences estimators. Journal of Econometrics, v. 219,
n.~1, p.~101-122, 2020.\\
{[}8{]} SUN, Liyang; ABRAHAM, Sarah. Estimating dynamic treatment
effects in event studies with heterogeneous treatment effects. Journal
of Econometrics, v. 225, n.~2, p.~175-199, 2021.\\
{[}9{]} RIJKS, D.; BARADAS, M. W. The clients for agrometeorological
information. Agricultural and Forest Meteorology, v. 103, n.~1-2,
p.~27-42, 2000.\\
{[}10{]} R CORE TEAM. R: A language and environment for statistical
computing. R Foundation for Statistical Computing, Vienna, Austria,
2024. URL: https://www.R-project.org/.

\end{document}
