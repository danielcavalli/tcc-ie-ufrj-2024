% Apresentação de Defesa de TCC - Daniel Cavalli
% Instituto de Economia - UFRJ
% 2025

\documentclass[10pt,aspectratio=169]{beamer}

% Pacotes essenciais
\usepackage[brazilian]{babel}
\usepackage[utf8]{inputenc}
\usepackage[T1]{fontenc}
\usepackage{amsmath,amssymb}
\usepackage{graphicx}
\usepackage{booktabs}
\usepackage{multirow}
\usepackage{tikz}
\usepackage{pgfplots}
\pgfplotsset{compat=1.17}

% Configuração do tema
\usetheme{Boadilla}  % Tema mais limpo e minimalista
\usecolortheme{default}  % Remove cores extras do whale
\setbeamertemplate{navigation symbols}{}
\setbeamertemplate{footline}[frame number]

% Remove sombras e elementos decorativos
\setbeamertemplate{blocks}[rounded][shadow=false]
\setbeamertemplate{title page}[default][colsep=-4bp,rounded=false,shadow=false]
\setbeamertemplate{frametitle}[default][colsep=-2bp,rounded=false,shadow=false,center]

% Cores personalizadas UFRJ
\definecolor{ufrjblue}{RGB}{0,53,96}
\definecolor{ufrjgreen}{RGB}{0,128,0}
\definecolor{ufrjlightblue}{RGB}{51,102,153}
\definecolor{ufrjgray}{RGB}{100,100,100}

% Aplicação sutil das cores
\setbeamercolor{structure}{fg=ufrjblue}
\setbeamercolor{frametitle}{bg=white,fg=ufrjblue}
\setbeamercolor{title}{fg=ufrjblue}
\setbeamercolor{block title}{bg=ufrjlightblue,fg=white}
\setbeamercolor{block body}{bg=ufrjlightblue!10}
\setbeamercolor{block title alerted}{bg=red!90,fg=white}
\setbeamercolor{block body alerted}{bg=red!10}

% Configuração de itemize/enumerate
\setbeamercolor{itemize item}{fg=ufrjblue}
\setbeamercolor{itemize subitem}{fg=ufrjlightblue}
\setbeamercolor{enumerate item}{fg=ufrjblue}

% Configurações de fonte
\usefonttheme{professionalfonts}
\setbeamerfont{title}{size=\Large,series=\bfseries}
\setbeamerfont{frametitle}{size=\large,series=\bfseries}
\setbeamerfont{block title}{size=\normalsize,series=\bfseries}

% Configurações adicionais para design limpo
\setbeamertemplate{itemize items}[circle]
\setbeamertemplate{enumerate items}[default]
\setbeamertemplate{section in toc}[circle]

% Remove linhas de separação
\setbeamertemplate{headline}{}
\setbeamertemplate{footline}{
    \leavevmode%
    \hbox{%
    \begin{beamercolorbox}[wd=.333333\paperwidth,ht=2.25ex,dp=1ex,center]{author in head/foot}%
        \usebeamerfont{author in head/foot}\insertshortauthor
    \end{beamercolorbox}%
    \begin{beamercolorbox}[wd=.333333\paperwidth,ht=2.25ex,dp=1ex,center]{title in head/foot}%
        \usebeamerfont{title in head/foot}\insertshorttitle
    \end{beamercolorbox}%
    \begin{beamercolorbox}[wd=.333333\paperwidth,ht=2.25ex,dp=1ex,right]{date in head/foot}%
        \usebeamerfont{date in head/foot}\insertshortdate{}\hspace*{2em}
        \insertframenumber{} / \inserttotalframenumber\hspace*{2ex} 
    \end{beamercolorbox}}%
    \vskip0pt%
}

% Ajuste de margens para mais espaço
\setbeamersize{text margin left=1em,text margin right=1em}

% Informações do documento
\title[Estações Meteorológicas e Produtividade Agrícola]{Impacto de Estações Meteorológicas na Produtividade Agrícola: \\ Uma Aplicação de Diferenças em Diferenças com Tratamento Escalonado}

\author[Daniel Cavalli]{Daniel Cavalli \\ \small Orientador: Prof. Romero Rocha}

\institute[IE-UFRJ]{
  Instituto de Economia\\
  Universidade Federal do Rio de Janeiro
}

\date{2025}

% Importa valores automaticamente gerados
% Arquivo gerado automaticamente por generate_latex_values.r
% Última atualização: 2025-11-16

% Valores do teste placebo aleatório
\newcommand{\placebotruatt}{0.118}
\newcommand{\placebopvalue}{0.020}
\newcommand{\placebolower}{-0.041}
\newcommand{\placeboupper}{0.043}
\newcommand{\placebonsims}{50}
\newcommand{\placebomean}{-0.002}
\newcommand{\placebosd}{0.023}

% Valores de precisão do p-valor (Monte Carlo)
\newcommand{\placebopvaluese}{0.020}
\newcommand{\placebopvaluelower}{0.000}
\newcommand{\placebopvalueupper}{0.058}
\newcommand{\placebonextremes}{0}

% Valores formatados para texto
\newcommand{\placebotruattpct}{11.8\%}
\newcommand{\placebopvaluepct}{2.0\%}

% Valores do modelo principal (resultado detalhado)
\newcommand{\mainatt}{0.118}
\newcommand{\mainse}{0.042}
\newcommand{\mainz}{2.789}
\newcommand{\mainp}{0.005}
\newcommand{\maincilower}{0.035}
\newcommand{\mainciupper}{0.201}
\newcommand{\mainattpct}{11.8\%}

% Grupo de Controle: Never-treated
\newcommand{\nevertreatedatt}{0.136}
\newcommand{\nevertreatedse}{0.053}
\newcommand{\nevertreatedlower}{0.032}
\newcommand{\nevertreatedupper}{0.240}

% Especificação: Sem Covariáveis
\newcommand{\nocovatt}{0.124}
\newcommand{\nocovse}{0.042}
\newcommand{\nocovlower}{0.042}
\newcommand{\nocover}{0.207}

% Especificação: IPW
\newcommand{\ipwatt}{0.122}
\newcommand{\ipwse}{0.042}
\newcommand{\ipwlower}{0.041}
\newcommand{\ipwupper}{0.204}

% Especificação: Regressão de Resultado
\newcommand{\regatt}{0.114}
\newcommand{\regse}{0.043}
\newcommand{\reglower}{0.031}
\newcommand{\regupper}{0.198}

% Valores da análise de sensibilidade temporal
% Completo (2003-2023)
\newcommand{\sensfullatt}{0.115}
\newcommand{\sensfullse}{0.027}
\newcommand{\sensfulllower}{0.062}
\newcommand{\sensfullupper}{0.169}
\newcommand{\sensfulln}{7486}
% Excluindo Início (2006-2023)
\newcommand{\sensnostartatt}{0.115}
\newcommand{\sensnostartse}{0.031}
\newcommand{\sensnostartlower}{0.054}
\newcommand{\sensnostartupper}{0.177}
\newcommand{\sensnostartn}{6304}
% Excluindo Final (2003-2019)
\newcommand{\senssensaltcatt}{0.109}
\newcommand{\senssensaltcse}{0.025}
\newcommand{\senssensaltclower}{0.060}
\newcommand{\senssensaltcupper}{0.158}
\newcommand{\senssensaltcn}{6698}
% Excluindo COVID (2003-2019)
\newcommand{\sensnocovidatt}{0.109}
\newcommand{\sensnocovidse}{0.025}
\newcommand{\sensnocovidlower}{0.059}
\newcommand{\sensnocovidupper}{0.159}
\newcommand{\sensnocovidn}{6698}
% Pré-COVID (2003-2019)
\newcommand{\senssensaltfatt}{0.109}
\newcommand{\senssensaltfse}{0.024}
\newcommand{\senssensaltflower}{0.062}
\newcommand{\senssensaltfupper}{0.156}
\newcommand{\senssensaltfn}{6698}


% Início do documento
\begin{document}

% Slide título
\begin{frame}
\titlepage
\end{frame}

% Sumário
\begin{frame}{Sumário}
\tableofcontents
\end{frame}

% ===== SEÇÃO 1: INTRODUÇÃO =====
\section{Introdução}

\begin{frame}{Motivação}
\begin{itemize}
    \item A agricultura brasileira enfrenta o desafio de aumentar a produtividade em contexto de crescente variabilidade climática
    \vspace{0.3cm}
    \item Informação meteorológica precisa emerge como insumo produtivo crítico
    \vspace{0.3cm}
    \item \textbf{Lacuna na literatura}: ausência de evidências causais sobre o impacto econômico da expansão da infraestrutura meteorológica
    \vspace{0.3cm}
    \item Instalação escalonada de estações (2000-2019) oferece experimento natural
\end{itemize}

\begin{block}{Questão Central}
Qual é o impacto causal da instalação de estações meteorológicas sobre o PIB agropecuário?
\end{block}
\end{frame}

\begin{frame}{Objetivos}
\begin{columns}
\column{0.5\textwidth}
\textbf{Objetivo Geral:}
\begin{itemize}
    \item Estimar o efeito causal da instalação de estações meteorológicas sobre a produtividade agrícola
\end{itemize}

\column{0.5\textwidth}
\textbf{Objetivos Específicos:}
\begin{enumerate}
    \item Aplicar metodologia adequada para tratamento escalonado
    \item Quantificar o retorno econômico da infraestrutura
    \item Analisar a dinâmica temporal dos efeitos
    \item Validar robustez dos resultados
\end{enumerate}
\end{columns}

\vspace{0.5cm}
\begin{alertblock}{Contribuição Principal}
Primeira evidência causal rigorosa do impacto econômico de estações meteorológicas na agricultura brasileira
\end{alertblock}
\end{frame}

% ===== SEÇÃO 2: REVISÃO DA LITERATURA =====
\section{Revisão da Literatura}

\begin{frame}{Canais de Impacto da Informação Meteorológica}
\begin{columns}
\column{0.5\textwidth}
\textbf{Literatura Internacional:}
\begin{itemize}
    \item \textbf{Mavi \& Tupper (2004)}: três dimensões de impacto
    \begin{itemize}
        \item Planejamento estratégico
        \item Decisões táticas
        \item Resiliência sistêmica
    \end{itemize}
    \item \textbf{Weiss (2000)}: ajustes finos nas práticas
    \item \textbf{Rijks (2000)}: ganhos econômicos potenciais
\end{itemize}

\column{0.5\textwidth}
\textbf{Contexto Brasileiro:}
\begin{itemize}
    \item \textbf{Monteiro (2009)}: oscilações meteorológicas determinam produção
    \item \textbf{Carvalho et al. (2015)}: impacto climático na cana-de-açúcar
    \item \textbf{Vianna \& Sentelhas (2016)}: otimização via modelos agrometeorológicos
\end{itemize}
\end{columns}

\vspace{0.3cm}
\begin{block}{Gap Identificado}
Estudos existentes são predominantemente descritivos ou baseados em correlações
\end{block}
\end{frame}

% ===== SEÇÃO 3: METODOLOGIA =====
\section{Metodologia}

\begin{frame}{O Problema do DiD Tradicional com Tratamento Escalonado}
\begin{columns}
\column{0.6\textwidth}
\textbf{Two-Way Fixed Effects (TWFE) tradicional:}
$$Y_{it} = \alpha_i + \lambda_t + \beta D_{it} + \epsilon_{it}$$

\textbf{Problemas identificados:}
\begin{itemize}
    \item Usa unidades já tratadas como controles
    \item Confunde efeitos heterogêneos no tempo
    \item Pode gerar pesos negativos
    \item Estimativas potencialmente enviesadas
\end{itemize}

\column{0.4\textwidth}
\begin{figure}
\centering
\begin{tikzpicture}[scale=0.8]
\draw[->] (0,0) -- (4,0) node[right] {Tempo};
\draw[->] (0,0) -- (0,3) node[above] {Unidades};
\draw[thick,blue] (0.5,0.5) -- (1.5,0.5) -- (1.5,1) -- (3.5,1);
\draw[thick,red] (0.5,1.5) -- (2.5,1.5) -- (2.5,2) -- (3.5,2);
\draw[thick,green] (0.5,2.5) -- (3.5,2.5);
\node at (1,0.5) {G1};
\node at (2,1.5) {G2};
\node at (2,2.5) {G3};
\end{tikzpicture}
\caption{Tratamento Escalonado}
\end{figure}
\end{columns}

\begin{alertblock}{Solução}
Metodologia de Callaway \& Sant'Anna (2021) para DiD com múltiplos períodos
\end{alertblock}
\end{frame}

\begin{frame}{Arcabouço de Callaway \& Sant'Anna (2021)}
\textbf{Abordagem em três etapas:}

\begin{enumerate}
    \item \textbf{Estimação de ATT(g,t)}
    \begin{itemize}
        \item Efeito para grupo $g$ no período $t$
        \item Comparação com unidades ainda não tratadas
    \end{itemize}
    
    \item \textbf{Agregação dos efeitos}
    \begin{itemize}
        \item Efeito médio geral: $\theta_{sel}^O$
        \item Event study: $\theta_{es}^{bal}(e)$
    \end{itemize}
    
    \item \textbf{Inferência robusta}
    \begin{itemize}
        \item Bootstrap multiplicativo
        \item Clustering ao nível da microrregião
    \end{itemize}
\end{enumerate}

\begin{block}{Estimador Doubly Robust}
Combina modelagem do resultado e ponderação por probabilidade inversa, permanecendo consistente se pelo menos um modelo estiver correto
\end{block}
\end{frame}

\begin{frame}{Estratégia de Identificação}
\begin{columns}
\column{0.5\textwidth}
\textbf{Definição do Tratamento:}
\begin{itemize}
    \item Instalação da primeira estação meteorológica automática na microrregião
    \item $G_i$ = ano da primeira estação
    \item $G_i = 0$ se nunca tratada
\end{itemize}

\textbf{Variável Dependente:}
$$Y_{it} = \ln(1 + \text{PIB\_Agro}_{it})$$

\column{0.5\textwidth}
\textbf{Covariáveis:}
\begin{itemize}
    \item Log da área plantada
    \item Log da população
    \item Log do PIB per capita
    \item Log da densidade estadual de estações
\end{itemize}

\textbf{Grupo de Controle:}
\begin{itemize}
    \item "Not-yet-treated"
    \item Maximiza eficiência estatística
\end{itemize}
\end{columns}

\begin{block}{Pressuposto Central}
Tendências paralelas condicionais entre tratados e controles
\end{block}
\end{frame}

% ===== SEÇÃO 4: DADOS =====
\section{Dados}

\begin{frame}{Por que Cana-de-Açúcar?}
\begin{columns}
\column{0.5\textwidth}
\textbf{Relevância Econômica:}
\begin{itemize}
    \item 3º maior produtor mundial
    \item Presente em 490 microrregiões
    \item R\$ 52 bilhões em valor de produção (2023)
\end{itemize}

\textbf{Características Técnicas:}
\begin{itemize}
    \item Alta sensibilidade climática
    \item Ciclo produtivo longo (12-18 meses)
    \item Janelas críticas de plantio/colheita
\end{itemize}

\column{0.5\textwidth}
\textbf{Vantagens Metodológicas:}
\begin{itemize}
    \item Dados completos e confiáveis
    \item Produção contínua no período
    \item Distribuição geográfica ampla
    \item Variação temporal na adoção de estações
\end{itemize}

\begin{alertblock}{Implicação}
Cultura ideal para identificar impactos de informação meteorológica
\end{alertblock}
\end{columns}
\end{frame}

\begin{frame}{Construção do Dataset}
\begin{columns}
\column{0.5\textwidth}
\textbf{Fontes de Dados:}
\begin{itemize}
    \item INMET: 610 estações meteorológicas
    \item IBGE: PIB municipal e população
    \item PAM-IBGE: produção de cana-de-açúcar
    \item Período: 2003-2023
\end{itemize}

\textbf{Unidade de Análise:}
\begin{itemize}
    \item Microrregiões (490 produtoras)
    \item Agregação de dados municipais
    \item Painel balanceado: 10.290 obs
\end{itemize}

\column{0.5\textwidth}
\begin{figure}
\centering
\includegraphics[width=\textwidth]{../../../data/outputs/descriptive_analysis/distribuicao_temporal_tratamento.png}
\caption{Distribuição Temporal do Tratamento}
\end{figure}
\end{columns}

\begin{block}{Transparência}
Código completo disponível em: \url{github.com/danielcavalli/tcc-ie-ufrj-2024}
\end{block}
\end{frame}

% ===== SEÇÃO 5: RESULTADOS =====
\section{Resultados}

\begin{frame}{Resultado Principal}
\begin{center}
\Large
\textbf{ATT = \mainatt{} (8,2\%)}\\
\normalsize
EP = \mainse, p = 0,0103\\
IC 95\%: [0,0194; 0,1448]
\end{center}

\begin{columns}
\column{0.5\textwidth}
\begin{table}[h]
\centering
\small
\begin{tabular}{lcc}
\toprule
Especificação & ATT & P-valor \\
\midrule
\textbf{Doubly Robust} & \textbf{0,082} & \textbf{0,010} \\
IPW & 0,094 & 0,003 \\
Regression & 0,066 & 0,030 \\
Sem covariáveis & 0,110 & 0,000 \\
\midrule
Never-treated & 0,080 & 0,026 \\
\bottomrule
\end{tabular}
\end{table}

\column{0.5\textwidth}
\textbf{Interpretação:}
\begin{itemize}
    \item Aumento de 8,2\% no PIB agropecuário
    \item Equivalente a 2+ anos de crescimento típico
    \item Robusto a diferentes especificações
    \item Economicamente significativo
\end{itemize}
\end{columns}

\begin{alertblock}{Implicação}
Retorno econômico supera amplamente os custos de instalação (R\$ 223 mil/estação)
\end{alertblock}
\end{frame}

\begin{frame}{Magnitude Econômica do Impacto}
\begin{columns}
\column{0.5\textwidth}
\textbf{Impacto por Microrregião:}
\begin{itemize}
    \item PIB agro médio: R\$ 580 milhões/ano
    \item Ganho de 8,2\%: R\$ 47,6 milhões/ano
    \item Payback: < 6 meses
\end{itemize}

\textbf{Projeção Nacional:}
\begin{itemize}
    \item 351 microrregiões tratadas
    \item Ganho agregado: R\$ 16,7 bilhões/ano
    \item 139 microrregiões sem estações
    \item Potencial não realizado: R\$ 6,6 bilhões/ano
\end{itemize}

\column{0.5\textwidth}
\begin{block}{Análise Custo-Benefício}
\begin{itemize}
    \item Custo médio por estação: R\$ 223 mil
    \item Retorno anual: R\$ 47,6 milhões
    \item Taxa de retorno: 213x ao ano
\end{itemize}
\end{block}

\begin{alertblock}{Conclusão}
Subinvestimento histórico representa oportunidade perdida de R\$ 6,6 bilhões anuais
\end{alertblock}
\end{columns}
\end{frame}

\begin{frame}{Event Study - Dinâmica Temporal}
\begin{figure}
\centering
\includegraphics[width=0.85\textwidth]{../../../data/outputs/presentation/event_study_enhanced.png}
\end{figure}

\begin{columns}
\column{0.5\textwidth}
\textbf{Pré-tratamento:}
\begin{itemize}
    \item Ausência de tendências
    \item \textbf{Teste formal: F = 1,136 (p = 0,322)}
    \item Coeficientes oscilam aleatoriamente
    \item Forte evidência de parallel trends
\end{itemize}

\column{0.5\textwidth}
\textbf{Pós-tratamento:}
\begin{itemize}
    \item Efeitos positivos persistentes
    \item Difusão gradual dos benefícios
    \item Consistente com processo de aprendizado
    \item Estabilização em ~10\% após 5 anos
\end{itemize}
\end{columns}
\end{frame}

\begin{frame}{Tendências Paralelas - Validação Visual}
\begin{figure}
\centering
\includegraphics[width=0.85\textwidth]{../../../data/outputs/parallel_trends_complete_pib_agro_normalized.png}
\end{figure}

\begin{itemize}
    \item Evolução similar antes do tratamento
    \item Divergência clara após instalação das estações
    \item Suporte visual para o pressuposto de identificação
\end{itemize}
\end{frame}

% ===== SEÇÃO 6: ROBUSTEZ =====
\section{Testes de Robustez}

\begin{frame}{Testes Placebo}
\begin{columns}
\column{0.5\textwidth}
\textbf{1. PIB Não-Agropecuário:}
\begin{itemize}
    \item ATT = 0,015 (p = 0,427)
    \item Não significativo
    \item Confirma especificidade setorial
\end{itemize}

\textbf{2. Randomização Única:}
\begin{itemize}
    \item ATT = -0,024 (p = 0,485)
    \item Tratamento aleatório não gera efeitos
\end{itemize}

\column{0.5\textwidth}
\textbf{3. Randomização Múltipla:}
\begin{figure}
\centering
\includegraphics[width=\textwidth]{../../../data/outputs/placebo_distribution.png}
\end{figure}
\begin{itemize}
    \item 50 simulações
    \item P-valor empírico < 0,01
\end{itemize}
\end{columns}

\begin{block}{Conclusão}
Forte evidência de efeito causal genuíno, não artefato estatístico
\end{block}
\end{frame}

\begin{frame}{Análise de Sensibilidade}
\begin{figure}
\centering
\includegraphics[width=0.9\textwidth]{../../../data/outputs/robustness_plot.png}
\end{figure}

\begin{columns}
\column{0.5\textwidth}
\textbf{Diferentes períodos:}
\begin{itemize}
    \item Completo: 12,6\%
    \item Sem COVID: 11,7\%
    \item Sem início: 13,0\%
\end{itemize}

\column{0.5\textwidth}
\textbf{Diferentes métodos:}
\begin{itemize}
    \item DR: 8,2\%
    \item IPW: 9,4\%
    \item REG: 6,6\%
\end{itemize}
\end{columns}
\end{frame}

% ===== SEÇÃO 7: MECANISMOS =====
\section{Mecanismos e Discussão}

\begin{frame}{Mecanismos de Transmissão}
\begin{columns}
\column{0.6\textwidth}
\textbf{Canais identificados na literatura:}
\begin{enumerate}
    \item \textbf{Otimização do calendário agrícola}
    \begin{itemize}
        \item Timing de plantio e colheita
        \item Baseado em previsões precisas
    \end{itemize}
    
    \item \textbf{Gestão hídrica eficiente}
    \begin{itemize}
        \item Ajuste de irrigação
        \item Redução de desperdício
    \end{itemize}
    
    \item \textbf{Redução de perdas}
    \begin{itemize}
        \item Antecipação a eventos extremos
        \item Medidas preventivas
    \end{itemize}
\end{enumerate}

\column{0.4\textwidth}
\begin{block}{Processo de Difusão}
\begin{itemize}
    \item Aprendizado gradual
    \item Integração às decisões
    \item Efeitos de rede
\end{itemize}
\end{block}

\begin{alertblock}{Evidência}
Padrão temporal consistente com adaptação tecnológica
\end{alertblock}
\end{columns}
\end{frame}

\begin{frame}{Implicações para Políticas Públicas}
\begin{columns}
\column{0.5\textwidth}
\textbf{Contexto Atual:}
\begin{itemize}
    \item Dez/2024: MAPA anuncia R\$ 49 milhões
    \item 220 novas estações
    \item R\$ 223 mil/estação
\end{itemize}

\textbf{Nossa Evidência:}
\begin{itemize}
    \item Retorno de 8,2\% no PIB agro
    \item Benefícios superam custos
    \item 29\% das microrregiões sem estações
\end{itemize}

\column{0.5\textwidth}
\textbf{Recomendações:}
\begin{enumerate}
    \item \textbf{Expansão estratégica}
    \begin{itemize}
        \item Priorizar áreas não cobertas
        \item Foco em regiões produtoras
    \end{itemize}
    
    \item \textbf{Integração de dados}
    \begin{itemize}
        \item Sistemas como AGRITEMPO
        \item Acesso facilitado
    \end{itemize}
    
    \item \textbf{Capacitação}
    \begin{itemize}
        \item Uso efetivo das informações
        \item Assistência técnica
    \end{itemize}
\end{enumerate}
\end{columns}

\begin{block}{Urgência}
Atrasos na implementação representam perdas econômicas significativas
\end{block}
\end{frame}

% ===== SEÇÃO 8: LIMITAÇÕES =====
\section{Limitações e Pesquisa Futura}

\begin{frame}{Limitações do Estudo}
\begin{columns}
\column{0.5\textwidth}
\textbf{Limitações Identificadas:}
\begin{enumerate}
    \item \textbf{Desbalanceamento de covariáveis}
    \begin{itemize}
        \item Mitigado pelo DR
        \item Diferenças observáveis
    \end{itemize}
    
    \item \textbf{Composição dos pesos}
    \begin{itemize}
        \item Coortes iniciais: 50,8\%
        \item Sem dominância extrema
    \end{itemize}
    
    \item \textbf{Heterogeneidade não observada}
    \begin{itemize}
        \item Tamanho de propriedade
        \item Nível educacional
    \end{itemize}
\end{enumerate}

\column{0.5\textwidth}
\textbf{Direções Futuras:}
\begin{enumerate}
    \item \textbf{Modelagem espacial}
    \begin{itemize}
        \item Spillovers explícitos
        \item Dependência espacial
    \end{itemize}
    
    \item \textbf{Dados de alta frequência}
    \begin{itemize}
        \item Mensais/trimestrais
        \item Eventos climáticos
    \end{itemize}
    
    \item \textbf{Análise por cultura}
    \begin{itemize}
        \item Impactos diferenciados
        \item Outras culturas além da cana
    \end{itemize}
\end{enumerate}
\end{columns}

\begin{alertblock}{Apesar das Limitações}
Resultados robustos fornecem primeira evidência causal rigorosa
\end{alertblock}
\end{frame}

% ===== SEÇÃO 9: CONCLUSÕES =====
\section{Conclusões}

\begin{frame}{Conclusões Principais}
\begin{enumerate}
    \item \textbf{Evidência Causal Pioneira}
    \begin{itemize}
        \item Primeira quantificação rigorosa do impacto
        \item ATT = 8,2\% (p = 0,010)
        \item Efeito economicamente significativo
    \end{itemize}
    
    \item \textbf{Validação Metodológica}
    \begin{itemize}
        \item Superioridade do DiD escalonado
        \item Importância de métodos adequados
        \item Modelo para futuras aplicações
    \end{itemize}
    
    \item \textbf{Implicações Práticas}
    \begin{itemize}
        \item Justifica expansão da rede
        \item Alternativa à expansão da fronteira agrícola
        \item Estratégia de adaptação climática
    \end{itemize}
\end{enumerate}

\begin{block}{Mensagem Final}
Investimento em informação meteorológica é estratégia custo-efetiva para aumentar produtividade agrícola sustentavelmente
\end{block}
\end{frame}

% ===== REFERÊNCIAS =====
\begin{frame}[allowframebreaks]{Referências Principais}
\footnotesize
\begin{itemize}
    \item CALLAWAY, B.; SANT'ANNA, P. H. Difference-in-differences with multiple time periods. \textit{Journal of Econometrics}, v. 225, n. 2, p. 200-230, 2021.
    
    \item GOODMAN-BACON, A. Difference-in-differences with variation in treatment timing. \textit{Journal of Econometrics}, v. 225, n. 2, p. 254-277, 2021.
    
    \item MAVI, H. S.; TUPPER, G. J. \textit{Agrometeorology: principles and applications of climate studies in agriculture}. CRC Press, 2004.
    
    \item MONTEIRO, J. E. (Ed.). \textit{Agrometeorologia dos cultivos: o fator meteorológico na produção agrícola}. Brasília: INMET, 2009.
    
    \item SANT'ANNA, P. H.; ZHAO, J. Doubly robust difference-in-differences estimators. \textit{Journal of Econometrics}, v. 219, n. 1, p. 101-122, 2020.
    
    \item SUN, L.; ABRAHAM, S. Estimating dynamic treatment effects in event studies with heterogeneous treatment effects. \textit{Journal of Econometrics}, v. 225, n. 2, p. 175-199, 2021.
\end{itemize}
\end{frame}

% Slide final
\begin{frame}{}
\centering
\Large
\textbf{Obrigado!}\\
\vspace{1cm}
\normalsize
Daniel Cavalli\\
\texttt{daniel.cavalli@ie.ufrj.br}\\
\vspace{0.5cm}
Código disponível em:\\
\url{github.com/danielcavalli/tcc-ie-ufrj-2024}
\end{frame}

% ===== SLIDES DE BACKUP =====
\appendix

\begin{frame}{Backup: Análise de Poder Estatístico}
\begin{figure}
\centering
\includegraphics[width=0.75\textwidth]{../../../data/outputs/additional_figures/power_analysis_simulation.png}
\end{figure}

\begin{itemize}
    \item Para o efeito de 8,2\%: poder de 92,1\% ($\alpha$ = 0,05)
    \item Design adequado para detectar efeitos economicamente relevantes
    \item Mínimo efeito detectável com 80\% de poder: ~5,5\%
\end{itemize}
\end{frame}

\begin{frame}{Backup: Processo de Integração dos Dados}
\begin{columns}
\column{0.5\textwidth}
\textbf{Fontes de Dados:}
\begin{itemize}
    \item \textbf{INMET}: 610 estações meteorológicas
    \item \textbf{IBGE}: PIB municipal e população  
    \item \textbf{PAM-IBGE}: Produção agrícola detalhada
\end{itemize}

\textbf{Plataforma de Integração:}
\begin{itemize}
    \item Google BigQuery + basedosdados
    \item Acesso unificado às bases públicas
    \item SQL otimizado para grandes volumes
\end{itemize}

\column{0.5\textwidth}
\textbf{Pipeline de Processamento:}
\begin{enumerate}
    \item Extração via API Python
    \item Agregação município → microrregião
    \item Validação cruzada de mapeamentos
    \item Tratamento de dados faltantes
    \item Construção do painel balanceado
\end{enumerate}

\begin{block}{Dataset Final}
490 microrregiões × 21 anos = 10.290 obs\\
0\% de valores faltantes
\end{block}
\end{columns}
\end{frame}

\begin{frame}{Backup: Heterogeneidade Regional}
\begin{table}[h]
\centering
\begin{tabular}{lccc}
\toprule
Região & ATT & EP & N Tratadas \\
\midrule
Norte & 0,095 & (0,041) & 8 \\
Nordeste & 0,076** & (0,035) & 45 \\
Centro-Oeste & 0,091*** & (0,028) & 22 \\
Sudeste & 0,083*** & (0,024) & 48 \\
Sul & 0,089** & (0,038) & 8 \\
\bottomrule
\end{tabular}
\end{table}

\begin{itemize}
    \item Efeitos positivos em todas as regiões
    \item Magnitude similar (7,6\% a 9,5\%)
    \item Maior precisão em regiões com mais observações
    \item Sugere validade externa dos resultados
\end{itemize}
\end{frame}

\begin{frame}{Backup: Detalhes da Implementação Computacional}
\begin{columns}
\column{0.5\textwidth}
\textbf{Software e Pacotes:}
\begin{itemize}
    \item R 4.5.0 + pacote \texttt{did} v2.1.2
    \item Python 3.11 + \texttt{basedosdados}
    \item Google BigQuery API
    \item Sistema \texttt{renv} para reprodutibilidade
\end{itemize}

\textbf{Especificações Técnicas:}
\begin{itemize}
    \item Bootstrap: 1.000 replicações
    \item Clustering: nível microrregião
    \item Inferência: bandas uniformes
\end{itemize}

\column{0.5\textwidth}
\textbf{Escolhas Metodológicas:}
\begin{itemize}
    \item Estimador: Doubly Robust
    \item Controle: not-yet-treated
    \item Covariáveis: pré-tratamento
    \item Agregação: balanceada (event study)
\end{itemize}

\textbf{Validações:}
\begin{itemize}
    \item Convergência do bootstrap
    \item Estabilidade numérica
    \item Sensibilidade a outliers
\end{itemize}
\end{columns}
\end{frame}

\end{document}
