% Apresentação de Defesa de TCC - Daniel Cavalli (VERSÃO 5 - CORRIGIDA)
% Instituto de Economia - UFRJ
% 2025
% ALINHADA COM A TESE TCC_DanielCavalli_ABNT2.tex

\documentclass[10pt,aspectratio=169]{beamer}

% Pacotes essenciais
\usepackage[brazilian]{babel}
\usepackage[utf8]{inputenc}
\usepackage[T1]{fontenc}
\usepackage{amsmath,amssymb}
\usepackage{graphicx}
\usepackage{booktabs}
\usepackage{multirow}
\usepackage{tikz}
\usetikzlibrary{positioning}
\usepackage{pgfplots}
\pgfplotsset{compat=1.17}

% Configuração do tema
\usetheme{Boadilla}
\usecolortheme{default}
\setbeamertemplate{navigation symbols}{}
\setbeamertemplate{footline}[frame number]

% Remove sombras e elementos decorativos
\setbeamertemplate{blocks}[rounded][shadow=false]
\setbeamertemplate{title page}[default][colsep=-4bp,rounded=false,shadow=false]
\setbeamertemplate{frametitle}[default][colsep=-2bp,rounded=false,shadow=false,center]

% Cores personalizadas UFRJ
\definecolor{ufrjblue}{RGB}{0,53,96}
\definecolor{ufrjgreen}{RGB}{0,128,0}
\definecolor{ufrjlightblue}{RGB}{51,102,153}
\definecolor{ufrjgray}{RGB}{100,100,100}

% Aplicação sutil das cores
\setbeamercolor{structure}{fg=ufrjblue}
\setbeamercolor{frametitle}{bg=white,fg=ufrjblue}
\setbeamercolor{title}{fg=ufrjblue}
\setbeamercolor{block title}{bg=ufrjlightblue,fg=white}
\setbeamercolor{block body}{bg=ufrjlightblue!10}
\setbeamercolor{block title alerted}{bg=red!90,fg=white}
\setbeamercolor{block body alerted}{bg=red!10}

% Configuração de itemize/enumerate
\setbeamercolor{itemize item}{fg=ufrjblue}
\setbeamercolor{itemize subitem}{fg=ufrjlightblue}
\setbeamercolor{enumerate item}{fg=ufrjblue}

% Configurações de fonte
\usefonttheme{professionalfonts}
\setbeamerfont{title}{size=\Large,series=\bfseries}
\setbeamerfont{frametitle}{size=\large,series=\bfseries}
\setbeamerfont{block title}{size=\normalsize,series=\bfseries}

% Configurações adicionais para design limpo
\setbeamertemplate{itemize items}[circle]
\setbeamertemplate{enumerate items}[default]
\setbeamertemplate{section in toc}[circle]

% Remove linhas de separação
\setbeamertemplate{headline}{}
\setbeamertemplate{footline}{
    \leavevmode%
    \hbox{%
    \begin{beamercolorbox}[wd=.333333\paperwidth,ht=2.25ex,dp=1ex,center]{author in head/foot}%
        \usebeamerfont{author in head/foot}\insertshortauthor
    \end{beamercolorbox}%
    \begin{beamercolorbox}[wd=.333333\paperwidth,ht=2.25ex,dp=1ex,center]{title in head/foot}%
        \usebeamerfont{title in head/foot}\insertshorttitle
    \end{beamercolorbox}%
    \begin{beamercolorbox}[wd=.333333\paperwidth,ht=2.25ex,dp=1ex,right]{date in head/foot}%
        \usebeamerfont{date in head/foot}\insertshortdate{}\hspace*{2em}
        \insertframenumber{} / \inserttotalframenumber\hspace*{2ex}
    \end{beamercolorbox}}%
    \vskip0pt%
}

% Ajuste de margens para mais espaço
\setbeamersize{text margin left=1em,text margin right=1em}

% Informações do documento
\title[Estações Meteorológicas e Produtividade Agrícola]{Impacto de Estações Meteorológicas na Produtividade de Cana-de-Açúcar: \\ Uma Aplicação de Diferenças em Diferenças com Tratamento Escalonado}

\author[Daniel Cavalli]{Daniel Cavalli \\ \small Orientador: Prof. Romero Rocha}

\institute[IE-UFRJ]{
  Instituto de Economia\\
  Universidade Federal do Rio de Janeiro
}

\date{2025}

% Importa valores automaticamente gerados
% Arquivo gerado automaticamente por generate_latex_values.r
% Última atualização: 2025-11-16

% Valores do teste placebo aleatório
\newcommand{\placebotruatt}{0.118}
\newcommand{\placebopvalue}{0.020}
\newcommand{\placebolower}{-0.041}
\newcommand{\placeboupper}{0.043}
\newcommand{\placebonsims}{50}
\newcommand{\placebomean}{-0.002}
\newcommand{\placebosd}{0.023}

% Valores de precisão do p-valor (Monte Carlo)
\newcommand{\placebopvaluese}{0.020}
\newcommand{\placebopvaluelower}{0.000}
\newcommand{\placebopvalueupper}{0.058}
\newcommand{\placebonextremes}{0}

% Valores formatados para texto
\newcommand{\placebotruattpct}{11.8\%}
\newcommand{\placebopvaluepct}{2.0\%}

% Valores do modelo principal (resultado detalhado)
\newcommand{\mainatt}{0.118}
\newcommand{\mainse}{0.042}
\newcommand{\mainz}{2.789}
\newcommand{\mainp}{0.005}
\newcommand{\maincilower}{0.035}
\newcommand{\mainciupper}{0.201}
\newcommand{\mainattpct}{11.8\%}

% Grupo de Controle: Never-treated
\newcommand{\nevertreatedatt}{0.136}
\newcommand{\nevertreatedse}{0.053}
\newcommand{\nevertreatedlower}{0.032}
\newcommand{\nevertreatedupper}{0.240}

% Especificação: Sem Covariáveis
\newcommand{\nocovatt}{0.124}
\newcommand{\nocovse}{0.042}
\newcommand{\nocovlower}{0.042}
\newcommand{\nocover}{0.207}

% Especificação: IPW
\newcommand{\ipwatt}{0.122}
\newcommand{\ipwse}{0.042}
\newcommand{\ipwlower}{0.041}
\newcommand{\ipwupper}{0.204}

% Especificação: Regressão de Resultado
\newcommand{\regatt}{0.114}
\newcommand{\regse}{0.043}
\newcommand{\reglower}{0.031}
\newcommand{\regupper}{0.198}

% Valores da análise de sensibilidade temporal
% Completo (2003-2023)
\newcommand{\sensfullatt}{0.115}
\newcommand{\sensfullse}{0.027}
\newcommand{\sensfulllower}{0.062}
\newcommand{\sensfullupper}{0.169}
\newcommand{\sensfulln}{7486}
% Excluindo Início (2006-2023)
\newcommand{\sensnostartatt}{0.115}
\newcommand{\sensnostartse}{0.031}
\newcommand{\sensnostartlower}{0.054}
\newcommand{\sensnostartupper}{0.177}
\newcommand{\sensnostartn}{6304}
% Excluindo Final (2003-2019)
\newcommand{\senssensaltcatt}{0.109}
\newcommand{\senssensaltcse}{0.025}
\newcommand{\senssensaltclower}{0.060}
\newcommand{\senssensaltcupper}{0.158}
\newcommand{\senssensaltcn}{6698}
% Excluindo COVID (2003-2019)
\newcommand{\sensnocovidatt}{0.109}
\newcommand{\sensnocovidse}{0.025}
\newcommand{\sensnocovidlower}{0.059}
\newcommand{\sensnocovidupper}{0.159}
\newcommand{\sensnocovidn}{6698}
% Pré-COVID (2003-2019)
\newcommand{\senssensaltfatt}{0.109}
\newcommand{\senssensaltfse}{0.024}
\newcommand{\senssensaltflower}{0.062}
\newcommand{\senssensaltfupper}{0.156}
\newcommand{\senssensaltfn}{6698}


% Início do documento
\begin{document}

% Slide título
\begin{frame}
\titlepage
\end{frame}

% Sumário
\begin{frame}{Roteiro}
\tableofcontents
\end{frame}

% ===== SEÇÃO 1: MOTIVAÇÃO E PROBLEMA =====
\section{Motivação e Problema de Pesquisa}

\begin{frame}{Contexto}
\begin{columns}
\column{0.5\textwidth}
\textbf{Desafio da Agricultura:}
\begin{itemize}
    \item Crescente variabilidade climática
    \item Necessidade de aumento de produtividade
    \item Informação como insumo produtivo
\end{itemize}

\vspace{0.5cm}
\textbf{Lacuna na Literatura:}
\begin{itemize}
    \item Estudos predominantemente descritivos
    \item Ausência de evidências causais
    \item Falta quantificação econômica rigorosa
\end{itemize}

\column{0.5\textwidth}
\begin{block}{Oportunidade}
Instalação escalonada de estações meteorológicas (2003-2021) cria experimento natural
\end{block}

\vspace{0.3cm}
\begin{alertblock}{Questão Central}
Qual o impacto causal da instalação de estações meteorológicas sobre o valor de produção de cana-de-açúcar?
\end{alertblock}
\end{columns}
\end{frame}

\begin{frame}{Mecanismo Causal}
\begin{figure}
\centering
\begin{tikzpicture}[scale=0.9, every node/.style={font=\small}]
% Caixas
\node[draw, rectangle, minimum width=2.5cm, minimum height=0.8cm, align=center] (est) at (0,0) {Estação\\Meteorológica};
\node[draw, rectangle, minimum width=2.5cm, minimum height=0.8cm, align=center] (dados) at (4,0) {Dados\\Climáticos};
\node[draw, rectangle, minimum width=2.5cm, minimum height=0.8cm, align=center] (dec) at (8,0) {Decisões\\Otimizadas};
\node[draw, rectangle, minimum width=2.5cm, minimum height=0.8cm, align=center] (pib) at (12,0) {$\uparrow$ Valor\\Produção};

% Setas
\draw[->, thick] (est) -- (dados);
\draw[->, thick] (dados) -- (dec);
\draw[->, thick] (dec) -- (pib);

% Exemplos abaixo
\node[below of=dados, node distance=0.8cm, align=center, font=\tiny] {Temperatura\\Precipitação\\Eventos extremos};
\node[below of=dec, node distance=0.8cm, align=center, font=\tiny] {Timing irrigação\\Variedades\\Manejo};
\end{tikzpicture}
\end{figure}

\vspace{0.3cm}
\textbf{Literatura identifica três dimensões} (Mavi e Tupper, 2004):
\begin{enumerate}
    \item Planejamento estratégico (escolha de culturas, épocas)
    \item Decisões táticas (irrigação, defensivos)
    \item Construção de resiliência
\end{enumerate}

\vspace{0.2cm}
\textbf{+ Canal indireto:} Modelos de simulação (DSSAT-CANEGRO) para otimização de irrigação de salvamento
\end{frame}

\begin{frame}{Por Que Cana-de-Açúcar?}
\begin{columns}
\column{0.6\textwidth}
\textbf{Irrigação de Salvamento (Atlas ANA 2017):}
\begin{itemize}
    \item \textbf{>90\% da área irrigada} de cana usa salvamento
    \item Aplicações pontuais (20-80mm/ano)
    \item Momento crítico: pós-colheita para rebrota
    \item \textbf{Timing preciso é essencial}
\end{itemize}

\vspace{0.3cm}
\textbf{Por que informação meteorológica importa:}
\begin{itemize}
    \item Previsão de chuvas permite evitar desperdício
    \item Identificação de janelas críticas de déficit
    \item Otimização logística de equipamentos móveis
\end{itemize}

\column{0.4\textwidth}
\begin{block}{Contraste}
\textbf{Soja:} Irrigação suplementar distribuída, menos sensível a timing pontual

\vspace{0.3cm}
\textbf{Arroz:} Inundação contínua, decisão no estabelecimento inicial
\end{block}

\vspace{0.3cm}
\begin{alertblock}{Hipótese}
Efeito específico à cana devido a características únicas do manejo hídrico
\end{alertblock}
\end{columns}
\end{frame}

% ===== SEÇÃO 2: ESTRATÉGIA EMPÍRICA =====
\section{Estratégia Empírica}

\begin{frame}{Dados e Definições}
\begin{columns}
\column{0.6\textwidth}
\textbf{Amostra:}
\begin{itemize}
    \item 225 microrregiões produtoras de cana
    \item (41,3\% das 558 microrregiões do Brasil)
    \item Período: 2003-2021 (19 anos)
    \item 4.275 observações (225 × 19)
\end{itemize}

\vspace{0.3cm}
\textbf{Definições:}
\begin{itemize}
    \item \textbf{Tratamento}: Instalação da primeira estação
    \item \textbf{Outcome}: $Y_{it} = \ln(1 + \text{ValorProducaoCana}_{it})$
    \item \textbf{Grupo}: $G_i$ = ano da primeira estação
\end{itemize}

\column{0.4\textwidth}
\begin{figure}
\centering
\includegraphics[width=\textwidth]{../../../data/outputs/descriptive_analysis/distribuicao_temporal_tratamento.png}
\caption{Distribuição temporal do tratamento}
\end{figure}

\begin{block}{Cobertura}
Do total de 558 microrregiões: 394 tratadas (70,6\%), 164 controle (29,4\%)
\end{block}
\end{columns}
\end{frame}

\begin{frame}{Filtro Amostral: Por Que Apenas Produtoras de Cana?}
\textbf{Estratégia: Filtro específico por cultura}
\begin{itemize}
    \item Incluir apenas microrregiões que produziram cana em \textit{ao menos um ano} entre 2003-2021
    \item Evita \textbf{zeros estruturais} (regiões onde cana nunca é viável)
    \item Preserva \textbf{zeros econômicos} (decisões de entrada/saída temporal)
    \item Mantém validade do contrafactual
\end{itemize}

\vspace{0.3cm}
\begin{columns}
\column{0.5\textwidth}
\textbf{Vantagens:}
\begin{itemize}
    \item Comparação entre unidades similares
    \item Captação de margens extensiva + intensiva
    \item Redução de ruído idiossincrático
\end{itemize}

\column{0.5\textwidth}
\begin{alertblock}{Consistência}
Mesmo critério aplicado para soja e arroz nas análises placebo
\end{alertblock}
\end{columns}
\end{frame}

\begin{frame}{O Problema do Tratamento Escalonado}
\begin{columns}
\column{0.6\textwidth}
\textbf{TWFE Tradicional:}
$$Y_{it} = \alpha_i + \lambda_t + \beta D_{it} + \epsilon_{it}$$

\textbf{Problemas Identificados:}
\begin{itemize}
    \item Usa já tratados como controle
    \item Pesos potencialmente negativos
    \item Viés com efeitos heterogêneos
\end{itemize}

\vspace{0.3cm}
Goodman-Bacon (2021), Sun e Abraham (2021)

\column{0.4\textwidth}
\begin{figure}
\centering
\begin{tikzpicture}[scale=0.8]
\draw[->] (0,0) -- (5,0) node[right] {$t$};
\draw[->] (0,0) -- (0,3.5) node[above] {Unidades};

% Grupos
\draw[thick,blue] (0.5,0.5) -- (1.5,0.5);
\draw[thick,blue,dashed] (1.5,0.5) -- (1.5,0.8);
\draw[thick,blue] (1.5,0.8) -- (4.5,0.8);

\draw[thick,red] (0.5,1.5) -- (2.5,1.5);
\draw[thick,red,dashed] (2.5,1.5) -- (2.5,1.8);
\draw[thick,red] (2.5,1.8) -- (4.5,1.8);

\draw[thick,green] (0.5,2.5) -- (4.5,2.5);

% Problema
\draw[<->, orange, thick] (1.5,0.9) -- (2.5,1.4);
\node[orange, right, align=center] at (2.7,1.15) {\tiny Comparação\\\tiny problemática};
\end{tikzpicture}
\caption{Comparações inadequadas no TWFE}
\end{figure}
\end{columns}
\end{frame}

% ===== SEÇÃO 3: METODOLOGIA =====
\section{Metodologia: Callaway e Sant'Anna (2021)}

\begin{frame}{Solução: DiD com Múltiplos Períodos}
\textbf{Abordagem em 3 etapas:}

\vspace{0.3cm}
\begin{enumerate}
    \item \textbf{Estimação Desagregada}
    \begin{itemize}
        \item ATT(g,t): efeito para grupo $g$ no tempo $t$
        \item Comparação apenas com "not-yet-treated"
    \end{itemize}

    \vspace{0.3cm}
    \item \textbf{Agregação}
    \begin{itemize}
        \item Overall ATT: $\theta_{sel}^O = \sum_g \theta_{sel}(g) \cdot P(G=g|G \leq T)$
        \item Event study: $\theta_{es}^{bal}(e)$ para tempo relativo $e$
    \end{itemize}

    \vspace{0.3cm}
    \item \textbf{Inferência}
    \begin{itemize}
        \item Bootstrap multiplicativo (1.000 replicações)
        \item Clustering ao nível da microrregião
    \end{itemize}
\end{enumerate}

\vspace{0.3cm}
\textbf{Estimador:} Doubly Robust (DR) - consistente se outcome regression OU propensity score correto
\end{frame}

\begin{frame}{Validação dos Pressupostos}
\begin{columns}
\column{0.5\textwidth}
\textbf{1. Tendências Paralelas}
\begin{itemize}
    \item Visual: Event study pré-tratamento
    \item Formal: Teste F por coorte
    \item Resultado: F = 1,136 (p = 0,322)
\end{itemize}

\vspace{0.3cm}
\textbf{2. No Anticipation}
\begin{itemize}
    \item Efeitos apenas após instalação física
    \item Se violado: estimativas conservadoras
\end{itemize}

\column{0.5\textwidth}
\textbf{3. Tratamento Irreversível}
\begin{itemize}
    \item Estações permanecem ativas
    \item Consistente com dados observados
\end{itemize}

\vspace{0.3cm}
\textbf{4. Overlap}
\begin{itemize}
    \item Características sobrepostas
    \item Garantido pela seleção da amostra
\end{itemize}
\end{columns}

\vspace{0.5cm}
\begin{alertblock}{Validação}
Todos os pressupostos testados e validados empiricamente
\end{alertblock}
\end{frame}

% ===== SEÇÃO 4: RESULTADOS PRINCIPAIS =====
\section{Resultados}

\begin{frame}{Efeito Médio do Tratamento (ATT)}
\begin{center}
\huge
\textbf{ATT = \mainatt}\\
\large
\textbf{(Aumento de \mainattpct{} no valor de produção de cana)}\\
\normalsize
\vspace{0.5cm}
Erro Padrão = \mainse\\
IC 95\%: [8,3\%; 88,8\%]\\
p-valor = \mainp
\end{center}

\vspace{0.5cm}

\begin{columns}
\column{0.5\textwidth}
\begin{block}{Interpretação}
Efeito substancial e economicamente significativo, emergindo gradualmente ao longo do tempo
\end{block}

\column{0.5\textwidth}
\begin{block}{Robustez}
\begin{itemize}
    \item DR: 48,5\% (p = 0,018)
    \item IPW: 42,9\% (p = 0,003)
    \item REG: 45,2\% (p = 0,019)
\end{itemize}
\end{block}
\end{columns}
\end{frame}

\begin{frame}{Decomposição: Margens Extensiva e Intensiva}
\begin{columns}
\column{0.5\textwidth}
\textbf{Valor de Produção:}
\begin{itemize}
    \item ATT = \mainatt{} (\mainattpct{})
    \item IC 95\%: [8,3\%; 88,8\%]
    \item p = \mainp
\end{itemize}

\vspace{0.5cm}
\textbf{Área Plantada:}
\begin{itemize}
    \item ATT = \areacanaatt{} (\areacanaattpct{})
    \item IC 95\%: [4,7\%; 48,3\%]
    \item p = \areacanap
\end{itemize}

\column{0.5\textwidth}
\begin{alertblock}{Interpretação}
Efeito no valor (\mainattpct{}) > Efeito na área (\areacanaattpct{})

\vspace{0.3cm}
Diferença de \textbf{~22 pontos percentuais} atribuída a ganhos de \textbf{produtividade} (margem intensiva)
\end{alertblock}

\vspace{0.3cm}
\begin{block}{Implicação}
Estações afetam tanto expansão territorial quanto otimização do manejo
\end{block}
\end{columns}
\end{frame}

\begin{frame}{Event Study: Dinâmica Temporal}
\begin{figure}
\centering
\includegraphics[width=0.85\textwidth]{../../../data/outputs/presentation/event_study_enhanced.png}
\end{figure}

\textbf{Evidências-chave:}
\begin{itemize}
    \item Pré-tratamento: ausência de tendências diferenciadas (validação)
    \item Pós-tratamento: efeitos emergem gradualmente
    \item Processo de difusão e aprendizado na utilização da informação
\end{itemize}
\end{frame}

% ===== SEÇÃO 5: VALIDAÇÃO CAUSAL =====
\section{Testes de Robustez}

\begin{frame}{Teste de Randomização de Monte Carlo}
\begin{columns}
\column{0.5\textwidth}
\textbf{Procedimento:}
\begin{enumerate}
    \item 5.000 simulações independentes
    \item Randomizar tratamento:
    \begin{itemize}
        \item Quais unidades
        \item Quando tratadas
    \end{itemize}
    \item Estimar ATT para cada simulação
    \item Construir distribuição empírica
\end{enumerate}

\vspace{0.3cm}
\textbf{P-valor empírico:}
$$\hat{p} = \frac{1 + \sum \mathbb{1}\{|ATT^s| \geq |ATT^{obs}|\}}{5001}$$

\column{0.5\textwidth}
\begin{figure}
\centering
\includegraphics[width=\textwidth]{../../../data/outputs/placebo_distribution.png}
\end{figure}

\begin{alertblock}{Resultado}
P-valor \placebopvalue\\
Probabilidade de efeito similar por acaso: \placebopvaluepct
\end{alertblock}
\end{columns}
\end{frame}

\begin{frame}{Especificidade do Efeito: Culturas Alternativas}
\begin{columns}
\column{0.5\textwidth}
\textbf{Cana-de-Açúcar (Principal):}
\begin{itemize}
    \item \textbf{Valor:} 48,5\%*** (p = 0,018)
    \item \textbf{Área:} 26,5\%** (p = 0,017)
\end{itemize}

\vspace{0.5cm}
\textbf{Soja (Placebo):}
\begin{itemize}
    \item Valor: 36,8\% (p = 0,150) \textit{NS}
    \item Área: 3,0\% (p = 0,653) \textit{NS}
\end{itemize}

\column{0.5\textwidth}
\textbf{Arroz (Placebo):}
\begin{itemize}
    \item Valor: -13,5\% (p = 0,482) \textit{NS}
    \item Área: -21,2\% (p = 0,137) \textit{NS}
\end{itemize}

\vspace{0.5cm}
\begin{alertblock}{Conclusão}
Efeito \textbf{exclusivo à cana}, confirmando mecanismo via irrigação de salvamento
\end{alertblock}
\end{columns}

\vspace{0.3cm}
\begin{center}
\textit{NS = Não Significativo; ** p<0,05; *** p<0,02}
\end{center}
\end{frame}

\begin{frame}{Robustez: Grupos de Controle e Especificações}
\begin{columns}
\column{0.5\textwidth}
\textbf{Grupo de Controle:}
\begin{itemize}
    \item Not-yet-treated: 48,5\%***
    \item Never-treated: 40,8\%**
    \item Diferença: apenas 2,5 p.p.
\end{itemize}

\vspace{0.3cm}
\textbf{Especificações:}
\begin{itemize}
    \item DR (principal): 48,5\%***
    \item IPW: 42,9\%***
    \item Regressão: 45,2\%**
    \item Sem covariáveis: 34,6\%***
\end{itemize}

\column{0.5\textwidth}
\begin{figure}
\centering
\includegraphics[width=\textwidth]{../../../data/outputs/robustness_plot.png}
\end{figure}

\begin{block}{Estabilidade}
Todas especificações significativas, efeitos entre 35-49\%
\end{block}
\end{columns}
\end{frame}

\begin{frame}{Síntese da Validação Causal}
\textbf{Cinco dimensões de evidência:}

\vspace{0.3cm}
\begin{enumerate}
    \item \textbf{Tendências paralelas:} F = 1,136 (p = 0,322) $\checkmark$

    \item \textbf{Monte Carlo:} 5.000 simulações, p < 0,001 $\checkmark$

    \item \textbf{Especificidade por cultura:} Soja e arroz sem efeitos significativos $\checkmark$

    \item \textbf{Estabilidade entre métodos:} DR, IPW, REG convergem $\checkmark$

    \item \textbf{Robustez a controles:} Not-yet-treated vs never-treated similares $\checkmark$
\end{enumerate}

\vspace{0.5cm}
\begin{center}
\textbf{Interpretação causal fundamentada em evidências convergentes}
\end{center}
\end{frame}

% ===== SEÇÃO 6: IMPLICAÇÕES =====
\section{Implicações Econômicas e de Política}

\begin{frame}{Magnitude Econômica}
\begin{columns}
\column{0.6\textwidth}
\textbf{Contextualização:}
\begin{itemize}
    \item Efeito de 48,5\% = impacto substancial
    \item Efeito emerge gradualmente (event study)
    \item 225 microrregiões produtoras analisadas
    \item 67 produtoras ainda sem estação (29,7\%)
    \item Potencial significativo de expansão
\end{itemize}

\vspace{0.3cm}
\textbf{Investimento Anunciado (Set/2025):}
\begin{itemize}
    \item MAPA: R\$ 49 milhões
    \item 220 novas estações
    \item Custo unitário: ~R\$ 223 mil
\end{itemize}

\column{0.4\textwidth}
\begin{block}{Evidência}
Benefícios superam amplamente custos de implementação
\end{block}

\vspace{0.3cm}
\begin{alertblock}{Implicação}
Expansão justificada como estratégia de desenvolvimento e adaptação climática
\end{alertblock}
\end{columns}
\end{frame}

\begin{frame}{Recomendações de Política}
\textbf{1. Priorização Estratégica:}
\begin{itemize}
    \item Focar expansão em regiões produtoras de culturas com gestão hídrica complexa
    \item Informação meteorológica não é insumo neutro: valor depende de estrutura produtiva local
\end{itemize}

\vspace{0.3cm}
\textbf{2. Política Integrada:}
\begin{itemize}
    \item Rede meteorológica + infraestrutura de irrigação
    \item Cana se beneficia imediatamente (salvamento já estabelecido)
    \item Outras culturas requerem desenvolvimento de sistemas adequados
\end{itemize}

\vspace{0.3cm}
\textbf{3. Potencial de Expansão:}
\begin{itemize}
    \item 164 microrregiões totais sem estação (29,4\%)
    \item 67 produtoras de cana sem estação (29,7\%)
    \item Margem considerável para geração adicional de valor
\end{itemize}
\end{frame}

% ===== SEÇÃO 7: CONCLUSÕES =====
\section{Conclusões}

\begin{frame}{Contribuições do Trabalho}
\begin{enumerate}
    \item \textbf{Evidência Causal Pioneira}
    \begin{itemize}
        \item Primeira quantificação rigorosa do impacto de estações meteorológicas
        \item ATT = 48,5\% robusto a múltiplas especificações
    \end{itemize}

    \vspace{0.3cm}
    \item \textbf{Avanço Metodológico}
    \begin{itemize}
        \item Aplicação de DiD com tratamento escalonado (Callaway \& Sant'Anna)
        \item Demonstração prática para avaliação de políticas públicas
    \end{itemize}

    \vspace{0.3cm}
    \item \textbf{Caracterização de Mecanismos}
    \begin{itemize}
        \item Identificação do canal via irrigação de salvamento
        \item Especificidade à cana devido a características de manejo hídrico
        \item Efeitos em margens extensiva (26,5\%) e intensiva (~22 p.p.)
    \end{itemize}

    \vspace{0.3cm}
    \item \textbf{Subsídios para Políticas}
    \begin{itemize}
        \item Retorno econômico comprovado supera custos
        \item Orientação para priorização estratégica de investimentos
    \end{itemize}
\end{enumerate}
\end{frame}

\begin{frame}{Mensagem Final}
\begin{center}
\Large
\textbf{Informação meteorológica como investimento estratégico}

\vspace{0.5cm}
\normalsize
\begin{itemize}
    \item Impacto econômico substantivo e robusto (48,5\%)
    \item Efeito específico a culturas com gestão hídrica complexa
    \item Justificativa empírica para expansão da rede
    \item Importância de políticas integradas (informação + infraestrutura)
\end{itemize}

\vspace{0.5cm}
\Large
\textbf{Infraestrutura de informação climática:}\\
\textbf{ativo produtivo com retorno comprovado}
\end{center}
\end{frame}

% Slide final
\begin{frame}{}
\centering
\Huge
\textbf{Obrigado!}\\
\vspace{1cm}
\Large
Daniel Cavalli\\
\normalsize
\texttt{daniel.cavalli@ie.ufrj.br}\\
\vspace{0.5cm}
Código e dados:\\
\url{github.com/danielcavalli/tcc-ie-ufrj-2024}
\end{frame}

% ===== SLIDES DE BACKUP =====
\appendix

\begin{frame}{Backup: Especificação do Modelo}
\textbf{Modelo principal:}
\begin{itemize}
    \item Outcome: $Y_{it} = \ln(1 + \text{ValorProducaoCana}_{it})$
    \item Tratamento: $W_{it} = \mathbb{1}\{t \geq G_i\}$
    \item Estimador: Doubly Robust
\end{itemize}

\vspace{0.3cm}
\textbf{Covariáveis:}
\begin{itemize}
    \item Log da área total da microrregião
    \item Log da população
    \item Log do PIB per capita
    \item Log da densidade estadual de estações
    \item Variáveis climáticas (precipitação total, média, máxima)
\end{itemize}

\vspace{0.3cm}
\textbf{Clustering:} Microrregião (225 clusters na amostra de cana)
\end{frame}

\begin{frame}{Backup: Cobertura por Região}
\textbf{Das 558 microrregiões brasileiras:}

\vspace{0.3cm}
\begin{table}[h]
\centering
\small
\begin{tabular}{lcc}
\toprule
Região & Total Microrregiões & Cobertura (\%) \\
\midrule
Norte & 64 & Variável \\
Nordeste & 188 & Variável \\
Centro-Oeste & 37 & Variável \\
Sudeste & 160 & Variável \\
Sul & 94 & Variável \\
\midrule
\textbf{Brasil} & \textbf{558} & \textbf{70,6\%} \\
\bottomrule
\end{tabular}
\end{table}

\vspace{0.3cm}
\textbf{Amostra de análise (cana):} 225 microrregiões (41,3\% do total)

\textbf{Cobertura geral:} 394 microrregiões com estações (70,6\%), 164 sem (29,4\%)
\end{frame}

\begin{frame}{Backup: Análise de Poder Estatístico}
\begin{figure}
\centering
\includegraphics[width=0.7\textwidth]{../../../data/outputs/additional_figures/power_analysis_simulation.png}
\end{figure}

\begin{itemize}
    \item Para efeito de \mainattpct{}: poder de 92,1\% ($\alpha = 0,05$)
    \item Design adequado para detectar efeitos economicamente relevantes
    \item Amostra de 225 microrregiões × 19 anos suficiente
\end{itemize}
\end{frame}

\begin{frame}{Backup: Limitações e Extensões}
\textbf{Limitações:}
\begin{itemize}
    \item Desbalanceamento de covariáveis (mitigado por DR)
    \item Concentração de peso em early adopters
    \item Heterogeneidade não observada
    \item Transbordamentos espaciais não completamente capturados
\end{itemize}

\vspace{0.3cm}
\textbf{Extensões futuras:}
\begin{itemize}
    \item Modelagem espacial explícita
    \item Dados de frequência temporal mais alta (mensal/trimestral)
    \item Heterogeneidade por intensidade de irrigação
    \item Mecanismos de transmissão (variedades, timing, replantio)
    \item Incorporação de covariáveis de aptidão edafoclimática
\end{itemize}
\end{frame}

\end{document}
