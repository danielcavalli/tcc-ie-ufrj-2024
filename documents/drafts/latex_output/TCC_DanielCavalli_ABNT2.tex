% abtex2-modelo-trabalho-academico.tex, v-1.9.7 laurocesar
% Modelo de Trabalho Acadêmico (tese de doutorado, dissertação de
% mestrado e trabalhos monográficos em geral) em conformidade com
% ABNT NBR 14724:2011
% ------------------------------------------------------------------------

\documentclass[
	% -- opções da classe memoir --
	12pt,				% tamanho da fonte
	%openright,			% seções começam em pág ímpar (insere página vazia caso preciso)
	oneside,			% para impressão em verso e anverso. Oposto a oneside
	a4paper,			% tamanho do papel.
	% -- opções da classe abntex2 --
	%chapter=TITLE,		% títulos de seções principais convertidos em letras maiúsculas
	%section=TITLE,		% títulos de seções convertidos em letras maiúsculas
	%subsection=TITLE,	% títulos de subseções convertidos em letras maiúsculas
	%subsubsection=TITLE,% títulos de subsubseções convertidos em letras maiúsculas
	% -- opções do pacote babel --
	english,			% idioma adicional para hifenização
	french,				% idioma adicional para hifenização
	spanish,			% idioma adicional para hifenização
	brazil				% o último idioma é o principal do documento
	]{abntex2}

% ---
% PACOTES
% ---

% Pacotes básicos
\usepackage{lmodern}			% Usa a fonte Latin Modern
\usepackage[T1]{fontenc}		% Selecao de codigos de fonte.
\usepackage[utf8]{inputenc}		% Codificacao do documento (conversão automática dos acentos)
\usepackage{indentfirst}		% Indenta o primeiro parágrafo de cada seção.
\usepackage{textcomp}			% Símbolos adicionais
\usepackage{color}				% Controle das cores
\usepackage{graphicx}			% Inclusão de gráficos
\usepackage{microtype} 			% para melhorias de justificação
\usepackage{amsmath}
\usepackage{amssymb}
\usepackage{mathtools}
\usepackage{longtable}
\usepackage{booktabs}
\usepackage{multirow}           % Para células multi-linha em tabelas
\usepackage{csquotes}           % Para aspas corretas
\usepackage{float}              % Para forçar posicionamento de figuras com [H]
\usepackage{placeins}           % Para controlar onde floats podem ir com \FloatBarrier

% Configuração para melhorar posicionamento de figuras
\renewcommand{\topfraction}{0.85}        % máximo da página que pode ser ocupado por floats no topo
\renewcommand{\bottomfraction}{0.7}     % máximo da página que pode ser ocupado por floats no fundo
\renewcommand{\textfraction}{0.15}      % mínimo da página que deve ser texto
\renewcommand{\floatpagefraction}{0.66} % mínimo da página de floats que deve estar ocupada
\setcounter{topnumber}{3}
\setcounter{bottomnumber}{3}
\setcounter{totalnumber}{4}

% Pacotes de citações
\usepackage[brazilian,hyperpageref]{backref}	 % Paginas com as citações na bibl
\usepackage[alf,abnt-url=yes,abnt-emphasize=bf]{abntex2cite}	% Citações padrão ABNT

% Configuração para quebra de URLs longas
\PassOptionsToPackage{hyphens}{url}

% ---
% CONFIGURAÇÕES DE PACOTES
% ---

% Configurações do pacote backref
\renewcommand{\backrefpagesname}{Citado na(s) página(s):~}
% Texto padrão antes do número das páginas
\renewcommand{\backref}{}
% Define os textos da citação
\renewcommand*{\backrefalt}[4]{
	\ifcase #1 %
		Nenhuma citação no texto.%
	\or
		Citado na página #2.%
	\else
		Citado #1 vezes nas páginas #2.%
	\fi}%
% ---

% ---
% Informações de dados para CAPA e FOLHA DE ROSTO
% ---
\titulo{IMPACTO DE ESTAÇÕES METEOROLÓGICAS NA CULTURA DE CANA-DE-AÇÚCAR: UMA APLICAÇÃO DE DIFERENÇAS EM DIFERENÇAS COM TRATAMENTO ESCALONADO}
\autor{Daniel Cavalli}
\local{Rio de Janeiro}
\data{Dezembro 2025}
\orientador{Prof. Romero Rocha}
\instituicao{%
  UNIVERSIDADE FEDERAL DO RIO DE JANEIRO
  \par
  INSTITUTO DE ECONOMIA
  \par
  CURSO DE GRADUAÇÃO EM CIÊNCIAS ECONÔMICAS}
\tipotrabalho{Monografia}
% Preâmbulo: tipo do trabalho, objetivo, instituição e área de concentração
\preambulo{Monografia apresentada ao Instituto de Economia da Universidade Federal do Rio de Janeiro como parte dos requisitos necessários à obtenção do título de Bacharel em Ciências Econômicas.}
% ---

% ---
% Inclusão de valores gerados automaticamente pelo R
% ---
% Arquivo gerado automaticamente por generate_latex_values.r
% Última atualização: 2025-11-18

% Valores do teste placebo aleatório
\newcommand{\placebotruatt}{0.485}
\newcommand{\placebopvalue}{< 0,001}
\newcommand{\placebolower}{-0.269}
\newcommand{\placeboupper}{0.286}
\newcommand{\placebonsims}{5000}
\newcommand{\placebomean}{0.002}
\newcommand{\placebosd}{0.140}

% Valores de precisão do p-valor (Monte Carlo)
\newcommand{\placebopvaluese}{0.000}
\newcommand{\placebopvaluelower}{0.000}
\newcommand{\placebopvalueupper}{0.001}
\newcommand{\placebonextremes}{0}

% Valores formatados para texto
\newcommand{\placebotruattpct}{48.5\%}
\newcommand{\placebopvaluepct}{0.02\%}

% Valores do modelo principal (resultado detalhado)
\newcommand{\mainatt}{0.485}
\newcommand{\mainse}{0.205}
\newcommand{\mainz}{2.362}
\newcommand{\mainp}{0.018}
\newcommand{\maincilower}{0.083}
\newcommand{\mainciupper}{0.888}
\newcommand{\mainattpct}{48.5\%}

% Valores de área cana (outcome secundário)
\newcommand{\areacanaatt}{0.265}
\newcommand{\areacanaattpct}{26.5\%}
\newcommand{\areacanase}{0.111}
\newcommand{\areacanap}{0.017}
\newcommand{\areacanalower}{0.047}
\newcommand{\areacanaupper}{0.483}

% Valores de área soja
\newcommand{\areasojaat}{0.030}
\newcommand{\areasojase}{0.068}
\newcommand{\areasojap}{0.653}
\newcommand{\areasojlower}{-0.102}
\newcommand{\areasojaupper}{0.163}

% Valores de área arroz
\newcommand{\areaarrozatt}{-0.212}
\newcommand{\areaarrozse}{0.143}
\newcommand{\areaarrozp}{0.137}
\newcommand{\areaarrozlower}{-0.492}
\newcommand{\areaarrozupper}{0.068}

% Valores de valor de produção soja
\newcommand{\valorsojaat}{0.368}
\newcommand{\valorsojase}{0.256}
\newcommand{\valorsojap}{0.150}
\newcommand{\valorsojlower}{-0.133}
\newcommand{\valorsojaupper}{0.869}

% Valores de valor de produção arroz
\newcommand{\valorarrozatt}{-0.135}
\newcommand{\valorarrozse}{0.192}
\newcommand{\valorarrozp}{0.482}
\newcommand{\valorarrozlower}{-0.510}
\newcommand{\valorarrozupper}{0.241}

% Grupo de Controle: Never-treated
\newcommand{\nevertreatedatt}{0.408}
\newcommand{\nevertreatedse}{0.200}
\newcommand{\nevertreatedlower}{0.016}
\newcommand{\nevertreatedupper}{0.801}

% Especificação: Sem Covariáveis
\newcommand{\nocovatt}{0.346}
\newcommand{\nocovse}{0.168}
\newcommand{\nocovlower}{0.016}
\newcommand{\nocover}{0.675}

% Especificação: IPW
\newcommand{\ipwatt}{0.429}
\newcommand{\ipwse}{0.199}
\newcommand{\ipwlower}{0.040}
\newcommand{\ipwupper}{0.819}

% Especificação: Regressão de Resultado
\newcommand{\regatt}{0.452}
\newcommand{\regse}{0.193}
\newcommand{\reglower}{0.073}
\newcommand{\regupper}{0.832}

% Valores do Valor de Produção Cana em Microrregiões Produtoras
\newcommand{\valorfiltatt}{0.477}
\newcommand{\valorfiltse}{0.211}
\newcommand{\valorfiltcilower}{0.064}
\newcommand{\valorfiltciupper}{0.890}
\newcommand{\valorfiltatppct}{47.7\%}
\newcommand{\valorfiltpvalue}{0.024}
\newcommand{\valorfiltnmicro}{225}
\newcommand{\valorfiltretention}{5172.4\%}

% Valores da análise de sensibilidade temporal
% Completo (2003-2023)
\newcommand{\sensfullatt}{0.110}
\newcommand{\sensfullse}{0.028}
\newcommand{\sensfulllower}{0.055}
\newcommand{\sensfullupper}{0.164}
\newcommand{\sensfulln}{7334}
% Excluindo Início (2006-2023)
\newcommand{\sensnostartatt}{0.109}
\newcommand{\sensnostartse}{0.031}
\newcommand{\sensnostartlower}{0.048}
\newcommand{\sensnostartupper}{0.171}
\newcommand{\sensnostartn}{6176}
% Excluindo Final (2003-2019)
\newcommand{\senssensaltcatt}{0.104}
\newcommand{\senssensaltcse}{0.024}
\newcommand{\senssensaltclower}{0.056}
\newcommand{\senssensaltcupper}{0.151}
\newcommand{\senssensaltcn}{6562}
% Excluindo COVID (2003-2019)
\newcommand{\sensnocovidatt}{0.104}
\newcommand{\sensnocovidse}{0.025}
\newcommand{\sensnocovidlower}{0.055}
\newcommand{\sensnocovidupper}{0.153}
\newcommand{\sensnocovidn}{6562}
% Pré-COVID (2003-2019)
\newcommand{\senssensaltfatt}{0.104}
\newcommand{\senssensaltfse}{0.025}
\newcommand{\senssensaltflower}{0.055}
\newcommand{\senssensaltfupper}{0.153}
\newcommand{\senssensaltfn}{6562}


% ---
% Configurações de aparência do PDF final

% alterando o aspecto da cor azul
\definecolor{blue}{RGB}{41,5,195}

% informações do PDF
\makeatletter
\hypersetup{
     	breaklinks=true,			% permite quebra de links em múltiplas linhas
     	%pagebackref=true,
		pdftitle={\@title},
		pdfauthor={\@author},
    	pdfsubject={\imprimirpreambulo},
	    pdfcreator={LaTeX with abnTeX2},
		pdfkeywords={estações meteorológicas}{valor de produção}{cana-de-açúcar}{MapBiomas}{irrigação}{diferenças em diferenças escalonada}{Callaway e Sant'Anna}{informação climática},
		colorlinks=true,       		% false: boxed links; true: colored links
    	linkcolor=blue,          	% color of internal links
    	citecolor=blue,        		% color of links to bibliography
    	filecolor=magenta,      	% color of file links
		urlcolor=blue,
		bookmarksdepth=4
}
\makeatother
% ---

% ---
% Espaçamentos entre linhas e parágrafos
% ---

% O tamanho do parágrafo é dado por:
\setlength{\parindent}{1.3cm}

% Controle do espaçamento entre um parágrafo e outro:
\setlength{\parskip}{0.2cm}  % tente também \onelineskip

% ---
% compila o indice
% ---
\makeindex
% ---

% ---
% Valores gerados automaticamente já incluídos em data/outputs/latex_values.tex
% Removido auto_values para evitar conflitos de macros
% ---

% ---
% Configuração para seções com quebra de página
% ---
\makeatletter
% Mantém quebra de página entre as seções principais
% \renewcommand{\clearforchapter}{}  % Comentado para manter quebras de página
\makeatother

% ---
% Configuração de espaçamentos das seções conforme ABNT
% ---
% ABNT: 2 espaços de 1,5 entrelinhas antes do título
\setlength{\beforechapskip}{3\baselineskip}
% ABNT: 1 espaço de 1,5 entrelinhas depois do título
\setlength{\afterchapskip}{1.5\baselineskip}

% Ajusta também o espaçamento das seções
\setlength{\beforesecskip}{1.5\baselineskip}
\setlength{\aftersecskip}{1.5\baselineskip}
% ---

% ----
% Início do documento
% ----
\begin{document}

% Idioma principal do documento
\selectlanguage{brazil}

% Remove espaços extras entre frases
\frenchspacing

% ----------------------------------------------------------
% ELEMENTOS PRÉ-TEXTUAIS
% ----------------------------------------------------------
% \pretextual

% ---
% Capa
% ---
\imprimircapa
% ---

% ---
% Folha de rosto
% (o * indica que haverá a ficha bibliográfica)
% ---
\imprimirfolhaderosto*
% ---

% ---
% Inserir folha de aprovação
% ---

% Folha de aprovação - elemento obrigatório da NBR 14724/2011 (seção 4.2.1.3)
% Após a defesa, incluir a versão assinada pela banca:
% \includepdf{folhadeaprovacao_final.pdf}
\begin{folhadeaprovacao}

  \begin{center}
    {\ABNTEXchapterfont\large\imprimirautor}

    \vspace*{\fill}\vspace*{\fill}
    \begin{center}
      \ABNTEXchapterfont\bfseries\Large\imprimirtitulo
    \end{center}
    \vspace*{\fill}

    \hspace{.45\textwidth}
    \begin{minipage}{.5\textwidth}
        \imprimirpreambulo
    \end{minipage}%
    \vspace*{\fill}
   \end{center}

   Trabalho aprovado. \imprimirlocal, \today:

   \assinatura{\textbf{\imprimirorientador} \\ Orientador}
   \assinatura{\textbf{Professor} \\ Convidado 1}
   \assinatura{\textbf{Professor} \\ Convidado 2}

   \begin{center}
    \vspace*{0.5cm}
    {\large\imprimirlocal}
    \par
    {\large\imprimirdata}
    \vspace*{1cm}
  \end{center}

\end{folhadeaprovacao}
% ---

% ---
% Agradecimentos
% ---
\begin{agradecimentos}

Agradeço à minha família, que sempre me apoiou durante toda a graduação sem nunca me pressionar além do necessário. Vocês foram fundamentais para que eu pudesse acumular o conhecimento necessário para montar este trabalho.

À minha namorada, Juliana, obrigado por me incentivar a terminar a faculdade e o TCC quando eu já não tinha mais vontade ou razão para tal. Seu apoio fez toda a diferença em me convencer da importância disso.

À minha cachorra, Moana, companheira fiel de tantas horas de escrita, modelagem e análise de dados. Sua presença tornou as longas madrugadas de trabalho muito mais suportáveis.

À Cláudia.

\end{agradecimentos}
% ---

% ---
% Epígrafe
% ---
\begin{epigrafe}
    \vspace*{\fill}
	\begin{flushright}
		\textit{``In mathematics you don't understand things.\\
		You just get used to them.''}\\
		(John von Neumann)
	\end{flushright}
\end{epigrafe}
% ---

% ---
% RESUMOS
% ---

% resumo em português
\setlength{\absparsep}{18pt} % ajusta o espaçamento dos parágrafos do resumo
\begin{resumo}
Este estudo examina o impacto da instalação de estações meteorológicas automáticas sobre a produção de cana-de-açúcar no Brasil. Utilizando dados de 225 microrregiões produtoras entre 2003 e 2021, exploramos a variação temporal e geográfica na instalação de estações para identificar efeitos causais através do método de Diferenças em Diferenças com adoção escalonada (Callaway e Sant'Anna, 2021). Os resultados indicam um aumento de \mainattpct{} no valor de produção da cana-de-açúcar nas microrregiões que receberam estações meteorológicas. A análise revela que esse efeito ocorre tanto através da expansão da área plantada (\areacanaattpct{}) quanto por ganhos de produtividade. Testes de robustez confirmam a validade dos resultados, incluindo ausência de tendências pré-tratamento e especificidade do efeito para a cultura da cana. Os achados sugerem que o acesso a informações climáticas precisas pode contribuir para melhorias no planejamento agrícola, particularmente em culturas que demandam manejo intensivo como a cana-de-açúcar. O estudo demonstra o papel potencial da infraestrutura informacional no desenvolvimento do setor agrícola brasileiro.

 Palavras-chave: estações meteorológicas. valor de produção. cana-de-açúcar. MapBiomas. irrigação. diferenças em diferenças escalonada. Callaway e Sant'Anna. informação climática. margem extensiva e intensiva.
\end{resumo}

% resumo em inglês
\begin{resumo}[Abstract]
 \begin{otherlanguage*}{english}
This study examines the impact of automatic weather station installation on sugarcane production in Brazil. Using panel data from 225 sugarcane-producing microregions between 2003 and 2021, we exploit temporal and geographic variation in station installation to identify causal effects through a staggered Differences-in-Differences approach (Callaway and Sant'Anna, 2021). Results indicate a \mainattpct{} increase in sugarcane production value in microregions that received weather stations. The analysis reveals that this effect operates through both planted area expansion (\areacanaattpct{}) and productivity gains. Robustness tests confirm the validity of results, including absence of pre-treatment trends and effect specificity to sugarcane cultivation. The findings suggest that access to precise climate information may contribute to improved agricultural planning, particularly for management-intensive crops like sugarcane. The study demonstrates the potential role of informational infrastructure in agricultural sector development in Brazil.

   Keywords: weather stations. production value. sugarcane. MapBiomas. irrigation. staggered differences-in-differences. Callaway and Sant'Anna. climate information. extensive and intensive margins.
 \end{otherlanguage*}
\end{resumo}

% ---
% inserir lista de ilustrações
% ---
\pdfbookmark[0]{\listfigurename}{lof}
\listoffigures*
\cleardoublepage
% ---

% ---
% inserir lista de tabelas
% ---
\pdfbookmark[0]{\listtablename}{lot}
\listoftables*
\cleardoublepage
% ---

% ---
% inserir o sumario
% ---
\pdfbookmark[0]{\contentsname}{toc}
\tableofcontents*
\cleardoublepage
% ---

% ----------------------------------------------------------
% ELEMENTOS TEXTUAIS
% ----------------------------------------------------------
\textual

% ----------------------------------------------------------
% INTRODUÇÃO INTEGRADA - VERSÃO UNIFICADA
% ----------------------------------------------------------
\chapter{Introdução}
\label{cap:introducao}

% Abertura forte conectando o problema econômico central
A agricultura brasileira enfrenta o desafio permanente de aumentar a produtividade em um contexto de crescente variabilidade climática. Com a produção agrícola global sendo amplamente determinada por oscilações meteorológicas durante o ciclo produtivo \cite{monteiro2009}, e as mudanças climáticas já impactando significativamente a produtividade mundial \cite{ortiz2020}, a questão central não é mais se o clima afeta a agricultura mas como mitigar seus efeitos adversos e aproveitar janelas de oportunidade. Neste contexto, a informação meteorológica precisa emerge como insumo produtivo crítico, potencialmente capaz de transformar incerteza em risco gerenciável.

% Desenvolvimento do argumento com suporte da literatura
A literatura documenta extensivamente os canais através dos quais informações meteorológicas podem aumentar a produtividade agrícola. \citeonline{mavi2004} identificam três dimensões principais: planejamento estratégico (escolha de culturas e épocas de plantio), decisões táticas (timing de irrigação, aplicação de defensivos) e construção de resiliência sistêmica. \citeonline{weiss2000} demonstram que estações meteorológicas locais permitem ajustes finos nas práticas agrícolas, enquanto \citeonline{rijks2000} quantificam os ganhos econômicos potenciais de serviços meteorológicos bem estruturados. No Brasil, sistemas como AGRITEMPO e SISDAGRO já operacionalizam essas informações mas sua efetividade depende crucialmente da densidade e qualidade da rede de estações meteorológicas subjacente.

% Mecanismo causal explícito
O mecanismo causal pode ser descrito de forma sequencial: a instalação de uma estação meteorológica local gera dados climáticos precisos e em tempo real, que são processados e disponibilizados aos produtores através de boletins, alertas e sistemas de informação. Com acesso a previsões mais precisas sobre temperatura, precipitação e eventos extremos, os agricultores podem otimizar decisões críticas, desde o momento ideal de plantio e colheita até a aplicação precisa de insumos e gestão eficiente de irrigação.

Além do uso direto pelos produtores, esses dados alimentam modelos computacionais avançados de simulação agrícola. O modelo DSSAT CSM-CANEGRO, amplamente utilizado no setor sucroalcooleiro, integra informações meteorológicas locais para simular o crescimento e desenvolvimento da cana-de-açúcar sob diferentes cenários de manejo. Com dados meteorológicos precisos, o modelo permite testar virtualmente diferentes estratégias de irrigação, particularmente para a irrigação de salvamento, prática essencial para estender a longevidade dos canaviais e garantir a sobrevivência da soqueira após cada corte. A otimização do timing de aplicação pode resultar em incrementos de produtividade significativos, especialmente em solos mais vulneráveis \cite{vianna2016}.

Essas decisões melhor informadas, principalmente através do melhor timing da irrigação de salvamento e outras práticas de manejo, traduzem-se em redução de perdas, aumento de produtividade e consequentemente expansão da área cultivada e crescimento da atividade agrícola local.

% Identificação da lacuna acadêmica
Embora o mecanismo teórico seja bem estabelecido e os potenciais benefícios amplamente reconhecidos, a literatura carece de evidências causais robustas sobre o impacto econômico da infraestrutura de monitoramento climático. Estudos existentes são predominantemente descritivos ou correlacionais, deixando questões fundamentais sem resposta: Qual é o retorno econômico real da instalação de estações meteorológicas? Como esse impacto evolui ao longo do tempo? Os benefícios justificam os custos de expansão da rede?

% Oportunidade de identificação
A expansão da rede de estações meteorológicas no Brasil entre 2003 e 2021 oferece uma oportunidade única para responder essas questões. Durante esse período, diferentes microrregiões receberam estações em momentos distintos, criando variação temporal e geográfica que permite identificação causal. Das 558 microrregiões brasileiras, 394 receberam pelo menos uma estação meteorológica automática durante o período analisado, enquanto 164 permaneceram sem cobertura, fornecendo grupos de tratamento e controle bem definidos.

% Desafio metodológico
Entretanto, esse padrão de adoção escalonada (staggered adoption) apresenta desafios metodológicos importantes. Como demonstrado por \citeonline{goodman2021} e \citeonline{sun2021}, o estimador tradicional de Diferenças em Diferenças com Efeitos Fixos Bidimensionais (TWFE) produz resultados enviesados quando o tratamento é adotado em diferentes períodos. O problema surge porque unidades já tratadas são inadvertidamente usadas como controles para unidades tratadas posteriormente e efeitos exógenos ao tratamento ao longo do tempo são incorretamente agregados.

% Solução metodológica
Este trabalho supera essas limitações utilizando o arcabouço de \citeonline{callaway2021} para Diferenças em Diferenças com múltiplos períodos. O método agrupa unidades por coorte de tratamento e compara cada grupo tratado apenas com unidades ainda não tratadas, evitando os vieses do TWFE tradicional e permitindo a estimação de efeitos heterogêneos por grupo e tempo.

% Estratégia empírica
Para capturar o impacto econômico completo das estações meteorológicas, focamos no valor de produção de cana-de-açúcar, cultura com extensa adoção de infraestrutura de irrigação e especialmente sensível à disponibilidade de informação meteorológica precisa. Utilizamos o logaritmo dessa variável para interpretar os resultados em termos percentuais, complementando a análise com a área plantada para distinguir entre ganhos de produtividade (margem intensiva) e expansão territorial (margem extensiva). A especificidade desse efeito é validada através de testes placebo com soja e arroz, culturas menos dependentes do timing preciso de irrigação. Restringimos a análise a microrregiões produtoras de cada cultura, garantindo que comparamos apenas unidades relevantes e evitando contaminação por regiões estruturalmente diferentes.

Nossa análise revela que a instalação de estações meteorológicas gera um aumento de \mainattpct{} no valor de produção de cana-de-açúcar nas microrregiões tratadas. Este efeito econômico substancial opera através de dois canais: expansão de \areacanaattpct{} na área plantada (margem extensiva) e ganhos de produtividade (margem intensiva). Os resultados mostram-se robustos a múltiplas especificações alternativas e análises de sensibilidade. A inferência por randomização com 5.000 simulações confirma que o efeito observado é estatisticamente distinguível de atribuições aleatórias do tratamento (p<0,001), enquanto testes placebo com soja e arroz demonstram especificidade do impacto para a cana-de-açúcar.

% Contribuições específicas
As contribuições deste trabalho são quádruplas:

\begin{enumerate}
\item Evidência causal pioneira: primeira quantificação rigorosa do impacto econômico de estações meteorológicas na agricultura brasileira, preenchendo uma lacuna crítica para políticas públicas baseadas em evidências.

\item Avanço metodológico: demonstração da aplicabilidade e importância dos novos métodos de DiD escalonado em contextos agrícolas, contribuindo para a literatura metodológica aplicada.

\item Caracterização da dinâmica: documentação dos efeitos na margem intensiva (produtividade) e extensiva (área plantada), com implicações para avaliação de investimentos em infraestrutura meteorológica.

\item Subsídios para expansão da rede: evidências de que o retorno econômico supera amplamente os custos, justificando a expansão da infraestrutura meteorológica como estratégia de adaptação climática e aumento de produtividade.
\end{enumerate}

% Diálogo com literatura recente
Nossos resultados dialogam com desenvolvimentos recentes na literatura. \citeonline{burke2021} argumentam que avanços em tecnologias de informação representam uma das principais fronteiras para aumentar a produtividade agrícola no século XXI. \citeonline{monteiro2017} demonstram a importância de modelos agrometeorológicos para identificação de gaps de produtividade na agricultura brasileira. \citeonline{crost2018} e \citeonline{gatti2021} demonstram, usando métodos similares aos nossos, como infraestrutura pode mitigar impactos climáticos. Este trabalho contribui para essa literatura emergente ao fornecer a primeira evidência causal direta sobre estações meteorológicas.

% Estrutura do trabalho
O restante deste trabalho está organizado da seguinte forma. A Seção 2 apresenta a metodologia completa, incluindo o arcabouço de Diferenças em Diferenças com múltiplos períodos de \citeonline{callaway2021}, a estratégia empírica, definição do tratamento, variáveis e especificação do modelo. A Seção 3 apresenta os resultados, começando pela implementação computacional e seguindo com os efeitos estimados, análises de robustez e testes de validação. A Seção 4 conclui com implicações para políticas públicas e direções para pesquisa futura.

% ----------------------------------------------------------
% Metodologia
% ----------------------------------------------------------
\chapter{Metodologia}

Para este trabalho, utilizaremos como principal referência o artigo de \citeonline{callaway2021}, que apresenta uma extensão do modelo de Diferenças em Diferenças (DiD) para cenários com múltiplos períodos e momentos distintos de adoção do tratamento.

\section{Introdução ao Modelo}

No DiD clássico, assume-se um grupo tratado que recebe a intervenção em um momento específico e um grupo controle que nunca é tratado. Sob essa configuração, a diferença no tempo entre pré e pós-tratamento e a diferença entre grupos tratado e controle fornecem a estimativa do efeito causal. Entretanto, para o caso analisado neste trabalho há múltiplos períodos e vários grupos recebendo o tratamento em momentos distintos ao longo dos 22 anos do período de análise. A abordagem de DiD tradicional, nesse caso, pode gerar estimativas enviesadas devido à heterogeneidade do tratamento ao longo do tempo, resultando em interpretação ambígua.

O modelo de \citeonline{callaway2021} surge como uma forma de permitir que esses cenários de tratamento escalonado, frequentemente mais comuns no mundo real do que experimentos naturais, possam ser avaliados adequadamente. Por permitir a identificação de efeitos médios do tratamento específicos para cada grupo e período, acomoda a heterogeneidade do momento de adoção e suas dinâmicas, além de fornecer uma interpretação mais clara dos parâmetros causais.

\section{Fundamentos do modelo}

O modelo proposto pode ser entendido em três etapas conceituais:

\begin{enumerate}
\item \textbf{Identificação de parâmetros causais desagregados:} Primeiro, são obtidas estimativas do efeito causal para cada combinação de grupo tratado e período após a adoção (denotados por ATT(g,t)), focando em captar o efeito específico para um determinado conjunto de unidades tratadas em um dado momento do tempo.

\item \textbf{Agregação desses parâmetros:} Em seguida, esses parâmetros individuais, definidos para grupos e períodos específicos, podem ser combinados para produzir medidas resumidas de efeitos, como efeitos médios globais, ao longo do tempo, por coorte de tratamento ou segundo o tempo decorrido desde a intervenção.

\item \textbf{Estimação e inferência:} Por fim, procedimentos estatísticos são empregados para estimar esses parâmetros, bem como inferir sobre sua significância estatística.
\end{enumerate}

\subsection{Group-Time Average Treatment Effects ATT(g,t)}

O parâmetro fundamental dessa abordagem é o ATT(g,t), que representa o Efeito Médio do Tratamento para o grupo g no período t. Ao contrário do DiD tradicional, onde há um único efeito estimado, aqui obtemos uma coleção de efeitos, cada um refletindo o impacto do tratamento em um grupo que começou a ser tratado em um determinado momento e está sendo avaliado em um período específico após o início do tratamento.

Com isso é possível capturar heterogeneidades relacionadas:
\begin{itemize}
\item Ao grupo (unidades diferentes podem ter características e contextos distintos);
\item Ao momento de início do tratamento (tratamentos iniciados em diferentes épocas podem ter efeitos variados devido a condições econômicas, políticas ou sociais);
\item Ao tempo decorrido desde o tratamento (efeitos imediatos versus efeitos de longo prazo podem diferir).
\end{itemize}

\subsection{Identificação}

O artigo de \citeonline{callaway2021} apresenta uma série de pressupostos para identificação dos parâmetros causais. Boa parte delas não difere muito dos pressupostos do DiD tradicional. Abaixo destaco algumas importantes mudanças:

\begin{enumerate}
\item \textbf{Tendências Paralelas Condicionais:} A ideia central do DiD é que, na ausência de tratamento, as unidades tratadas seguiriam a mesma tendência de evolução dos resultados das unidades não tratadas. Existem diferenças conceituais entre o DiD tradicional e o DiD Staggered:
   \begin{itemize}
   \item \textbf{Pressuposto 4 - ``never-treated'':} Aqui, o grupo de comparação é formado por unidades que nunca recebem tratamento ao longo de todo o período observado. Pressupõe-se que, condicionalmente a covariáveis observáveis, esses ``never-treated'' representam a contrafactual apropriada para o que teria acontecido com os grupos tratados caso não tivessem sido tratados.
   \item \textbf{Pressuposto 5 - ``not-yet-treated'':} Nesse caso, o grupo de controle para um determinado período e grupo tratado é formado por unidades que ainda não foram tratadas até aquele momento mas que virão a ser tratadas no futuro. Essa abordagem aproveita a natureza escalonada do tratamento para criar um grupo de comparação internamente consistente.
   \end{itemize}

\item \textbf{Pressuposto 3 - Antecipação Limitada do Tratamento:} Admite-se que as unidades não são afetadas pelo tratamento antes de sua efetiva implementação, ou que se conheçam efeitos de antecipação limitados e controláveis. Caso haja antecipação, o modelo permite incorporar essa informação, desde que os períodos de antecipação sejam conhecidos e adequadamente modelados.

\item \textbf{Sobreposição (Overlap):} É necessário que haja sobreposição entre as características das unidades tratadas e as unidades de controle, garantindo que as diferenças observadas possam ser atribuídas ao tratamento e não a dessemelhanças estruturais entre grupos.
\end{enumerate}

\subsubsection{Validade dos Pressupostos no Contexto de Estações Meteorológicas}

É importante verificar como estes pressupostos se aplicam ao nosso contexto específico:

\textbf{No Anticipation}: No caso de estações meteorológicas, este pressuposto é amplamente mas não perfeitamente, satisfeito. Embora as informações meteorológicas localizadas e precisas só existam após a instalação física da estação, reconhecemos que produtores podem utilizar dados de estações vizinhas, com menor precisão. Se houver algum grau de antecipação, nosso estimador tende a ser conservador, subestimando o verdadeiro impacto da estação, pois parte do efeito seria capturado antes do período oficial de tratamento. Isso fortalece nossas conclusões: se encontramos efeitos significativos mesmo com possível antecipação, o impacto real tende a ser ainda maior.

Tratamento Irreversível: Uma vez instalada, assume-se que a estação permanece operacional. Nossa análise não considera casos de desativação de estações, tratando a adoção como permanente (staggered adoption).

Tendências Paralelas Condicionais: Este é o pressuposto mais crítico e testável. Nossa análise fornece forte evidência empírica através do teste formal (F = 1,136, p = 0,3215) e da inspeção visual dos períodos pré-tratamento no event study, onde os coeficientes oscilam aleatoriamente em torno de zero sem tendência sistemática.

\subsection{Estimação}

Para estimar o ATT(g,t), são propostas três abordagens principais:

\begin{enumerate}
\item \textbf{Regressão de Resultado (Outcome Regression - OR):} Modela-se diretamente o resultado nos grupos de controle, condicionando a covariáveis pré-tratamento. O efeito é então obtido comparando a predição contrafactual com o resultado efetivo observado nas unidades tratadas.

\item \textbf{Ponderação por Probabilidade Inversa (Inverse Probability Weighting - IPW):} Aqui, pondera-se cada unidade pela probabilidade condicional de tratamento. Ao ajustar esses pesos, obtém-se um contrafactual equilibrado, simulando um cenário onde o tratamento foi aplicado aleatoriamente.

\item \textbf{Duplamente Robusto (Doubly Robust - DR):} Combina OR e IPW, resultando em um estimador robusto a erros de especificação. Mesmo se um dos modelos (resultado ou probabilidade) estiver incorretamente especificado, a consistência pode ser mantida. Na prática, essa abordagem é frequentemente recomendada por oferecer maior segurança em cenários reais, onde a especificação perfeita do modelo é incerta.
\end{enumerate}

\subsection{Procedimentos de Inferência}

Seguindo as recomendações de \citeonline{callaway2021}, todos os erros-padrão e intervalos de confiança reportados neste estudo foram calculados utilizando bootstrap multiplicativo com 1.000 replicações. Este procedimento é particularmente importante por duas razões:

\begin{enumerate}
\item Clustering: Com dados em painel e tratamento ao nível da microrregião, é essencial considerar a correlação dentro dos clusters. O bootstrap multiplicativo implementado no pacote \texttt{did} automaticamente respeita a estrutura de clustering dos dados.

\item Múltiplas Hipóteses: Em análises de event study, múltiplos coeficientes são estimados e testados simultaneamente (um para cada período relativo). O procedimento de bootstrap garante inferência válida mesmo neste contexto de testes múltiplos, fornecendo bandas de confiança uniformes.
\end{enumerate}

A escolha do bootstrap sobre aproximações assintóticas tradicionais também oferece melhor desempenho em amostras finitas, particularmente relevante para períodos pré-tratamento distantes onde o número de observações pode ser menor.

\subsection{Agregação de Efeitos}

Após estimar os ATT(g,t) para cada combinação grupo-tempo, \citeonline{callaway2021} propõem diferentes esquemas de agregação para obter medidas resumidas do efeito do tratamento. A escolha do esquema de agregação depende da questão de pesquisa específica.

\subsubsection{Agregação Simples com Pesos Positivos}

Uma primeira possibilidade seria simplesmente fazer a média de todos os ATT(g,t) identificados:

\begin{equation}
\theta^O_W = \frac{1}{\kappa} \sum_{g \in \mathcal{G}} \sum_{t=2}^{T} \mathbf{1}\{t \geq g\} \cdot ATT(g,t) \cdot P(G = g | G \leq T)
\end{equation}

onde $\kappa = \sum_{g \in \mathcal{G}} \sum_{t=2}^{T} \mathbf{1}\{t \geq g\} \cdot P(G = g | G \leq T)$ garante que os pesos somem um.

Embora $\theta^O_W$ evite os problemas de pesos negativos do TWFE tradicional, ele tem a desvantagem de sistematicamente atribuir mais peso a grupos que participam do tratamento por mais tempo.

\subsubsection{Efeito Médio do Tratamento sobre os Tratados (Recomendado)}

Para superar essa limitação, \citeonline{callaway2021} recomendam o seguinte parâmetro como medida geral do efeito médio de participar do tratamento:

\begin{equation}
\theta^O_{sel} = \sum_{g \in \mathcal{G}} \theta_{sel}(g) \cdot P(G = g | G \leq T)
\end{equation}

onde $\theta_{sel}(g)$ é o efeito médio de participar do tratamento para unidades no grupo $g$:

\begin{equation}
\theta_{sel}(g) = \frac{1}{T - g + 1} \sum_{t=g}^{T} ATT(g,t)
\end{equation}

Este parâmetro primeiro calcula o efeito médio para cada grupo (através de todos os períodos pós-tratamento) e então faz a média desses efeitos entre grupos. Assim, $\theta^O_{sel}$ representa o efeito médio de participar do tratamento experimentado por todas as unidades que alguma vez participaram do tratamento. Sua interpretação é análoga ao ATT no DiD canônico com dois períodos e dois grupos.

\subsubsection{Agregações para Event Studies}

Para análises de event study que examinam a dinâmica temporal dos efeitos, utilizamos a agregação balanceada, que evita problemas de mudanças na composição dos grupos ao longo do tempo relativo ao tratamento.

\section{Especificação do Modelo}

Nossa análise baseia-se em um painel de dados escalonado que pode ser formalmente descrito como $\mathcal{D} = \{(Y_{it}, W_{it}, X_{it})\}_{i=1,t=1}^{N,T}$, onde:

\begin{itemize}
\item $N$: número total de unidades (microrregiões) no painel.
\item $T$: número total de períodos de tempo (anos) no painel.
\item $i \in \{1, 2, \ldots, N\}$: índice que identifica a unidade (microrregião).
\item $t \in \{1, 2, \ldots, T\}$: índice que identifica o período de tempo (ano).
\item $Y_{it}$: logaritmo do valor de produção de cana-de-açúcar (em R\$) da microrregião $i$ no ano $t$.
\item $W_{it}$: indicador binário de tratamento (1 se a microrregião $i$ possui estação meteorológica ativa no ano $t$, 0 caso contrário).
\item $X_{it}$: vetor de covariadas da microrregião $i$ no ano $t$.
\end{itemize}

A abordagem de \citeonline{callaway2021} permite estimar o Efeito Médio do Tratamento sobre os Tratados (ATT) específico para cada coorte $g$ (grupo de unidades tratadas no mesmo período) e tempo $t$, denotado por $ATT(g,t)$. Estes efeitos podem então ser agregados de diferentes formas para obter estimativas de interesse para políticas públicas.


\subsection{Definição do Tratamento e Unidades de Análise}

O tratamento é definido como a instalação de pelo menos uma estação meteorológica automática em funcionamento na microrregião. A escolha da microrregião como unidade de análise, em detrimento do nível municipal, fundamenta-se na própria definição institucional dessas unidades e em considerações técnico-econométricas:

\begin{enumerate}
\item Homogeneidade produtiva e espacial: segundo o IBGE, as microrregiões são definidas com base em dois indicadores fundamentais: (i) estrutura da produção, incluindo análise da utilização da terra, orientação agrícola, estrutura dimensional dos estabelecimentos e relações de produção agropecuária; e (ii) interação espacial, considerando a área de influência de centros sub-regionais. Esta definição institucional garante que agrupamos municípios com características agrícolas similares, tornando as microrregiões unidades ideais para capturar o impacto de infraestrutura informacional sobre decisões de produção agrícola homogêneas.

\item Correspondência à área de influência meteorológica: Uma estação meteorológica fornece dados representativos para um raio de atuação, dependendo da topografia e condições climáticas locais. As microrregiões, por agregarem municípios contíguos com interação espacial estabelecida (conforme critério do IBGE), aproximam municípios que condições climáticas semelhantes. Municípios individuais seriam unidades excessivamente granulares, ignorando os transbordamentos espaciais da informação meteorológica entre localidades vizinhas que compartilham padrões produtivos.

\item Redução de ruído idiossincrático: A agregação de municípios com estrutura produtiva similar (primeiro critério do IBGE) em uma única unidade de análise reduz a influência de choques localizados específicos a municípios individuais (mudanças políticas locais, investimentos privados pontuais), preservando o sinal do tratamento enquanto minimiza ruído nas estimativas.

\item Estabilidade institucional e poder estatístico: As microrregiões mantêm fronteiras estáveis ao longo do período analisado, diferentemente dos municípios sujeitos a desmembramentos. A agregação resulta em 558 unidades cobrindo todo o território brasileiro, com 342 apresentando janela pré-tratamento válida para identificação, oferecendo equilíbrio ideal entre representatividade espacial e tamanho amostral.
\end{enumerate}

Portanto, a microrregião não é apenas uma conveniência estatística mas a unidade natural para análise de políticas agrícolas, pois captura simultaneamente homogeneidade produtiva, interações espaciais e a escala geográfica relevante para difusão de informação meteorológica.

\subsection{Construção dos Grupos de Tratamento}

Seguindo a notação de \citeonline{callaway2021}, definimos $G_i$ como o ano em que a microrregião $i$ recebe sua primeira estação meteorológica. Para unidades nunca tratadas durante o período de análise, convencionamos $G_i = 0$. Esta codificação é essencial para a implementação computacional e permite a utilização dessas unidades como grupo de controle potencial.

A distribuição temporal da adoção revela padrões relevantes: observa-se uma concentração significativa de instalações em 2006-2008, coincidindo com programas federais de expansão da rede meteorológica, seguida por adoção mais esparsa nos anos subsequentes. Das 558 microrregiões analisadas, a maioria eventualmente recebeu estações ao longo do período de estudo.

\subsection{Variável Dependente e Transformações}

A variável dependente principal é o logaritmo natural do valor de produção de cana-de-açúcar, definida como:

\begin{equation}
Y_{it}^{\text{valor}} = \ln(1 + \text{valor\_producao\_cana}_{it})
\end{equation}

onde $\text{valor\_producao\_cana}_{it}$ representa o valor total da produção de cana-de-açúcar (em R\$) na microrregião $i$ no ano $t$, deflacionado pelo Índice de Preços ao Produtor Amplo, segundo o conceito de Oferta Global e Disponibilidade Interna (IPA-OG-DI) da Fundação Getulio Vargas, com ano-base 2021.

A escolha do IPA-OG-DI como deflator justifica-se por sua adequação metodológica ao contexto agrícola. O Índice de Preços ao Produtor Amplo (IPA), calculado pela FGV desde 1947, mensura a variação de preços de produtos agropecuários e industriais nas transações interempresariais, ou seja, nos estágios de comercialização anteriores ao consumo final. O conceito de Oferta Global (OG) refere-se à classificação dos produtos segundo sua origem (agrícola ou industrial), enquanto Disponibilidade Interna (DI) indica que o índice considera apenas a produção disponível no mercado doméstico, excluindo exportações e incluindo importações. Esta especificação captura as pressões inflacionárias que efetivamente afetam os produtores rurais na venda de suas colheitas, diferentemente do IPCA, que reflete preços ao consumidor final e incorpora margens de varejo e serviços. Por aproximar-se do ambiente de formação de preços relevante para a atividade agrícola, o IPA constitui o deflator padrão em estudos de economia agrícola brasileira.

A adoção de 2021 como ano-base segue duas justificativas. Em primeiro lugar, trata-se do último ano do painel de dados, o que facilita a interpretação dos valores em termos de poder de compra corrente e permite comparação direta com estatísticas recentes do setor. Em segundo lugar, a escolha do ano final da série evita distorções associadas a anos atípicos no interior do painel: anos como 2008 ou 2015-2016 foram marcados por volatilidade cambial e de commodities que poderiam introduzir ruído na série deflacionada caso fossem adotados como base.

Complementarmente, analisamos também a área plantada:

\begin{equation}
Y_{it}^{\text{área}} = \ln(1 + \text{area\_plantada\_cana}_{it})
\end{equation}

A escolha do valor de produção como variável de resultado principal justifica-se por quatro razões fundamentais:

\begin{enumerate}
\item Captura completa do impacto econômico: O valor de produção incorpora tanto a margem extensiva (expansão de área cultivada) quanto a margem intensiva (ganhos de produtividade por hectare). Informação meteorológica pode afetar não apenas decisões de quanto plantar mas também quando plantar, qual variedade usar e como otimizar o manejo. O valor captura todos esses canais de impacto.

\item Relevância para política pública: Formuladores de política e stakeholders estão primariamente interessados no impacto econômico total da infraestrutura informacional. O valor de produção conecta diretamente a instalação de estações ao bem-estar econômico regional, facilitando análises de custo-benefício.

\item Especificidade da cana-de-açúcar: Conforme documentado pelo Atlas da Irrigação \cite{ana2017atlas}, a cana possui características únicas de manejo hídrico que a tornam particularmente sensível à disponibilidade de informação meteorológica precisa. A irrigação por salvamento, que representa mais de 90\% da área irrigada de cana no Brasil, pode aumentar a produtividade em até 30\% quando aplicada com timing ótimo.

\item Decomposição do efeito através da área: A análise paralela da área plantada permite decomposição do efeito total. Se o efeito no valor excede o efeito na área, isso indica ganhos de produtividade. Se são similares, o impacto é primariamente via expansão territorial.
\end{enumerate}

A transformação logarítmica $\ln(1+x)$ evita problemas computacionais quando há observações com valor zero (microrregiões sem produção de cana), mantendo essas observações na amostra e reduzindo a heterocedasticidade tipicamente observada em dados econômicos.

\subsection{Covariáveis e Especificação do Modelo}

A especificação do modelo inclui um conjunto de covariáveis socioeconômicas e climáticas cuidadosamente selecionadas para controlar por fatores que podem influenciar tanto a probabilidade de receber uma estação meteorológica quanto o valor de produção de cana-de-açúcar:

\begin{enumerate}
\item Log da área total: Controla pelo tamanho territorial da microrregião, capturando disponibilidade de terras para expansão agrícola.

\item Log da população: Proxy para disponibilidade de mão de obra

\item Log do PIB per capita: Captura o nível de desenvolvimento econômico e capacidade de investimento local em tecnologias agrícolas.

\item Log da densidade de estações na UF: Variável construída agregando o número de estações meteorológicas ao nível estadual, normalizada pela área. Esta variável é crucial para capturar potenciais efeitos de transbordamento regional, reconhecendo que informações meteorológicas podem fluir entre microrregiões vizinhas dentro do mesmo estado.

\item Variáveis climáticas: Incluímos controles para precipitação total anual, precipitação média mensal e precipitação máxima mensal (em logaritmo), todas derivadas de dados ERA5, para isolar o efeito informacional das estações meteorológicas do efeito direto das condições climáticas sobre decisões de plantio.
\end{enumerate}

A inclusão da densidade estadual de estações merece destaque especial. Esta variável permite um pseudo-mapeamento dos efeitos de transbordamento regional, considerando que:

\begin{itemize}
\item Informações meteorológicas têm natureza de bem público, podendo beneficiar áreas além da localização física da estação
\item A instalação de estações melhora a qualidade das previsões meteorológicas para toda a região, criando um efeito sistêmico que se propaga principalmente dentro dos limites estaduais
\item Produtores podem se beneficiar indiretamente da maior densidade de estações no estado através de previsões mais precisas e dados climáticos mais confiáveis
\end{itemize}

\subsection{Conjuntos de dados por cultura e tratamento de zeros estruturais}
\label{subsec:crop_specific_zeros}

Um aspecto metodológico central deste estudo é o tratamento de zeros estruturais nos outcomes por cultura, por meio da construção de amostras \textit{crop-specific}. Em vez de considerar toda a malha de microrregiões, a análise é conduzida em subconjuntos formados apenas por unidades que, em algum momento, produziram a cultura em questão ao longo do período de estudo. O procedimento segue três etapas:

\begin{enumerate}
    \item \textbf{Identificação de produtores}: para cada cultura analisada (cana, soja, arroz), identificam-se as microrregiões que registraram produção positiva em pelo menos um ano entre 2003 e 2021. A condição de produtor em algum momento funciona como proxy de viabilidade agroeconômica mínima para a cultura na microrregião.

    \item \textbf{Filtragem da amostra}: a análise principal restringe-se, então, às 225 microrregiões que produziram cana-de-açúcar em pelo menos um ano do período (41{,}3\% do total), excluindo-se 320 microrregiões nas quais a cultura nunca foi observada. Procedimento análogo é adotado para as bases de soja e arroz.

    \item \textbf{Preservação do painel}: uma vez definido o conjunto \textit{crop-specific} de microrregiões, mantêm-se todas as observações ano--microrregião para essas unidades, incluindo anos com produção igual a zero. Esses zeros são interpretados como zeros econômicos (flutuações de decisão produtiva ou choques transitórios), e não como zeros estruturais decorrentes de inviabilidade física ou econômica da cultura.
\end{enumerate}

Essa estratégia oferece algumas vantagens metodológicas relevantes para a identificação dos efeitos das estações meteorológicas:

\begin{itemize}
    \item \textbf{Maior validade do contrafactual}: ao restringir a comparação a microrregiões em que o outcome é potencialmente acionável, evita-se a construção de contrafactuais entre regiões produtoras e regiões em que a cultura nunca se manifesta ao longo do período. A condição de já ter produzido a cultura em algum momento opera, assim, como uma proxy observada de suporte relevante para a análise.

    \item \textbf{Preservação das margens extensiva e intensiva}: dentro do subconjunto de microrregiões produtoras, preservam-se tanto a margem extensiva (entrada e saída temporária da cultura ao longo do tempo) quanto a margem intensiva (variação na área e no valor produzido condicionalmente à produção positiva), uma vez que zeros temporários permanecem na base.

    \item \textbf{Redução de diluição por zeros estruturais}: ao excluir unidades nas quais a cultura nunca é observada, reduz-se a probabilidade de que zeros estruturais, associados a restrições agroclimáticas ou econômicas severas, atenuem artificialmente os efeitos estimados das estações em regiões efetivamente produtoras.

    \item \textbf{Consistência entre culturas}: a adoção do mesmo critério \textit{crop-specific} para cana, soja e arroz assegura que os resultados permaneçam comparáveis entre culturas, uma vez que, em todos os casos, a identificação é construída sobre subconjuntos de microrregiões com histórico observado de produção.
\end{itemize}

Resultados adicionais (não apresentados em detalhe) indicam que especificações estimadas sem o filtro \textit{crop-specific} tendem a produzir coeficientes de impacto menores e, em diversos casos, estatisticamente não significativos, sugerindo que a inclusão indiscriminada de microrregiões estruturalmente não produtoras pode mascarar efeitos locais relevantes das estações meteorológicas nas cadeias produtivas analisadas.

\subsection{O Estimador Duplamente Robusto}

Para a estimação dos efeitos causais, adotamos o estimador \textit{Duplamente Robusto} (DR) proposto por \citeonline{santanna2020}, que combina modelos de regressão para o resultado com ponderação por probabilidade inversa. Esta abordagem oferece propriedades estatísticas desejáveis:

\begin{itemize}
\item Dupla proteção contra má especificação: O estimador permanece consistente se pelo menos um dos dois modelos (resultado ou score de propensão) estiver corretamente especificado.

\item Eficiência melhorada: Sob especificação correta de ambos os modelos, o DR atinge a fronteira de eficiência semiparamétrica.

\item Robustez a extremos: A combinação de métodos mitiga problemas associados a pesos extremos no IPW puro.
\end{itemize}

\subsection{Escolha do Grupo de Controle}

Uma decisão metodológica importante na implementação do estimador de \citeonline{callaway2021} refere-se à escolha do grupo de controle. O pacote \texttt{did} oferece duas opções principais:

\begin{itemize}
\item Not-yet-treated: Utiliza como controle tanto unidades nunca tratadas quanto unidades ainda não tratadas no período $t$. Esta abordagem maximiza o tamanho da amostra de controle e é particularmente útil em contextos com poucos ou nenhum never-treated.

\item Never-treated: Restringe o grupo de controle apenas às unidades que nunca receberam tratamento durante todo o período amostral. Embora conceitualmente mais limpo, pode resultar em poder estatístico reduzido.
\end{itemize}

Para esta análise, adotamos como padrão o grupo \textbf{not-yet-treated} por três razões: (i) maximiza a eficiência estatística ao utilizar toda a informação disponível; (ii) é apropriado para nosso contexto onde a adoção ocorre gradualmente ao longo do tempo; e (iii) os resultados mostram-se robustos a ambas as especificações (diferença de apenas 2,5\%), validando esta escolha metodológica.

\section{Especificação da Análise de Estudo de Evento}

A análise de estudo de evento constitui o núcleo da estratégia empírica adotada, permitindo examinar como o efeito do tratamento evolui dinamicamente ao longo do tempo. Esta abordagem é particularmente adequada para o contexto analisado por três razões fundamentais:

\begin{enumerate}
\item Teste de tendências paralelas: Permite verificar visualmente e estatisticamente se os grupos tratados e controle seguiam trajetórias similares antes do tratamento, validando o pressuposto fundamental de identificação.

\item Dinâmica de adoção tecnológica: Captura o processo gradual de difusão e aprendizado associado ao uso de informações meteorológicas, reconhecendo que os benefícios podem não ser imediatos.

\item Heterogeneidade temporal: Acomoda a possibilidade de que os efeitos variem com o tempo de exposição ao tratamento, seja por acumulação de conhecimento ou mudanças nas práticas agrícolas.
\end{enumerate}

\subsection{Formalização da Análise de Estudo de Evento}

Definimos o tempo relativo ao tratamento como $e = t - g$, onde $g$ é o ano de instalação da primeira estação e $t$ é o período calendário. Assim, $e < 0$ representa períodos pré-tratamento, $e = 0$ marca o início do tratamento, e $e > 0$ captura períodos pós-tratamento.

A agregação dos efeitos ATT(g,t) em função do tempo relativo segue a especificação:

\begin{equation}
\theta_{es}^{bal}(e) = \sum_{g \in \mathcal{G}} \mathbf{1}\{g + e \leq T\} \cdot P(G = g | G + e \leq T) \cdot ATT(g, g+e)
\end{equation}

onde:
\begin{itemize}
\item $\theta_{es}^{bal}(e)$ representa o efeito médio do tratamento $e$ períodos após sua introdução
\item $\mathcal{G}$ é o conjunto de coortes de adoção (excluindo nunca tratados)
\item $P(G = g | G + e \leq T)$ são pesos que garantem que cada coorte contribua proporcionalmente ao número de unidades tratadas
\item $\mathbf{1}\{g + e \leq T\}$ assegura que incluímos apenas coortes observadas por pelo menos $e$ períodos pós-tratamento
\end{itemize}

Esta especificação garante comparabilidade entre períodos, ponderando adequadamente a contribuição de cada coorte conforme sua representatividade na amostra.

É importante notar que, conforme alertam \citeonline{callaway2021}, event studies longos podem sofrer de mudanças na composição dos grupos contribuindo para cada período relativo. Em nosso caso, com tratamento escalonado de 2000 a 2019, períodos relativos extremos ($e > 15$ ou $e < -15$) são estimados com base em poucas coortes, o que explica a maior variabilidade observada nesses períodos. A agregação balanceada $\theta_{es}^{bal}(e)$ mitiga este problema ao fixar o conjunto de grupos contribuintes.

% ----------------------------------------------------------
% Resultados
% ----------------------------------------------------------
\chapter{Resultados}

\section{Implementação Computacional}

\subsection{Software e Pacotes Utilizados}

A implementação empírica foi realizada utilizando o software R (versão 4.5+) em conjunto com o pacote \texttt{did} (versão 2.1.2), desenvolvido por \citeonline{callaway2021}. O uso deste pacote oficial garante conformidade estrita com os procedimentos propostos no artigo metodológico, implementando fielmente os estimadores e procedimentos de inferência. As principais funcionalidades utilizadas incluem:

\begin{itemize}
\item Cálculo dos ATT(g,t) com inferência via bootstrap clusterizado por microrregião
\item Agregações flexíveis (overall, dynamic/event-study, group, calendar)
\item Múltiplos estimadores: Doubly Robust (DR), IPW e Regressão
\item Tratamento adequado de dados desbalanceados e zeros via transformação log1p
\end{itemize}

Complementarmente, utilizamos os pacotes \texttt{tidyverse} para manipulação de dados e visualizações, \texttt{foreach/doParallel} para paralelização dos testes Monte Carlo, e \texttt{gt/kableExtra} para geração de tabelas. A reprodutibilidade é garantida através do sistema \texttt{renv} de gerenciamento de dependências, com todos os pacotes versionados no arquivo \texttt{renv.lock}.

\subsection{Transparência e Reprodutibilidade}

Em alinhamento com as melhores práticas de ciência aberta, todo o código está disponível publicamente no repositório GitHub \url{https://github.com/danielcavalli/tcc-ie-ufrj-2024}. O repositório implementa reprodutibilidade completa através de:

\begin{itemize}
\item Sistema \texttt{renv} com versionamento exato de todos os pacotes R
\item \texttt{Makefile} documentando todos os comandos de execução
\item Dados processados incluídos (CSV) com documentação das fontes
\item Scripts python para replicação da base de dados desde a origem
\item Histórico completo de versionamento via Git
\end{itemize}

\subsection{Estrutura dos Dados e Processo de Extração}

O conjunto de dados foi construído através de um processo sistemático de extração e agregação utilizando Python, o pacote \texttt{basedosdados} e a API do Google BigQuery. O script python \texttt{analise\_did\_microrregions.py}, disponível no repositório do projeto, documenta todo o processo de construção do dataset. As etapas principais incluem:

\subsubsection{Fontes de Dados e Extração}

\begin{enumerate}
\item Mapeamento Município-Microrregião: Extraído da tabela\\
\texttt{\small br\_bd\_diretorios\_brasil.municipio}, identificando 5.570 municípios em 558 microrregiões brasileiras.

\item Estações Meteorológicas: Dados de 610 estações do Instituto Nacional de Meteorologia (INMET) extraídos da tabela \texttt{\small br\_inmet\_bdmep.estacao}, incluindo coordenadas geográficas e data de fundação. Após agregação por microrregião, identificamos 394 microrregiões com pelo menos uma estação (70,6\% de cobertura).

\item População Municipal: Dados anuais da tabela \texttt{\small br\_ibge\_populacao.municipio}, agregados para o nível de microrregião através de soma simples.

\item PIB Municipal: Valores totais e agropecuários extraídos de\\
\texttt{\small br\_ibge\_pib.municipio}, incluindo PIB total e valor adicionado da agropecuária.

\item Valor de Produção Agrícola (PAM): Dados da Pesquisa Agrícola Municipal do IBGE extraídos da tabela \texttt{\small br\_ibge\_pam.lavoura\_temporaria}, fornecendo valores de produção em reais para cana-de-açúcar, soja e arroz ao nível municipal.

\item Área Plantada por Cultura (MapBiomas): Dados de cobertura e uso do solo derivados de classificação de imagens de satélite Landsat. Para cada cultura agrícola (cana-de-açúcar, soja, arroz), obtivemos a área total em km² ao nível municipal, posteriormente agregada para microrregiões. Estes dados oferecem vantagens significativas sobre estimativas autorreportadas:
\begin{itemize}
\item Medição objetiva baseada em sensoriamento remoto
\item Consistência temporal na metodologia de classificação
\item Cobertura completa do território nacional
\item Redução de vieses associados a estimativas subjetivas
\end{itemize}

\item Variáveis Climáticas (ERA5): Dados de precipitação em alta resolução espacial, incluindo totais anuais e médias mensais, utilizados como controles climáticos na especificação econométrica.
\end{enumerate}

\subsubsection{Construção das Variáveis de Tratamento}

O tratamento foi definido como a instalação da primeira estação meteorológica automática em funcionamento na microrregião. Para cada microrregião $i$:

\begin{itemize}
\item $G_i$ = ano da primeira estação instalada (0 se nunca tratada)
\item \texttt{tratado} = 1 se $G_i > 0$, 0 caso contrário
\item \texttt{pos\_tratamento} = 1 se $\text{ano} \geq G_i$ e \texttt{tratado} = 1
\end{itemize}

Das 558 microrregiões no dataset final,394 foram tratadas em algum momento (70,6\%) e 164 permaneceram como controle durante todo o período.

\subsubsection{Tratamento de dados faltantes e qualidade da base}
\label{subsubsec:missing_quality}

A implementação do estimador de Callaway e Sant'Anna (2021) requer uma estrutura de painel relativamente completa e consistente ao longo do tempo. Nesta subseção, sintetizam-se as principais decisões de tratamento de dados que sustentam a qualidade da base utilizada nos exercícios empíricos.

\begin{enumerate}
    \item \textbf{Completude dos dados}: o conjunto final de dados não apresenta valores ausentes para as variáveis principais de interesse (população, PIB e produção agrícola). Essa característica dispensa o uso de procedimentos de imputação, evitando a introdução de incerteza adicional associada a hipóteses específicas de preenchimento de lacunas e assegurando que as estimativas reflitam diretamente a variação observada nas séries originais.

    \item \textbf{Zeros estruturais e definição de suporte}: anos sem produção são mantidos como valores zero, e não tratados como dados faltantes, em consonância com a estratégia \textit{crop-specific} descrita na Seção~\ref{subsec:crop_specific_zeros}. Dessa forma, zeros temporários dentro do conjunto de microrregiões produtoras são interpretados como variação econômica (margem extensiva e intensiva), ao passo que zeros estruturais (microrregiões que nunca produziram a cultura no período) são excluídos da amostra analítica.

    \item \textbf{Consistência territorial}: a agregação de informações municipais para o nível de microrregiões segue o mapeamento oficial do Instituto Brasileiro de Geografia e Estatística (IBGE), o que garante consistência territorial ao longo de todo o período 2003--2021, mesmo diante de eventuais alterações na malha municipal.

    \item \textbf{Clustering de erros-padrão}: como os dados constituem um painel microrregião $\times$ ano, é plausível que os erros de uma mesma microrregião apresentem correlação ao longo do tempo, devido a características não observadas persistentes ou choques comuns recorrentes. Para obter inferência robusta, os erros-padrão são clusterizados ao nível da microrregião, permitindo correlação arbitrária dos resíduos dentro de cada unidade ao longo dos 21 anos e assumindo independência entre microrregiões. Esse procedimento tende a produzir erros-padrão mais conservadores e confere maior confiabilidade aos testes de significância estatística.
\end{enumerate}

O \textit{script python} de extração e tratamento foi construído de forma modular, permitindo reproduzir integralmente o processo de limpeza e, se desejado, replicar a análise para outros produtos agrícolas mediante simples alteração do parâmetro \texttt{PRODUTOS\_AGRICOLAS}. Essa modularidade facilita exercícios de robustez e extensões comparativas com outras culturas.

\section{Resultados Principais}

\subsection{Efeito Médio do Tratamento}

A estimação do efeito médio do tratamento sobre os tratados (ATT) via estimador doubly robust revela um impacto positivo e estatisticamente significativo da instalação de estações meteorológicas sobre o logaritmo do valor de produção de cana-de-açúcar:

\textbf{ATT Principal (Valor de Produção) = \mainatt} (EP = \mainse, z = \mainz, p = \mainp, IC 95\%: [\maincilower; \mainciupper])

Como a variável dependente está em logaritmo, este coeficiente pode ser interpretado aproximadamente como uma variação percentual. Assim, as microrregiões produtoras de cana que receberam estações meteorológicas experimentaram, em média, um aumento de aproximadamente \textbf{\mainattpct} no valor de produção em relação ao contrafactual de não receber a estação.

\subsubsection{Decomposição do Efeito: Margens Extensiva e Intensiva}

Para compreender os mecanismos subjacentes, analisamos também o impacto sobre a área plantada como outcome secundário:

\textbf{ATT (Área Plantada) = \areacanaatt} (EP = \areacanase, p = \areacanap)

O efeito sobre a área plantada de \areacanaattpct{} indica expansão na margem extensiva. A comparação entre os dois efeitos revela que o impacto no valor de produção (\mainattpct) excede substancialmente o efeito na área (\areacanaattpct), sugerindo ganhos significativos na margem intensiva através de melhorias na produtividade. Esta evidência é consistente com a hipótese de que a informação meteorológica permite não apenas decisões de expansão (quando e onde plantar) mas também otimização do manejo agrícola (variedades, irrigação, aplicação de insumos).

A Tabela \ref{tab:main_results} apresenta os resultados principais e especificações de robustez:

\begin{table}[htbp]
\centering
\caption{Resultados Principais: Efeito sobre o Logaritmo do Valor de Produção de Cana-de-Açúcar}
\label{tab:main_results}
\begin{tabular}{lccc}
\toprule
Especificação & ATT & EP & IC 95\% \\
\midrule
\textbf{Especificação Principal (DR)} & \mainatt** & (\mainse) & [\maincilower; \mainciupper] \\
\midrule
\multicolumn{4}{l}{\textit{Especificações Alternativas}} \\
Sem Covariáveis & \nocovatt** & (\nocovse) & [\nocovlower; \nocover] \\
IPW & \ipwatt** & (\ipwse) & [\ipwlower; \ipwupper] \\
Regressão de Resultado & \regatt** & (\regse) & [\reglower; \regupper] \\
\midrule
\multicolumn{4}{l}{\textit{Grupo de Controle Alternativo}} \\
Never-treated & \nevertreatedatt** & (\nevertreatedse) & [\nevertreatedlower; \nevertreatedupper] \\
\bottomrule
\end{tabular}
\end{table}

\textit{Notas: *** p<0,01, ** p<0,05, * p<0,10. Erros-padrão clusterizados ao nível da microrregião. A análise principal utiliza 4.275 observações (225 microrregiões produtoras de cana × 19 anos). O filtro crop-specific garante comparações válidas entre regiões com potencial agrícola similar. Os testes com culturas alternativas (soja, arroz) aplicam filtros análogos para suas respectivas amostras de produtores.}

\textit{Fonte: Elaboração própria a partir dos dados do estudo.}

\subsubsection{Interpretação da Robustez dos Resultados}

A consistência dos resultados entre diferentes especificações fornece evidência crucial sobre a confiabilidade das estimativas:

\begin{enumerate}
\item Estabilidade entre métodos de estimação: Os três estimadores (Doubly Robust, IPW e Regressão de Resultado) produzem ATTs notavelmente similares, todos significativos ao nível de 1\%. Esta convergência indica que nossos resultados não dependem de pressupostos específicos de um único método mas refletem um efeito causal genuíno robusto a diferentes abordagens de identificação.

\item Robustez à especificação de covariáveis: A especificação sem covariáveis permanece próxima ao modelo principal, sugerindo que o efeito não é dirigido por fatores de confusão observáveis. Isso fortalece a interpretação causal, pois indica que a variação exógena na instalação de estações é suficiente para identificar o efeito.

\item Consistência entre grupos de controle: A estimativa usando apenas unidades never-treated é ligeiramente superior mas estatisticamente indistinguível da especificação principal com not-yet-treated. Esta similaridade valida o pressuposto de tendências paralelas entre diferentes escolhas de grupo de controle.

\item Implicação econômica da robustez: A estreita faixa de variação dos ATTs entre todas as especificações estabelece limites confiáveis para o impacto econômico, com magnitude economicamente significativa que justifica investimentos em infraestrutura meteorológica.
\end{enumerate}

\subsection{Análise de Estudo de Evento e Dinâmica Temporal}

A análise de estudo de evento fornece evidências fundamentais sobre a evolução temporal dos efeitos do tratamento. A Figura \ref{fig:eventstudy} apresenta as estimativas pontuais e intervalos de confiança para períodos relativos ao início do tratamento.

\begin{figure}[H]
\centering
\caption{Estudo de Evento - Dinâmica Temporal dos Efeitos da Instalação de Estações Meteorológicas}
\label{fig:eventstudy}
\includegraphics[width=0.75\textwidth]{../../../data/outputs/presentation/event_study_enhanced.png}

Nota: A figura apresenta as estimativas pontuais (linha azul) e intervalos de confiança de 95\% (área sombreada) dos efeitos do tratamento em função do tempo relativo à instalação da estação. O período e=0 marca o ano de instalação.

Fonte: Elaboração própria a partir dos dados do estudo.
\end{figure}

A Figura \ref{fig:eventstudy_area_cana} apresenta o estudo de evento para a área plantada de cana-de-açúcar, representando a margem extensiva do efeito das estações meteorológicas. Este resultado complementa a análise do valor de produção ao mostrar como os produtores ajustam suas decisões de alocação de terra em resposta à disponibilidade de informações meteorológicas.

\begin{figure}[H]
\centering
\caption{Estudo de Evento: Impacto na Área Plantada de Cana-de-açúcar (Margem Extensiva)}
\label{fig:eventstudy_area_cana}
\includegraphics[width=0.75\textwidth]{../../../data/outputs/presentation/event_study_area_cana_enhanced.png}

Nota: A figura apresenta as estimativas pontuais e intervalos de confiança de 95\% dos efeitos do tratamento na área plantada de cana-de-açúcar. Este resultado representa a margem extensiva, indicando mudanças nas decisões de alocação de terra.

Fonte: Elaboração própria a partir dos dados do estudo.
\end{figure}


\subsubsection{Período Pré-Tratamento: Validação das Tendências Paralelas}
\label{sec:parallel_trends}

A análise dos períodos anteriores ao tratamento (e < 0) é fundamental para validar o pressuposto de identificação. Um aspecto crucial revelado pelo gráfico é que, antes da instalação das estações, os efeitos estimados oscilam aleatoriamente em torno de zero, indicando que o impacto das estações meteorológicas ainda não era sentido pelas microrregiões, exatamente como esperado se o tratamento for exógeno.

Para garantir robustez na validação das tendências paralelas, implementamos três testes complementares:

1. Análise Visual do Event Study (Teste Informal)

O gráfico de event study mostra que os efeitos pré-tratamento:
\begin{itemize}
\item Apresentam média de 0,0122 (DP = 0,124), estatisticamente indistinguível de zero (teste t: p = 0,2924)
\item Oscilam aleatoriamente sem padrão sistemático crescente ou decrescente
\item Demonstram variabilidade consistente com flutuações aleatórias esperadas
\end{itemize}

Embora este seja um indicativo importante, a análise visual sozinha não é suficientemente robusta para validar o pressuposto.

2. Teste F de Tendências Paralelas por Coorte

Este teste formal avalia se as tendências pré-tratamento diferem sistematicamente entre grupos definidos pelo ano de adoção (coortes). Especificamente:
\begin{itemize}
\item Hipótese nula: As taxas de crescimento pré-tratamento são iguais entre coortes
\item Metodologia: Regressão com interações coorte × tempo no período pré-tratamento
\item Resultado: F-statistic = 1,136 (p-valor = 0,3215)
\item Interpretação: Não rejeitamos a hipótese nula, fornecendo forte evidência de tendências paralelas
\end{itemize}

Esta análise revela que, no período pré-tratamento, todos os grupos seguem trajetórias paralelas, divergindo apenas após o tratamento, padrão consistente com causalidade.

A convergência dos testes (visual e estatístico formal) fornece evidência robusta de que o pressuposto de tendências paralelas é válido, legitimando a interpretação causal dos resultados.

\subsubsection{Dinâmica Pós-Tratamento: Difusão Gradual dos Benefícios}

O padrão temporal dos efeitos pós-tratamento revela uma dinâmica interessante: os benefícios não são imediatos mas crescem gradualmente ao longo do tempo. Isso sugere um processo de adaptação e aprendizado no uso das informações meteorológicas.

Este padrão é consistente com um processo de difusão tecnológica onde:
\begin{enumerate}
\item A informação meteorológica precisa ser interpretada e integrada às decisões de plantio
\item Os agricultores aprendem gradualmente a otimizar o uso das informações
\item Efeitos de rede emergem conforme mais produtores adotam melhores práticas
\end{enumerate}

\FloatBarrier
\section{Testes de Robustez e Diagnósticos}

Para avaliar de forma sistemática a credibilidade dos resultados, implementamos uma bateria abrangente de testes de robustez e diagnósticos. Esta seção documenta quatro dimensões principais: a sensibilidade à escolha do grupo de controle, a evidência de inferência baseada em randomização por meio do teste de Monte Carlo, a especificidade do efeito a variáveis de resultado alternativas e a robustez das estimativas a diferentes métodos de estimação. Ao final, sintetizamos como essas evidências se articulam em uma interpretação causal consistente.

\subsection{Sensibilidade ao Grupo de Controle}

Em modelos de diferenças em diferenças com adoção escalonada, a definição do grupo de controle pode afetar tanto a magnitude quanto a significância das estimativas. Testamos duas especificações alternativas, utilizando como contrafactual o conjunto de unidades not-yet-treated e, em seguida, o conjunto de unidades never-treated.

\begin{table}[htbp]
\centering
\caption{Comparação de Estimativas por Grupo de Controle}
\label{tab:controle}
\begin{tabular}{lccc}
\toprule
Grupo de Controle & ATT & Erro Padrão & IC 95\% \\
\midrule
Not-yet-treated & \mainatt & \mainse & [\maincilower; \mainciupper] \\
Never-treated & \nevertreatedatt & \nevertreatedse & [\nevertreatedlower; \nevertreatedupper] \\
\bottomrule
\end{tabular}

\textit{Fonte: Elaboração própria a partir dos dados do estudo.}
\end{table}

As estimativas para os dois grupos de controle são positivas e estatisticamente significativas, com intervalos de confiança amplamente sobrepostos. Essa convergência indica que não há evidência de viés de seleção diferencial relevante entre as especificações, que as unidades ainda não tratadas constituem controles válidos para identificação causal e que o pressuposto de tendências paralelas se mantém de forma robusta em relação à escolha do grupo de controle. Em particular, a magnitude do efeito estimado permanece estável, sugerindo que a estratégia de identificação não é sensível à forma específica de definição do contrafactual.

\subsection{Inferência Baseada em Randomização: Teste de Monte Carlo}

Para além da comparação entre grupos de controle, avaliamos se o ATT estimado poderia ser explicado por mero acaso no timing de tratamento. Para isso, implementamos um teste de randomização de Monte Carlo que constrói uma distribuição de referência dos efeitos sob a hipótese nula de ausência de impacto das estações meteorológicas.

\subsubsection{Formalização do teste}

Consideramos o conjunto de dados $\mathcal{D} = \{(Y_{it}, W_{it}, X_{it})\}_{i=1,t=1}^{N,T}$ (conforme definido na Seção 2.3), em que:

\begin{itemize}
  \item $\mathcal{W} = \{W_{it}\}_{i,t}$ é a matriz de tratamento observada.
  \item $\text{ATT}(\mathcal{W})$ denota o efeito médio do tratamento estimado pelo método doubly robust dado $\mathcal{W}$.
  \item $\mathcal{S}$ é o espaço de todas as possíveis atribuições de tratamento.
\end{itemize}

A hipótese nula é de inexistência de efeito causal:
\begin{equation}
H_0: Y_{it}(1) = Y_{it}(0) \quad \forall i,t.
\end{equation}

Nesse caso, qualquer atribuição de tratamento $\mathcal{W} \in \mathcal{S}$ deveria ser igualmente plausível, de modo que a distribuição dos ATTs sob permutações aleatórias do tratamento fornece a distribuição nula apropriada.

\subsubsection{Procedimento de randomização}

Como o número total de permutações possíveis é extremamente elevado, aproximamos a distribuição exata por meio de simulações de Monte Carlo. O algoritmo segue os passos:

\begin{enumerate}
  \item \textbf{Geração de atribuições aleatórias}

  Para cada uma das $S$ simulações,
  \begin{itemize}
    \item selecionamos aleatoriamente $N_{\text{tratado}}$ microrregiões para receber o tratamento, mantendo o mesmo número de unidades tratadas do estudo original;
    \item para cada unidade selecionada, atribuímos aleatoriamente um ano de instalação da estação no intervalo $[2005, 2019]$.
  \end{itemize}

  A restrição temporal garante pelo menos dois anos de dados antes e depois do tratamento para cada unidade.

  \item \textbf{Estimação dos ATTs placebo}

  Para cada atribuição aleatória $s$, estimamos o ATT pelo mesmo estimador doubly robust utilizado na análise principal:
  \begin{equation}
  \text{ATT}^{(s)} = \text{ATT}(\mathcal{W}^{(s)}), \quad s = 1, \ldots, S.
  \end{equation}

  \item \textbf{Cálculo do p-valor empírico}

  Seguindo \cite{davison1997}, o p-valor empírico é dado por
  \begin{equation}
  \hat{p} = \frac{1 + \#\{\text{extremos}\}}{S + 1},
  \end{equation}
  em que $\#\{\text{extremos}\}$ é o número de simulações em que o ATT em valor absoluto é maior ou igual ao ATT observado. Essa correção de amostra finita impede a obtenção de p-valores exatamente nulos e é padrão em testes de permutação.
\end{enumerate}

A intuição é direta: se o tratamento não tivesse efeito, o ATT observado não deveria estar na cauda da distribuição gerada por atribuições aleatórias. Um ATT verdadeiro muito extremo nessa distribuição fornece evidência contra $H_0$.

\subsubsection{Aspectos computacionais}

A implementação do teste incorpora algumas escolhas práticas relevantes:

\begin{itemize}
  \item \textbf{Margem temporal:} Restringimos $t_i^* \in [2005, 2019]$ para assegurar janelas mínimas de observação antes e depois do tratamento.
  \item \textbf{Paralelização:} Utilizamos computação paralela via \texttt{foreach} e \texttt{doParallel} em R, distribuindo as $S$ simulações entre múltiplos núcleos.
  \item \textbf{Validação:} Apenas simulações com convergência bem-sucedida entram no cálculo de $\hat{p}$.
  \item \textbf{Reprodutibilidade:} Cada simulação utiliza uma semente aleatória única ($\text{seed} \times 1000 + s$), o que garante reprodutibilidade mesmo em ambiente paralelo.
\end{itemize}

\subsubsection{Resultados do teste de Monte Carlo}

A Tabela \ref{tab:placebo_results} resume as estatísticas do teste de randomização.

\begin{table}[htbp]
\centering
\caption{Resultados do teste de randomização de Monte Carlo}
\label{tab:placebo_results}
\begin{tabular}{lc}
\toprule
\textbf{Estatística} & \textbf{Valor} \\
\midrule
Número de simulações & \placebonsims{} \\
ATT observado & \placebotruatt \\
Média dos placebos & \placebomean \\
Desvio padrão dos placebos & \placebosd \\
Intervalo empírico 95\% & [\placebolower; \placeboupper] \\
P-valor empírico & \placebopvalue \\
\bottomrule
\end{tabular}

\textit{Notas: o teste simula \placebonsims{} atribuições aleatórias de tratamento, preservando a estrutura temporal do painel. O p-valor empírico corresponde à proporção de simulações em que o ATT em valor absoluto é maior ou igual ao observado.}

\textit{Fonte: Elaboração própria a partir dos dados do estudo.}
\end{table}

A Figura \ref{fig:placebo} apresenta a distribuição dos ATTs placebo em comparação com o ATT verdadeiro.

\begin{figure}[H]
\centering
\caption{Distribuição dos ATTs placebo em comparação com o ATT verdadeiro}
\label{fig:placebo}
\includegraphics[width=0.8\textwidth]{../../../data/outputs/placebo_distribution.png}

Nota: o histograma mostra a distribuição dos ATTs estimados em \placebonsims{} iterações do teste de randomização, em que o tratamento é atribuído aleatoriamente. Cada barra representa a frequência de ATTs placebo em cada intervalo. A linha vermelha tracejada indica o ATT do modelo principal (\placebotruatt). A distribuição placebo está aproximadamente centrada em zero, enquanto o ATT verdadeiro se localiza na cauda direita, ocorrendo em proporção igual a \placebopvaluepct{} das simulações aleatórias (p-valor empírico \placebopvalue). Isso indica que um efeito da magnitude observada é pouco compatível com flutuações puramente aleatórias no timing do tratamento.

\textit{Fonte: Elaboração própria a partir dos dados do estudo.}
\end{figure}

\subsubsection{Interpretação}

O teste de randomização indica que o ATT observado \placebotruatt{} ocupa a cauda da distribuição gerada por atribuições aleatórias de tratamento, com p-valor empírico \placebopvalue. Em termos práticos, a probabilidade de obter um efeito de magnitude ao menos tão elevada quanto o estimado apenas por coincidências no timing de instalação das estações é igual a \placebopvaluepct{}. O resultado é incompatível com variação aleatória e permite rejeitar com confiança a hipótese de ausência de efeito.

É importante notar que esse teste afasta a hipótese de que o resultado decorre apenas do calendário de adoção mas não elimina outras ameaças potenciais à identificação, como variáveis omitidas ou choques específicos que afetem simultaneamente tratamento e desfechos. A interpretação causal continua condicionada à validade dos pressupostos discutidos na seção de metodologia.

\subsection{Especificidade do Efeito: Variáveis de Resultado Alternativas}

Uma característica central de um efeito causal genuíno é sua especificidade. Se as estações meteorológicas afetam decisões de manejo hídrico específicas da cana-de-açúcar, esperamos observar impactos concentrados em variáveis diretamente relacionadas a essa cultura e não em outras culturas agrícolas.

Para explorar essa dimensão, reestimamos o modelo principal para um conjunto de variáveis de resultado alternativas. A Tabela \ref{tab:alternative_outcomes} resume os efeitos estimados.

\begin{table}[htbp]
\centering
\caption{Especificidade do efeito: variáveis de resultado alternativas}
\label{tab:alternative_outcomes}
\begin{tabular}{lcccl}
\toprule
\textbf{Variável de resultado} & \textbf{ATT} & \textbf{EP} & \textbf{IC 95\%} & \textbf{p-valor} \\
\midrule
\textit{Variáveis principais (cana-de-açúcar)} & & & & \\
Valor de produção (log) & \mainatt** & (\mainse) & [\maincilower; \mainciupper] & \mainp \\
Área plantada (log) & \areacanaatt** & (\areacanase) & [\areacanalower; \areacanaupper] & \areacanap \\
\midrule
\textit{Culturas alternativas} & & & & \\
Valor soja (log) & \valorsojaat & (\valorsojase) & [\valorsojlower; \valorsojaupper] & \valorsojap \\
Valor arroz (log) & \valorarrozatt & (\valorarrozse) & [\valorarrozlower; \valorarrozupper] & \valorarrozp \\
Área soja (log) & \areasojaat & (\areasojase) & [\areasojlower; \areasojaupper] & \areasojap \\
Área arroz (log) & \areaarrozatt & (\areaarrozse) & [\areaarrozlower; \areaarrozupper] & \areaarrozp \\
Valor outras lavouras (log) & \valoroutrasatt & (\valoroutrasse) & [\valoroutraslower; \valoroutrasupper] & \valoroutrasp \\
Área outras lavouras (log) & \areaoutrasatt & (\areaoutrasse) & [\areaoutraslower; \areaoutrasupper] & \areaoutrasp \\
\bottomrule
\end{tabular}

\textit{Notas: *** p<0,01, ** p<0,05, * p<0,10. Todos os modelos utilizam o estimador doubly robust com filtros crop-specific. Cada cultura é analisada apenas nas microrregiões onde é efetivamente produzida, o que evita zeros estruturais e garante comparações entre unidades com potencial agrícola semelhante.}

\textit{Fonte: Elaboração própria a partir dos dados do estudo.}
\end{table}

Os resultados sugerem um padrão de especificidade nítido. As variáveis diretamente ligadas à cana-de-açúcar apresentam efeitos positivos e estatisticamente significativos, tanto em valor de produção (\mainattpct) quanto em área plantada (\areacanaattpct). A diferença entre essas magnitudes é informativa: um efeito mais intenso em valor do que em área indica que as estações estão associadas não apenas à expansão territorial da cultura na margem extensiva mas também a ganhos de produtividade na margem intensiva.

Por outro lado, nem soja nem arroz exibem efeitos significativos, seja em valor de produção, seja em área. Isso sugere que o impacto das estações não corresponde a um choque agrícola amplo mas está vinculado a características específicas da cana-de-açúcar e do seu manejo hídrico.

A Figura \ref{fig:alternative_outcomes} apresenta uma síntese visual desses resultados.

\begin{figure}[H]
\centering
\caption{Efeitos sobre variáveis de resultado alternativas: especificidade à cana-de-açúcar}
\label{fig:alternative_outcomes}
\includegraphics[width=0.85\textwidth]{../../../data/outputs/alternative_outcomes_effects.png}

Nota: o gráfico reporta os ATTs estimados e intervalos de confiança de 95\% para as variáveis de resultado analisadas. Os efeitos sobre valor de produção e área plantada de cana-de-açúcar são estatisticamente significativos, enquanto as variáveis associadas a soja e arroz não apresentam impactos relevantes. A análise utiliza filtros crop-specific, garantindo comparações apenas entre produtores de cada cultura.

\textit{Fonte: Elaboração própria a partir dos dados do estudo.}
\end{figure}

Do ponto de vista mecanístico, essa especificidade é consistente com as evidências agronômicas documentadas pelo Atlas da Irrigação da Agência Nacional de Águas \cite{ana2017atlas}. A cana-de-açúcar responde de forma intensa à qualidade da informação meteorológica, especialmente em sistemas com irrigação de salvamento, que representam a maior parte da área irrigada no Brasil. Nesses sistemas, decisões de timing da irrigação após cada corte anual dependem fortemente de previsões atualizadas de precipitação e condições climáticas de curto prazo. Melhorar a qualidade dessa informação tende a elevar a produtividade por hectare e, ao mesmo tempo, reforçar incentivos para expansão da área plantada.

\subsection{Robustez a Diferentes Métodos de Estimação}

Outro aspecto relevante diz respeito à dependência dos resultados em relação ao método econométrico específico. Para avaliar essa dimensão, comparamos três estimadores amplamente utilizados em contextos de diferenças em diferenças com adoção escalonada: o estimador doubly robust, um estimador ponderado por propensity score (IPW) e um estimador de regressão baseado apenas na modelagem do resultado condicional às covariáveis.

\begin{table}[htbp]
\centering
\caption{Comparação de métodos de estimação}
\label{tab:metodos}
\begin{tabular}{lccc}
\toprule
Método & ATT & Erro padrão & P-valor \\
\midrule
Doubly robust (DR) & \mainatt & \mainse & \mainp \\
IPW & \ipwatt & \ipwse & 0,003 \\
Regression (REG) & \regatt & \regse & 0,019 \\
\bottomrule
\end{tabular}

\textit{Fonte: Elaboração própria a partir dos dados do estudo.}
\end{table}

Cada abordagem explora uma dimensão distinta da estrutura causal. O estimador doubly robust combina modelagem do resultado e do tratamento e é consistente se pelo menos um dos modelos estiver corretamente especificado. O IPW utiliza apenas os pesos derivados do propensity score, priorizando o balanceamento das covariáveis entre tratados e controles. O estimador de regressão, por sua vez, baseia-se exclusivamente na modelagem do resultado condicional, assumindo que o controle por covariáveis observáveis é suficiente para capturar diferenças sistemáticas entre grupos.

As três estratégias produzem estimativas positivas e estatisticamente significativas, com magnitudes próximas (DR: \mainattpct{}, IPW: 42,9\%, REG: 45,2\%). Essa estabilidade sugere que o resultado não é artefato de uma especificação particular mas reflete um padrão robusto na relação entre instalação de estações meteorológicas e valor de produção de cana.

A Figura \ref{fig:robustness} resume graficamente essa comparação.

\begin{figure}[H]
\centering
\caption{Análise de robustez: comparação de especificações e métodos de estimação}
\label{fig:robustness}
\includegraphics[width=0.75\textwidth]{../../../data/outputs/robustness_plot.png}

Nota: o gráfico apresenta estimativas pontuais e intervalos de confiança de 95\% para diferentes especificações de modelo e métodos de estimação. Todas as estimativas são estatisticamente significativas e de magnitude semelhante, reforçando a robustez dos resultados.

\textit{Fonte: Elaboração própria a partir dos dados do estudo.}
\end{figure}

\subsection{Síntese da Robustez dos Resultados}

O conjunto de evidências de robustez aponta para uma interpretação causal consistente dos efeitos estimados.

Primeiro, a análise de sensibilidade ao grupo de controle mostra que as estimativas são estáveis quando comparamos unidades tratadas a not-yet-treated ou a never-treated. As magnitudes e níveis de significância são semelhantes, o que indica ausência de viés de composição relevante associado à definição do contrafactual.

Segundo, o teste de randomização de Monte Carlo demonstra que o ATT verdadeiro ocupa a cauda da distribuição gerada por atribuições aleatórias de tratamento. O p-valor empírico \placebopvalue{} implica que um efeito de magnitude igual ou superior à observada é pouco provável sob a hipótese de ausência de impacto das estações. Essa evidência reforça a leitura de que o resultado não decorre de coincidências no calendário de adoção.

Terceiro, a análise de variáveis de resultado alternativas revela que o efeito é específico à cana-de-açúcar. Culturas alternativas não apresentam impactos significativos, o que é consistente com o mecanismo de transmissão via decisões de irrigação e manejo hídrico detalhado na literatura agronômica.

Por fim, a comparação entre diferentes métodos de estimação indica que o ATT permanece positivo, estatisticamente significativo e de magnitude semelhante em todas as abordagens consideradas. A combinação de estabilidade entre grupos de controle, evidência de randomização, especificidade do efeito e robustez a métodos sugere que as estações meteorológicas geram impactos econômicos substantivos e persistentes sobre o valor de produção da cana-de-açúcar.

Em conjunto, esses resultados sustentam uma interpretação na qual o aumento da densidade de estações melhora a qualidade da informação meteorológica disponível para produtores, o que se traduz em decisões mais eficientes de irrigação e manejo, elevação da produtividade por hectare e expansão da área plantada.

\section{Discussão e Interpretação Econômica}

\subsection{Contextualização da magnitude do efeito}

O efeito estimado de \mainattpct{} sobre o valor de produção da cana-de-açúcar tem dimensão econômica expressiva. Considerando que o valor bruto de produção do setor alcançou aproximadamente R\$ 105 bilhões em 2024, esse impacto corresponde a um acréscimo potencialmente relevante na geração de renda nas regiões canavieiras. A decomposição entre expansão de área (\areacanaattpct) e ganhos de produtividade indica que a melhoria na informação meteorológica afeta simultaneamente decisões de alocação de terra e de manejo dentro das áreas já cultivadas.

A resposta na margem extensiva aparece na expansão da área plantada, sugerindo que produtores incorporam ao cultivo parcelas antes destinadas a outras culturas ou a usos alternativos. Esse resultado tem implicações diretas para o planejamento territorial em regiões canavieiras, pois uma infraestrutura meteorológica mais densa pode induzir realocações de uso do solo em direção a atividades mais intensivas em informação climática.

Do ponto de vista de política pública, o desenho recente da política federal reforça a relevância empírica desses achados. Em setembro de 2025, o Ministério da Agricultura, Pecuária e Abastecimento (MAPA) anunciou um investimento adicional de R\$ 49 milhões para a instalação de 220 novas estações meteorológicas automáticas \cite{mapa2025}, o que implica um custo médio em torno de R\$ 223 mil por estação. À luz dos resultados deste estudo, esses gastos apresentam retorno mensurável sobre o valor de produção da cana-de-açúcar, o que fornece uma justificativa econômica clara para a continuidade e a ampliação desse tipo de investimento.

A análise também revela espaço relevante para expansão da rede. Entre as microrregiões da amostra de microregiões produtoras, 67 (29,7\%) permanecem sem estações meteorológicas e ao todo 164 microregiões seguem sem estações. Isso sugere margem considerável para geração adicional de valor por meio da instalação estratégica de novos equipamentos. No entanto, os resultados indicam que a maximização dos benefícios dessa infraestrutura depende de políticas complementares. Enquanto a cana-de-açúcar se beneficia imediatamente graças à irrigação por salvamento já estabelecida, a difusão de efeitos para culturas como soja, arroz ou milho exige, além da informação meteorológica, o desenvolvimento de sistemas de irrigação ajustados às exigências agronômicas de cada cultura. Uma política integrada, que combine expansão da rede meteorológica com investimentos em infraestrutura hídrica apropriada, tem potencial para ampliar de forma substantiva o retorno social do investimento em informação climática.

A evidência de especificidade do efeito, documentada na Tabela \ref{tab:alternative_outcomes}, reforça essa interpretação. O impacto concentrado na cana-de-açúcar e a ausência de efeitos significativos em culturas alternativas sugerem que políticas de expansão da infraestrutura meteorológica podem ser otimizadas priorizando regiões com maior presença de culturas altamente dependentes de gestão hídrica precisa. Em outras palavras, a informação meteorológica não é um insumo neutro: seu valor econômico depende de quão bem a estrutura produtiva local consegue transformar previsões mais acuradas em decisões de manejo mais eficientes.

Em conjunto, esses resultados indicam que a infraestrutura meteorológica gera retornos econômicos mensuráveis em culturas com manejo complexo e fortemente dependente de informação, como a cana-de-açúcar sob irrigação por salvamento. A especificidade dos efeitos aponta para a importância de incorporar características agronômicas e tecnológicas no planejamento da expansão da rede de estações, de forma a alinhar a localização dos investimentos públicos ao potencial de resposta produtiva de cada região.

\subsection{Mecanismo: Irrigação, Informação Meteorológica e Expansão de Área}

A especificidade do efeito à cana-de-açúcar não é acidental. Ela reflete características agronômicas desta cultura que a tornam particularmente sensível à disponibilidade de informação meteorológica local e atualizada. Esta subseção desenvolve o mecanismo causal proposto, com base nas evidências do Atlas da Irrigação da Agência Nacional de Águas \cite{ana2017atlas} sobre os sistemas de irrigação em cana no Brasil.

\subsubsection{Sistemas de Irrigação em Cana-de-Açúcar no Brasil}

O Atlas da Irrigação trata a cana-de-açúcar de forma específica na estimativa da demanda hídrica nacional, o que reflete a importância estratégica e a complexidade do manejo hídrico desta cultura. O documento distingue três formas de manejo de irrigação para cana:

\begin{enumerate}
\item \textbf{Irrigação plena}: supre praticamente todo o déficit hídrico ao longo do ciclo produtivo, com suspensão da aplicação por volta do décimo mês para favorecer a maturação e o acúmulo de sacarose.

\item \textbf{Irrigação suplementar}: cobre aproximadamente metade do déficit hídrico em períodos críticos, também com interrupção na fase final do ciclo para maturação.

\item \textbf{Irrigação por salvamento}: consiste em aplicações pontuais de lâminas relativamente pequenas de água, da ordem de 20 a 80 mm ao ano, em janelas curtas ou em estágios fenológicos específicos. O objetivo central é evitar perdas irreversíveis de produtividade em situações de déficit hídrico crítico, por exemplo após o corte ou durante veranicos prolongados que ameacem a rebrota.
\end{enumerate}

O Atlas documenta que a \textbf{irrigação por salvamento responde por mais de 90\% da área irrigada de cana-de-açúcar no Brasil}. Essa predominância é resultado de uma escolha econômica e agronômica. Irrigação plena e suplementar, embora representem apenas cerca de 8\% da área irrigada de cana, concentram aproximadamente 43\% da demanda hídrica da cultura, pois se situam em regiões com déficits climáticos mais elevados. Em contraste, o salvamento cobre cerca de 92\% da área irrigada e responde por algo em torno de 57\% da demanda total de água da cana, distribuída em muitas microrregiões com déficit moderado, onde a irrigação contínua não se justificaria economicamente.

O Atlas enfatiza que a quantidade de água aplicada sobre um hectare sob irrigação plena ao longo de um ano equivale aproximadamente ao volume distribuído em quatro hectares sob irrigação suplementar e quinze hectares sob irrigação por salvamento. Na prática, o salvamento é implementado sobretudo com sistemas de carretel enrolador (``hidro roll'') ou pivôs rebocáveis, o que reforça seu caráter móvel e pontual. Esses equipamentos são deslocados entre talhões e acionados apenas quando o déficit hídrico atinge níveis que colocam em risco o potencial produtivo do canavial.

\subsubsection{Informação Meteorológica e Decisões de Timing}

A característica central da irrigação por salvamento, que explica sua conexão com a informação meteorológica, é a \textbf{criticidade do timing de aplicação}. Diferentemente da irrigação plena ou suplementar, que tendem a seguir calendários de aplicação mais estruturados ao longo do ciclo, o salvamento é acionado em poucos momentos decisivos, condicionados ao comportamento efetivo do clima em cada safra.

Em termos operacionais, três dimensões se destacam:

\begin{itemize}
\item \textbf{Momento pós-corte}: após cada colheita anual, a soqueira necessita de umidade adequada para garantir rebrota vigorosa e longevidade do canavial. Em anos com chuvas regulares, essa necessidade é atendida pela precipitação natural. Em anos com veranicos ou estiagens na fase logo após o corte, o produtor enfrenta a decisão de acionar ou não a irrigação de salvamento. Aplicações precoces em excesso podem desperdiçar água ou favorecer doenças, enquanto aplicações tardias comprometem a rebrota e reduzem a produtividade futura.

\item \textbf{Janelas climáticas e risco de veranico}: a decisão de aplicar ou adiar o salvamento depende de forma direta da informação meteorológica. Previsões de chuva nos dias subsequentes, estimativas de evapotranspiração, umidade do solo e indicadores de permanência do veranico entram no cálculo econômico do produtor. Estações locais, ao refinar tanto a medição quanto a previsão para a microrregião, permitem distinguir entre secas momentâneas de baixa gravidade e episódios de déficit hídrico que justificam o custo de ligar o equipamento.

\item \textbf{Otimização logística com equipamentos móveis}: sistemas de carretel ou pivô rebocável atendem vários talhões com capacidade limitada de aplicação em cada janela de tempo. Melhor informação meteorológica local melhora a priorização entre áreas, concentrando a irrigação onde o déficit hídrico é mais severo ou onde a fase fenológica é mais sensível ao estresse.
\end{itemize}

A instalação de estações meteorológicas automáticas altera esse problema de decisão em dois sentidos. Em primeiro lugar, reduz a incerteza sobre o estado atual do clima e do balanço hídrico na escala da microrregião, o que afina o gatilho de acionamento do salvamento. Em segundo lugar, melhora a qualidade das previsões de curtíssimo prazo, o que permite adiar aplicações quando há alta probabilidade de chuvas iminentes e antecipá-las quando a persistência da seca se torna mais provável.

O resultado é um uso mais eficiente da água de irrigação, com maior probabilidade de acionar o salvamento precisamente nos anos e nas janelas em que o estresse hídrico teria efeitos permanentes sobre o canavial. Esse ajuste fino tende a elevar a produtividade média por hectare ao longo do tempo e, ao mesmo tempo, torna economicamente mais atraente a expansão da área de cana em regiões onde a chuva é geralmente suficiente mas sujeita a veranicos que, sem acesso a informação meteorológica de alta qualidade, representariam risco excessivo para novos investimentos em área plantada.

\subsubsection{Por Que Especificidade à Cana-de-Açúcar?}

Os resultados nulos para soja e arroz (Tabela \ref{tab:alternative_outcomes}) são consistentes com este mecanismo e encontram respaldo na literatura agronômica especializada. As diferenças nos sistemas de manejo hídrico entre essas culturas explicam por que a informação meteorológica local gera retornos econômicos mensuráveis especificamente na cana-de-açúcar.

\begin{itemize}
\item \textbf{Arroz irrigado e o efeito tampão da inundação}: O arroz irrigado por inundação opera em um ambiente microclimático artificialmente estabilizado. \citeonline{confalonieri2005} demonstram quantitativamente que a lâmina d'água mantida sobre os arrozais atua como isolante térmico (\textit{thermal buffering}), protegendo o meristema da planta contra flutuações bruscas de temperatura e desidratação que afetariam severamente culturas de sequeiro. Esse efeito tampão significa que o ambiente produtivo do arroz é artificialmente estabilizado pela própria água de inundação, tornando a previsão meteorológica diária menos crítica para a sobrevivência da planta. A decisão hídrica fundamental no arroz irrigado é o estabelecimento e manutenção inicial da lâmina d'água, não ajustes pontuais de timing ao longo do ciclo em resposta a variações climáticas de curto prazo. Consequentemente, a margem de decisão que poderia ser otimizada com informação meteorológica mais precisa é substancialmente menor do que na cana-de-açúcar.

\item \textbf{Soja e a primazia da fenologia sobre o clima imediato}: Na cultura da soja, a irrigação é predominantemente suplementar e guiada por estádios fenológicos críticos, especialmente os estádios reprodutivos R1 a R5 (floração e enchimento de grãos). \citeonline{dogan2007} documentam que a resposta da soja ao déficit hídrico está fortemente condicionada ao estádio fenológico em que ocorre o estresse, sendo os períodos de floração e formação de vagens particularmente sensíveis. Diferentemente da cana, onde a falta de água no salvamento pós-corte pode causar morte da soqueira (perda irreversível do ativo produtivo), na soja a irrigação representa uma decisão de otimização de rendimento marginal baseada no calendário da safra. A decisão de irrigar depende mais do momento fenológico da cultura do que de uma janela climática crítica de poucos dias. Isso confirma a menor elasticidade da decisão de irrigação em relação a um dado meteorológico de ``sobrevivência'', reduzindo o valor marginal da informação climática local de alta frequência.

\item \textbf{Cana-de-açúcar e a criticidade do salvamento pós-corte}: Única entre as grandes culturas brasileiras na predominância da irrigação por salvamento, a cana-de-açúcar apresenta uma janela de decisão extremamente sensível nos dias subsequentes a cada corte anual. Nesse período, a soqueira recém-cortada depende de umidade adequada para garantir rebrota vigorosa; a ausência de irrigação em momento crítico pode comprometer irreversivelmente o potencial produtivo do canavial por múltiplas safras. Essa característica agronômica, combinada com a predominância de equipamentos móveis (carretel enrolador e pivô rebocável) que exigem priorização entre talhões, cria uma demanda por informação meteorológica local e atualizada sem paralelo em outras culturas. A especificidade agronômica traduz-se, assim, em especificidade estatística nos resultados empíricos.
\end{itemize}

O Atlas da Irrigação enfatiza que ``calcular a demanda de água para irrigação é uma tarefa complexa, envolvendo muitas variáveis e equações para determinar as necessidades hídricas diárias da cultura (além da chuva e contribuição do solo) em cada estágio fenológico e sob clima local, mais as perdas entre captação e água efetivamente chegando à planta.'' Esta complexidade, especialmente pronunciada para cana sob salvamento, explica por que informação meteorológica local de alta qualidade gera retornos econômicos mensuráveis especificamente nesta cultura.

\subsection{Limitações e pesquisa futura}
\label{subsec:limitations_future}

Embora os resultados sejam robustos a múltiplas especificações e testes, algumas limitações importantes devem ser reconhecidas.

\subsubsection{Deflator de preços agrícolas}

Uma limitação metodológica relevante diz respeito ao índice utilizado para deflacionar os valores de produção agrícola. Neste estudo, empregamos o IPA-OG-DI Geral (Índice de Preços ao Produtor Amplo, Oferta Global, Disponibilidade Interna) da Fundação Getulio Vargas como deflator dos valores nominais de produção extraídos da Pesquisa Agrícola Municipal.

O procedimento ideal seria utilizar o IPA-OG-DI Agrícola, subíndice específico que captura com maior precisão a evolução dos preços no setor agropecuário e, portanto, permite isolar de forma mais acurada as variações reais de quantidade produzida das variações puramente nominais. Contudo, o acesso a esta série desagregada está restrito a pesquisadores credenciados junto ao sistema FGV Standard, condição não disponível para este trabalho.

A utilização do índice geral introduz potencial ruído na mensuração dos valores reais, uma vez que a trajetória de preços agrícolas pode divergir da inflação geral da economia em determinados períodos. Em particular, choques de oferta agrícola (safras excepcionais ou quebras de produção) e variações nos preços internacionais de commodities podem gerar movimentos nos preços agrícolas não capturados pelo deflator geral. Essa limitação deve ser considerada na interpretação das magnitudes estimadas, embora não comprometa a direção dos efeitos identificados, dado que o viés potencial afeta tratados e controles de forma simétrica dentro de cada período.

\subsubsection{Composição dos pesos e heterogeneidade entre coortes}

O estimador agregado de Callaway e Sant'Anna pondera os efeitos de cada coorte de tratamento pelo número de períodos pós-tratamento observados e pelo tamanho relativo de cada grupo. A análise detalhada dos efeitos por coorte (Apêndice~\ref{chap:did_effects_table}) revela heterogeneidade substantiva nas estimativas pontuais:

\begin{itemize}
    \item as coortes iniciais (2004, 2006, 2008, 2010) apresentam efeitos positivos e de maior magnitude, com ATTs variando entre 0{,}47 e 1{,}02;
    \item as coortes tardias (2015, 2016, 2017) apresentam efeitos próximos de zero ou levemente negativos, com ATTs entre -0{,}01 e 0{,}10;
    \item individualmente, nenhum efeito por coorte atinge significância estatística aos níveis convencionais, embora o efeito agregado seja significativo a 5\%.
\end{itemize}

Essa aparente contradição, em que efeitos individuais não significativos produzem um agregado significativo, decorre do ganho de poder estatístico obtido pela agregação. Os erros-padrão das estimativas por coorte são substancialmente maiores (0{,}08 a 0{,}60) do que o erro-padrão do efeito global (0{,}23), refletindo o menor tamanho amostral de cada subgrupo. A agregação permite que o sinal comum entre coortes emerja, ainda que a imprecisão individual impeça conclusões sobre heterogeneidade.

Essa estrutura de pesos implica que o ATT agregado reflete mais fortemente a experiência das coortes iniciais, que dispõem de mais períodos pós-tratamento para contribuir com a estimação. Caso os efeitos genuínos sejam de fato menores para adotantes tardios (por exemplo, devido a retornos decrescentes à informação meteorológica ou saturação da rede), o efeito agregado pode superestimar o impacto esperado de expansões futuras da infraestrutura.

\subsubsection{Outras limitações importantes}

\begin{itemize}
    \item \textbf{Heterogeneidade não observada}: os efeitos podem variar por características não observadas, como tamanho médio de propriedade, nível educacional dos produtores ou acesso a crédito e assistência técnica. Embora a análise de tendências paralelas (Seção~\ref{sec:parallel_trends}) sugira trajetórias similares no período pré-tratamento, não é possível eliminar completamente esta preocupação.

    \item \textbf{Externalidades espaciais}: a especificação atual pode não capturar completamente benefícios que transbordam para microrregiões vizinhas. Embora a inclusão da densidade estadual de estações mitigue parcialmente esse problema, transbordamentos em escalas espaciais mais finas podem permanecer não identificados.

    \item \textbf{Complementaridades tecnológicas}: a interação com outras tecnologias agrícolas modernas (Landsat, agricultura de precisão, drones, entre outras) não é modelada explicitamente, o que pode levar a uma sobreestimação dos efeitos totais em contextos de adoção tecnológica múltipla.

    \item \textbf{Qualidade e uso efetivo da informação}: a análise assume que a instalação de uma estação implica disponibilidade e uso das informações meteorológicas relevantes mas variações na qualidade dos dados, na manutenção das estações e na capacidade local de interpretação não são observadas diretamente.
\end{itemize}

\subsubsection{Direções para pesquisa futura}

Estudos futuros poderiam expandir e fortalecer os resultados deste trabalho a partir de diferentes frentes:

\begin{itemize}
    \item \textbf{Aprimoramento da definição de suporte e viabilidade da cultura}: embora o recorte da amostra para microrregiões que em algum momento produziram cana-de-açúcar constitua uma estratégia pragmática para mitigar zeros estruturais, ele utiliza a própria decisão produtiva como proxy de viabilidade agroeconômica. Uma extensão natural seria incorporar covariáveis externas de aptidão da cultura, derivadas de informações edafoclimáticas ou de zoneamentos agrícolas, de modo a caracterizar de forma mais explícita a viabilidade estrutural de produzir cana e a avaliar a robustez dos resultados a definições alternativas de suporte amostral. Essa agenda permitiria refinar a distinção entre microrregiões estruturalmente inviáveis e microrregiões viáveis que optaram por não produzir, aproximando ainda mais a identificação dos efeitos locais da instalação das estações meteorológicas.

    \item \textbf{Modelagem espacial explícita}: incorporar dependência espacial e transbordamentos mediante modelos econométricos espaciais, permitindo quantificar com maior precisão os efeitos indiretos das estações sobre microrregiões vizinhas e avaliar o alcance geográfico dos impactos estimados.

    \item \textbf{Dados de maior frequência temporal}: utilizar dados mensais ou trimestrais de produção e clima para capturar com mais detalhe a dinâmica temporal dos efeitos e sua relação com eventos climáticos específicos (secas, enchentes, ondas de calor), bem como com o calendário agrícola de cada cultura.

    \item \textbf{Heterogeneidade por intensidade de irrigação}: investigar se os efeitos variam entre microrregiões com diferentes níveis de adoção de irrigação, em particular irrigação de salvamento, potencialmente utilizando informações do Censo Agropecuário sobre infraestrutura hídrica e práticas de manejo da água.

    \item \textbf{Mecanismos de transmissão}: explorar empiricamente os canais específicos pelos quais a informação meteorológica se traduz em decisões de expansão de área e ajustes de manejo (calibração do timing de irrigação, escolha de variedades, decisões de replantio), possivelmente por meio de estudos de caso, parcerias com produtores e usinas, ou combinações de dados administrativos com levantamentos em campo.
\end{itemize}

\section{Síntese dos Resultados Empíricos}

Os três resultados principais são:

1. Magnitude e Significância do Efeito: O ATT estimado de \mainattpct{} no valor de produção representa impacto econômico substancial. Este efeito é estatisticamente significativo (p < 0,05) e decompõe-se em expansão de área (\areacanaattpct) e ganhos de produtividade, refletindo ajustamentos tanto na margem extensiva quanto intensiva em resposta a melhor informação meteorológica.

2. Dinâmica Temporal dos Impactos: O event study revela ausência de tendências pré-tratamento diferenciadas e efeitos positivos pós-tratamento, sugerindo processos de aprendizado e adaptação tecnológica, não apenas um choque único de produtividade.

3. Robustez e Validade Causal: Os resultados sobrevivem a múltiplos testes de robustez:
\begin{itemize}
\item Ausência de tendências pré-tratamento (validando parallel trends)
\item Teste de randomização múltipla com 5.000 permutações (p-valor empírico < 0,001)
\item Especificidade à cultura: efeito exclusivo ao valor e área de cana-de-açúcar, sem efeitos em culturas alternativas (soja, arroz)
\item Consistência entre diferentes métodos de estimação (DR, IPW, REG: todas significativas)
\item Robustez à escolha do grupo de controle (not-yet-treated vs. never-treated)
\end{itemize}

Estes achados fornecem evidência causal rigorosa sobre o impacto da informação meteorológica na produtividade agrícola, demonstrando que investimentos em infraestrutura de dados climáticos geram retornos econômicos mensuráveis e persistentes.

% ----------------------------------------------------------
% Conclusões Finais
% ----------------------------------------------------------
\chapter{Conclusões Finais}

Este trabalho investigou o impacto causal da instalação de estações meteorológicas sobre o valor de produção de cana-de-açúcar no Brasil, contribuindo para a literatura empírica sobre o papel da informação na tomada de decisões agrícolas e na produtividade do setor. Utilizando dados de satélite do MapBiomas e métodos econométricos de fronteira adequados para contextos de adoção escalonada, demonstramos que o acesso a informações meteorológicas precisas e localizadas gera aumento substancial de \mainattpct{} no valor de produção, decomposto em expansão de área (\areacanaattpct) e ganhos de produtividade.

Os resultados revelam especificidade notável: o efeito é exclusivo à cana-de-açúcar, sem impactos significativos sobre outras culturas (soja, arroz). Esta especificidade valida o mecanismo proposto, fundamentado nas características únicas de manejo hídrico da cana documentadas pelo Atlas da Irrigação: a predominância da irrigação por salvamento (>90\% da área irrigada) cria dependência crítica de timing preciso, que melhor informação meteorológica permite otimizar. A aplicação de filtros crop-specific garantiu comparações válidas, analisando cada cultura apenas entre seus produtores efetivos.

As implicações para políticas públicas são diretas: investimentos em infraestrutura meteorológica devem considerar as características agronômicas das culturas dominantes em cada região. Culturas com manejo complexo e dependente de decisões de timing preciso (como cana-de-açúcar sob irrigação por salvamento) apresentam maior potencial de retorno a investimentos em informação climática. Com 164 microrregiões (29,4\%) ainda sem estações meteorológicas, existe espaço significativo para ganhos adicionais através da expansão estratégica da rede.

Do ponto de vista metodológico, este estudo demonstra a importância de: (i) utilizar métodos adequados para tratamento escalonado, evitando vieses dos estimadores TWFE tradicionais; (ii) testar especificidade do efeito através de variáveis de resultado alternativas; e (iii) fundamentar mecanismos em evidências agronômicas e institucionais. A disponibilização completa do código e dados reforça nosso compromisso com transparência e reprodutibilidade.

Ao quantificar rigorosamente os benefícios da infraestrutura meteorológica e identificar suas características de especificidade, este estudo fornece subsídios para alocação mais eficiente de recursos públicos e privados em informação climática, contribuindo para o desenvolvimento sustentável da agricultura brasileira.

% ----------------------------------------------------------
% ELEMENTOS PÓS-TEXTUAIS
% ----------------------------------------------------------
\postextual
% ----------------------------------------------------------

% ----------------------------------------------------------
% Referências bibliográficas
% ----------------------------------------------------------
% Arquivo de referências bibliográficas
\bibliography{referencias}

% ----------------------------------------------------------
% Glossário
% ----------------------------------------------------------
%\glossary

% ----------------------------------------------------------
% Apêndices
% ----------------------------------------------------------

% ---
% Inicia os apêndices
% ---
\begin{apendicesenv}

% Imprime uma página indicando o início dos apêndices
\partapendices

% ----------------------------------------------------------
\chapter{Código do GitHub}
% ----------------------------------------------------------

O código completo utilizado nesta pesquisa, incluindo os scripts de coleta de dados, análise econométrica e geração de visualizações, está disponível no repositório GitHub:

\url{https://github.com/danielcavalli/tcc-ie-ufrj-2024}

O repositório contém:
\begin{itemize}
\item Scripts SQL para extração de dados do BigQuery
\item Código Python para processamento e limpeza dos dados
\item Scripts R para implementação do modelo de Callaway e Sant'Anna
\item Documentação detalhada dos procedimentos metodológicos
\item Instruções para reprodução dos resultados
\end{itemize}

% ---
\chapter{Estatísticas Descritivas Complementares}
% ---

Este apêndice apresenta estatísticas descritivas complementares que apoiam a análise principal.

\section{Distribuição Temporal do Tratamento}

A Figura \ref{fig:dist_temporal} apresenta a evolução temporal da instalação de estações meteorológicas:

\begin{figure}[h]
\centering
\caption{Distribuição Temporal da Instalação de Estações Meteorológicas}
\label{fig:dist_temporal}
\includegraphics[width=0.85\textwidth]{../../../data/outputs/descriptive_analysis/distribuicao_temporal_tratamento.png}

\textit{Nota:} O gráfico mostra o número de microrregiões que receberam sua primeira estação meteorológica em cada ano. Observa-se uma concentração significativa de instalações no período 2006-2008, coincidindo com programas federais de expansão da rede meteorológica.

\textit{Fonte:} Elaboração própria a partir dos dados do estudo.
\end{figure}

\begin{table}[h]
\centering
\caption{Numero de microrregioes tratadas por ano}
\label{tab:tratamento_temporal}
\begin{tabular}{lcc}
\toprule
Ano & Microrregioes com Primeira Estacao & N Obs \\
\midrule
\treatmenttablerows
\bottomrule
\end{tabular}

\textit{Fonte:} Elaboracao propria a partir dos dados do estudo. Periodo: 2003-2021 (\treatmentnanos{} anos).
\end{table}

\section{Estatísticas por Região}

\begin{table}[h]
\centering
\caption{Distribuição do tratamento por região}
\label{tab:tratamento_regional}
\begin{tabular}{lccc}
\toprule
Região & Microrregiões & Tratadas & \% Tratadas \\
\midrule
Norte & 15 & 8 & 53,3\% \\
Nordeste & 142 & 45 & 31,7\% \\
Centro-Oeste & 51 & 22 & 43,1\% \\
Sudeste & 160 & 48 & 30,0\% \\
Sul & 26 & 8 & 30,8\% \\
\midrule
Total & 558 & -- & -- \\
\bottomrule
\end{tabular}

\textit{Fonte:} Elaboração própria a partir dos dados do estudo.
\end{table}

\section{Produtividade Média por Status de Tratamento}

\begin{table}[h]
\centering
\caption{Produtividade média (ton/ha) por período}
\label{tab:produtividade_status}
\begin{tabular}{lcccc}
\toprule
Período & Nunca Tratadas & Ainda Não Tratadas & Já Tratadas & Diferença \\
\midrule
2000-2007 & 71,2 & 72,8 & 73,5 & 0,7 \\
2008-2014 & 72,5 & 74,3 & 78,9 & 4,6 \\
2015-2021 & 74,1 & 75,2 & 83,7 & 8,5 \\
\bottomrule
\end{tabular}

\textit{Fonte:} Elaboração própria a partir dos dados do estudo.
\end{table}

\section{Análises Descritivas Complementares}

A Figura \ref{fig:evolucao_pib} apresenta a evolução temporal comparativa entre PIB agropecuário e não-agropecuário:

\begin{figure}[h]
\centering
\caption{Evolução Temporal do PIB Agropecuário vs PIB Não-Agropecuário}
\label{fig:evolucao_pib}
\includegraphics[width=0.85\textwidth]{../../../data/outputs/descriptive_analysis/evolucao_temporal_pib.png}

\textit{Nota:} O gráfico mostra a evolução temporal do PIB agropecuário e do PIB não-agropecuário médio (em log) para as microrregiões da amostra. A comparação permite visualizar as dinâmicas distintas entre os setores ao longo do período analisado.

\textit{Fonte:} Elaboração própria a partir dos dados do estudo.
\end{figure}

A Figura \ref{fig:correlacao} apresenta a matriz de correlação entre as principais variáveis utilizadas no estudo:

\begin{figure}[h]
\centering
\caption{Matriz de Correlação das Variáveis Principais}
\label{fig:correlacao}
\includegraphics[width=0.85\textwidth]{../../../data/outputs/descriptive_analysis/matriz_correlacao.png}

\textit{Nota:} A matriz mostra as correlações entre PIB agropecuário, área plantada, população, precipitação e outras variáveis relevantes. Valores mais próximos de 1 indicam correlação positiva forte.

\textit{Fonte:} Elaboração própria a partir dos dados do estudo.
\end{figure}



% ----------------------------------------------------------
\chapter{Estudos de Evento por Outcome}
\label{chap:event_studies_appendix}
% ----------------------------------------------------------

Este apendice apresenta os graficos de estudo de evento (event study) para todos os outcomes analisados neste trabalho. Alem do outcome principal (valor de producao de cana-de-acucar), sao apresentados os resultados para area plantada e valor de producao de outras culturas, que servem como testes de especificidade e robustez.

\section{Area Plantada de Cana-de-acucar}

A Figura \ref{fig:event_area_cana} apresenta o estudo de evento para a area plantada de cana-de-acucar, representando a margem extensiva do efeito das estacoes meteorologicas.

\begin{figure}[H]
\centering
\caption{Event Study: Impacto das Estacoes Meteorologicas na Area Plantada de Cana}
\label{fig:event_area_cana}
\includegraphics[width=0.85\textwidth]{../../../data/outputs/presentation/event_study_area_cana_enhanced.png}

\textit{Nota:} A figura apresenta as estimativas pontuais e intervalos de confianca de 95\% dos efeitos do tratamento em funcao do tempo relativo a instalacao da estacao. Pontos azuis indicam efeitos estatisticamente significativos ao nivel de 5\%.

\textit{Fonte:} Elaboracao propria.
\end{figure}

\section{Testes de Especificidade: Outras Culturas}

As figuras a seguir apresentam os estudos de evento para outras culturas (soja e arroz), que servem como testes de especificidade. Espera-se que o efeito das estacoes meteorologicas seja menor ou nao significativo para culturas menos dependentes de irrigacao e manejo climatico preciso.

\subsection{Area Plantada de Soja}

\begin{figure}[H]
\centering
\caption{Event Study: Impacto das Estacoes Meteorologicas na Area Plantada de Soja}
\label{fig:event_area_soja}
\includegraphics[width=0.85\textwidth]{../../../data/outputs/presentation/event_study_area_soja_enhanced.png}

\textit{Nota:} A figura apresenta as estimativas pontuais e intervalos de confianca de 95\% dos efeitos do tratamento em funcao do tempo relativo a instalacao da estacao. Pontos azuis indicam efeitos estatisticamente significativos ao nivel de 5\%.

\textit{Fonte:} Elaboracao propria.
\end{figure}

\subsection{Area Plantada de Arroz}

\begin{figure}[H]
\centering
\caption{Event Study: Impacto das Estacoes Meteorologicas na Area Plantada de Arroz}
\label{fig:event_area_arroz}
\includegraphics[width=0.85\textwidth]{../../../data/outputs/presentation/event_study_area_arroz_enhanced.png}

\textit{Nota:} A figura apresenta as estimativas pontuais e intervalos de confianca de 95\% dos efeitos do tratamento em funcao do tempo relativo a instalacao da estacao. Pontos azuis indicam efeitos estatisticamente significativos ao nivel de 5\%.

\textit{Fonte:} Elaboracao propria.
\end{figure}

\subsection{Valor de Producao de Soja}

\begin{figure}[H]
\centering
\caption{Event Study: Impacto das Estacoes Meteorologicas no Valor de Producao de Soja}
\label{fig:event_valor_soja}
\includegraphics[width=0.85\textwidth]{../../../data/outputs/presentation/event_study_valor_producao_soja_enhanced.png}

\textit{Nota:} A figura apresenta as estimativas pontuais e intervalos de confianca de 95\% dos efeitos do tratamento em funcao do tempo relativo a instalacao da estacao. Pontos azuis indicam efeitos estatisticamente significativos ao nivel de 5\%.

\textit{Fonte:} Elaboracao propria.
\end{figure}

\subsection{Valor de Producao de Arroz}

\begin{figure}[H]
\centering
\caption{Event Study: Impacto das Estacoes Meteorologicas no Valor de Producao de Arroz}
\label{fig:event_valor_arroz}
\includegraphics[width=0.85\textwidth]{../../../data/outputs/presentation/event_study_valor_producao_arroz_enhanced.png}

\textit{Nota:} A figura apresenta as estimativas pontuais e intervalos de confianca de 95\% dos efeitos do tratamento em funcao do tempo relativo a instalacao da estacao. Pontos azuis indicam efeitos estatisticamente significativos ao nivel de 5\%.

\textit{Fonte:} Elaboracao propria.
\end{figure}

% ----------------------------------------------------------
\chapter{Tabela Detalhada de Efeitos DiD}
\label{chap:did_effects_table}
% ----------------------------------------------------------

Este apendice apresenta a tabela detalhada dos efeitos estimados pelo metodo de Diferencas-em-Diferencas escalonado de Callaway e Sant'Anna (2021). A tabela inclui tres tipos de efeitos: (i) o efeito medio global (ATT), (ii) os efeitos por coorte de tratamento, e (iii) os efeitos dinamicos em tempos selecionados relativos ao tratamento.

\section{Efeitos para Valor de Producao de Cana}

A Tabela \ref{tab:did_effects_valor} apresenta os efeitos estimados para o outcome principal: valor de producao de cana-de-acucar (em logaritmo).

\begin{table}[H]
\centering
\caption{Efeitos DiD Detalhados: Valor de Producao de Cana}
\label{tab:did_effects_valor}
\small
\begin{tabular}{llrrrrl}
\toprule
\textbf{Tipo} & \textbf{Grupo/Evento} & \textbf{ATT} & \textbf{EP} & \textbf{IC 95\%} & \textbf{p-valor} & \textbf{Sig.} \\
\midrule
\multicolumn{7}{l}{\textit{Efeito Global}} \\
& ATT Medio & \didvalorglobalatt & \didvalorglobalse & [\didvalorgloballower; \didvalorglobalupper] & \didvalorglobalp & \didvalorglobalsig \\
\midrule
\multicolumn{7}{l}{\textit{Efeitos por Coorte}} \\
& Coorte \didvalorcoorteayear & \didvalorcoorteaatt & \didvalorcoortease & [\didvalorcoortealower; \didvalorcoorteaupper] & \didvalorcoorteap & \didvalorcoorteasig \\
& Coorte \didvalorcoortebyear & \didvalorcoortebatt & \didvalorcoortebse & [\didvalorcoorteblower; \didvalorcoortebupper] & \didvalorcoortebp & \didvalorcoortebsig \\
& Coorte \didvalorcoortecyear & \didvalorcoortecatt & \didvalorcoortecse & [\didvalorcoorteclower; \didvalorcoortecupper] & \didvalorcoortecp & \didvalorcoortecsig \\
& Coorte \didvalorcoortedyear & \didvalorcoortedatt & \didvalorcoortedse & [\didvalorcoortedlower; \didvalorcoortedupper] & \didvalorcoortedp & \didvalorcoortedsig \\
& Coorte \didvalorcoorteeyear & \didvalorcoorteeatt & \didvalorcoorteese & [\didvalorcoorteelower; \didvalorcoorteeupper] & \didvalorcoorteep & \didvalorcoorteesig \\
& Coorte \didvalorcoortefyear & \didvalorcoortefatt & \didvalorcoortefse & [\didvalorcoorteflower; \didvalorcoortefupper] & \didvalorcoortefp & \didvalorcoortefsig \\
& Coorte \didvalorcoortegyear & \didvalorcoortegatt & \didvalorcoortegse & [\didvalorcoorteglower; \didvalorcoortegupper] & \didvalorcoortegp & \didvalorcoortegsig \\
& Coorte \didvalorcoortehyear & \didvalorcoortehatt & \didvalorcoortehse & [\didvalorcoortehlower; \didvalorcoortehupper] & \didvalorcoortehp & \didvalorcoortehsig \\
\midrule
\multicolumn{7}{l}{\textit{Efeitos Dinamicos}} \\
& t-5 & \didvalortminusfiveatt & \didvalortminusfivese & [\didvalortminusfivelower; \didvalortminusfiveupper] & \didvalortminusfivep & \didvalortminusfivesig \\
& t-3 & \didvalortminusthreeatt & \didvalortminusthreese & [\didvalortminusthreelower; \didvalortminusthreeupper] & \didvalortminusthreep & \didvalortminusthreesig \\
& t-1 & \didvalortminusoneatt & \didvalortminusonese & [\didvalortminusonelower; \didvalortminusoneupper] & \didvalortminusonep & \didvalortminusonesig \\
& t+0 & \didvalortpluszeroatt & \didvalortpluszerose & [\didvalortpluszerolower; \didvalortpluszeroupper] & \didvalortpluszerop & \didvalortpluszerosig \\
& t+1 & \didvalortplusoneatt & \didvalortplusonese & [\didvalortplusonelower; \didvalortplusoneupper] & \didvalortplusonep & \didvalortplusonesig \\
& t+2 & \didvalortplustwoatt & \didvalortplustwose & [\didvalortplustwolower; \didvalortplustwoupper] & \didvalortplustwop & \didvalortplustwosig \\
& t+3 & \didvalortplusthreeatt & \didvalortplusthreese & [\didvalortplusthreelower; \didvalortplusthreeupper] & \didvalortplusthreep & \didvalortplusthreesig \\
& t+5 & \didvalortplusfiveatt & \didvalortplusfivese & [\didvalortplusfivelower; \didvalortplusfiveupper] & \didvalortplusfivep & \didvalortplusfivesig \\
& t+10 & \didvalortplusonezeroatt & \didvalortplusonezerose & [\didvalortplusonezerolower; \didvalortplusonezeroupper] & \didvalortplusonezerop & \didvalortplusonezerosig \\
\bottomrule
\end{tabular}

\textit{Nota:} ATT = Efeito Medio do Tratamento sobre os Tratados. EP = Erro Padrao. IC 95\% = Intervalo de Confianca simultaneo de 95\%. Significancia: *** p<0.01, ** p<0.05, * p<0.1. Os efeitos dinamicos mostram tempos selecionados relativos ao tratamento (t-k = k periodos antes, t+k = k periodos apos).

\textit{Fonte:} Elaboracao propria com base no estimador de Callaway e Sant'Anna (2021).
\end{table}

\section{Efeitos para Area Plantada de Cana}

A Tabela \ref{tab:did_effects_area} apresenta os efeitos estimados para o outcome secundario: area plantada de cana-de-acucar (em logaritmo).

\begin{table}[H]
\centering
\caption{Efeitos DiD Detalhados: Area Plantada de Cana}
\label{tab:did_effects_area}
\small
\begin{tabular}{llrrrrl}
\toprule
\textbf{Tipo} & \textbf{Grupo/Evento} & \textbf{ATT} & \textbf{EP} & \textbf{IC 95\%} & \textbf{p-valor} & \textbf{Sig.} \\
\midrule
\multicolumn{7}{l}{\textit{Efeito Global}} \\
& ATT Medio & \didareaglobalatt & \didareaglobalse & [\didareagloballower; \didareaglobalupper] & \didareaglobalp & \didareaglobalsig \\
\midrule
\multicolumn{7}{l}{\textit{Efeitos por Coorte}} \\
& Coorte \didareacoorteayear & \didareacoorteaatt & \didareacoortease & [\didareacoortealower; \didareacoorteaupper] & \didareacoorteap & \didareacoorteasig \\
& Coorte \didareacoortebyear & \didareacoortebatt & \didareacoortebse & [\didareacoorteblower; \didareacoortebupper] & \didareacoortebp & \didareacoortebsig \\
& Coorte \didareacoortecyear & \didareacoortecatt & \didareacoortecse & [\didareacoorteclower; \didareacoortecupper] & \didareacoortecp & \didareacoortecsig \\
& Coorte \didareacoortedyear & \didareacoortedatt & \didareacoortedse & [\didareacoortedlower; \didareacoortedupper] & \didareacoortedp & \didareacoortedsig \\
& Coorte \didareacoorteeyear & \didareacoorteeatt & \didareacoorteese & [\didareacoorteelower; \didareacoorteeupper] & \didareacoorteep & \didareacoorteesig \\
& Coorte \didareacoortefyear & \didareacoortefatt & \didareacoortefse & [\didareacoorteflower; \didareacoortefupper] & \didareacoortefp & \didareacoortefsig \\
& Coorte \didareacoortegyear & \didareacoortegatt & \didareacoortegse & [\didareacoorteglower; \didareacoortegupper] & \didareacoortegp & \didareacoortegsig \\
& Coorte \didareacoortehyear & \didareacoortehatt & \didareacoortehse & [\didareacoortehlower; \didareacoortehupper] & \didareacoortehp & \didareacoortehsig \\
\midrule
\multicolumn{7}{l}{\textit{Efeitos Dinamicos}} \\
& t-5 & \didareatminusfiveatt & \didareatminusfivese & [\didareatminusfivelower; \didareatminusfiveupper] & \didareatminusfivep & \didareatminusfivesig \\
& t-3 & \didareatminusthreeatt & \didareatminusthreese & [\didareatminusthreelower; \didareatminusthreeupper] & \didareatminusthreep & \didareatminusthreesig \\
& t-1 & \didareatminusoneatt & \didareatminusonese & [\didareatminusonelower; \didareatminusoneupper] & \didareatminusonep & \didareatminusonesig \\
& t+0 & \didareatpluszeroatt & \didareatpluszerose & [\didareatpluszerolower; \didareatpluszeroupper] & \didareatpluszerop & \didareatpluszerosig \\
& t+1 & \didareatplusoneatt & \didareatplusonese & [\didareatplusonelower; \didareatplusoneupper] & \didareatplusonep & \didareatplusonesig \\
& t+2 & \didareatplustwoatt & \didareatplustwose & [\didareatplustwolower; \didareatplustwoupper] & \didareatplustwop & \didareatplustwosig \\
& t+3 & \didareatplusthreeatt & \didareatplusthreese & [\didareatplusthreelower; \didareatplusthreeupper] & \didareatplusthreep & \didareatplusthreesig \\
& t+5 & \didareatplusfiveatt & \didareatplusfivese & [\didareatplusfivelower; \didareatplusfiveupper] & \didareatplusfivep & \didareatplusfivesig \\
& t+10 & \didareatplusonezeroatt & \didareatplusonezerose & [\didareatplusonezerolower; \didareatplusonezeroupper] & \didareatplusonezerop & \didareatplusonezerosig \\
\bottomrule
\end{tabular}

\textit{Nota:} ATT = Efeito Medio do Tratamento sobre os Tratados. EP = Erro Padrao. IC 95\% = Intervalo de Confianca simultaneo de 95\%. Significancia: *** p<0.01, ** p<0.05, * p<0.1. Os efeitos dinamicos mostram tempos selecionados relativos ao tratamento (t-k = k periodos antes, t+k = k periodos apos).

\textit{Fonte:} Elaboracao propria com base no estimador de Callaway e Sant'Anna (2021).
\end{table}

\section{Interpretacao dos Resultados}

Os resultados detalhados nas tabelas acima permitem algumas observacoes importantes:

\begin{enumerate}
\item \textbf{Validacao das Tendencias Paralelas}: Os efeitos dinamicos pre-tratamento (t-5, t-3, t-1) nao sao estatisticamente significativos para ambos os outcomes, validando a hipotese de tendencias paralelas necessaria para a identificacao causal.

\item \textbf{Heterogeneidade por Coorte}: Os efeitos variam consideravelmente entre as coortes de tratamento. As coortes mais antigas (2004, 2006, 2010) apresentam efeitos maiores, possivelmente devido ao maior tempo de exposicao ao tratamento.

\item \textbf{Persistencia dos Efeitos}: Os efeitos dinamicos mostram crescimento ao longo do tempo pos-tratamento, com efeitos significativos emergindo a partir de t+5 e se intensificando em t+10, sugerindo que os beneficios das estacoes meteorologicas se acumulam ao longo do tempo.

\item \textbf{Consistencia entre Outcomes}: O padrao dos efeitos e similar entre valor de producao e area plantada, embora com magnitudes diferentes, sugerindo que o impacto ocorre tanto na margem extensiva (area) quanto na intensiva (produtividade).
\end{enumerate}

\end{apendicesenv}

%---------------------------------------------------------------------
% INDICE REMISSIVO
%---------------------------------------------------------------------
\phantompart
\printindex
%---------------------------------------------------------------------

\end{document}
